\documentclass[11pt]{article} % JASA requires 12 pt font for manuscripts
%\usepackage{JASA_manu}        % For JASA manuscript formatting

% for citations
\usepackage[authoryear]{natbib} % natbib required for JASA
\usepackage[colorlinks=true, citecolor=blue, linkcolor=blue]{hyperref}
\newcommand{\citetapos}[1]{\citeauthor{#1}{\textcolor{blue}{'s}} }

%\definecolor{Blue}{rgb}{0,0,0.5}

\usepackage{amsthm}

% for figures
\usepackage{graphicx}
\usepackage[font=small]{caption}
\usepackage{subfig}
\captionsetup[subfloat]{font=normalsize}
%\usepackage{subcaption}
\graphicspath{{figures/}}
\newcommand{\hh}[1]{{\color{orange} #1}}
\newcommand{\al}[1]{{\color{red} #1}}

% color in tables
\usepackage{rotating}
\usepackage{color}
\usepackage{colorbl}

% help with editing and coauthoring
\usepackage{todonotes}

% title formatting
\usepackage[compact,small]{titlesec}
% page formatting
\usepackage[margin = 1in]{geometry}
%\usepackage[parfill]{parskip}

% line spacing
\usepackage{setspace}
%\doublespace

% For math typsetting
\usepackage{bm}
\usepackage{amstext}
\usepackage{amssymb}
\usepackage{amsmath}
\usepackage{amsfonts}
\usepackage{multirow}

\newtheorem{proposition}{Proposition}
\newtheorem{theorem}{Theorem}
\newtheorem{definition}{Definition}
\newtheorem{algorithm}[theorem]{Algorithm}

% A few commands to make typing less tedious
\newcommand{\inv}{\ensuremath{^{-1}}}
\newcommand{\ginv}{\ensuremath{^{-}}}
\newcommand{\trans}{\ensuremath{^\prime}}
\newcommand{\E}{\ensuremath{\mathrm{E}}}
\newcommand{\var}{\ensuremath{\mathrm{Var}}}
\newcommand{\cov}{\ensuremath{\mathrm{Cov}}}
\DeclareMathOperator{\tr}{Trace}
\DeclareMathOperator{\rank}{rank}
\DeclareMathOperator*{\argmin}{arg\, min}

\title{Supplement to ``Are you Normal? The Problem of Confounded Residual Structures in Hierarchical Linear Models''}
\author{
	Adam Loy and Heike Hofmann\\
	Department of Statistics,
	Iowa State University
}

\begin{document}
\maketitle
%----------------------------------------------------------------------------------

The materials in this document supplement the information presented in ``Are you Normal? The Problem of Confounded Residual Structures in Hierarchical Linear Models''. Section~\ref{supp:evals} presents a simulation study evaluating the performance of existing proposals for residual analysis for hierarchical linear models. Section~\ref{supp:simstudy} presents the complete results for all simulation settings considered in the paper. 


\section{Evaluations of existing proposals}\label{supp:evals}
%--------------------------------------------------

\subsection{Model notation}
%--------------------------
Recall that the stacked representation of the hierarchical linear model is given by 
%
\begin{align}\label{eq:hlm}
	\underset{(n \times 1)}{\bm{y}} &= \underset{(n_i \times p)}{\bm{X}} \ \underset{(p \times 1)}{\bm{\beta}} + \underset{(n \times q)}{\bm{Z}} \ \underset{(q \times 1)}{\bm{b}} + \underset{(n \times 1)}{\bm{\varepsilon}},\\
	\bm{\varepsilon} & \overset{\text{iid}}{\sim} \mathcal{N}(\bm{0}, \ \bm{R}), \qquad \bm{b} \overset{\text{iid}}{\sim} \mathcal{N}(\bm{0},\ \bm{D}) \nonumber
\end{align}
%
where $\bm{y}$ is a vector of responses, $\bm{X}$ and $\bm{Z}$ are design matrices for the fixed and random effects, respectively, $\bm{\beta}$ is a vector of fixed effects, $\bm{b}$ is a vector of random effects, $\bm{\varepsilon}$ is a vector of error terms, and $\bm{R}$ and $\bm{D}$ are positive definite covariance matrices. Further, we assume that $\cov(\bm{\varepsilon},\ \bm{b}) = \bm{0}$. The above assumptions imply that, marginally, $\bm{y} \sim \mathcal{N}(\bm{X\beta},\ \bm{V})$ where $\bm{V} = \bm{ZDZ}\trans$.

\subsection{Residuals}

In this section we consider residuals that are commonly used to check the distributional assumptions in a hierarchical linear model. For more general discussions of residual analysis for hierarchical linear models we refer the reader to \cite{Haslett:2007vv} and \cite{Nobre:2007ej}.

\paragraph{Marginal residuals.} 
%-----------------------------
The marginal distribution of $\bm{y}$ leads to the marginal residuals which are defined as
%
\begin{equation}\label{eq:marginalresid}
\widehat{\bm{\zeta}}  = \bm{y} - \bm{X} \widehat{\bm{\beta}} =  \bm{V P y}
\end{equation}
%
where $ \bm{P} = \bm{V\inv} - \bm{V\inv X} \left( \bm{X\trans V\inv X} \right) \bm{X \trans V\inv}$, which reveal how the observations deviate from the global trend. The use of these residuals for distributional assessment provides an omnibus assessment of goodness-of-fit as the marginal residuals are a linear combination of the other residual quantities; however, this assessment requires the empirical distribution of the marginal residuals to resemble true distribution. Asymptotically, the variance of the marginal residuals is $\var(\widehat{\zeta}) = \bm{V}$ leading to correlated residuals. To obtain asymptotically uncorrelated residuals the marginal residuals can be scaled by the Cholesky root of $\bm{V}$ \citep{Houseman:2004gq}, $\bm{C}$, yielding
%
\begin{equation}\label{eq:choleskyresid}
\bm{z}_{\zeta}  = \bm{C}\inv \widehat{\bm{\zeta}}
\end{equation}
%


\paragraph{Level-1 residuals.}
%-----------------------------
The distribution of $\bm{y}$ conditional on the random effects, $\bm{b}$, is given by
%
\begin{equation}\label{eq:marginalmodel}
	\bm{y} | \bm{b} \sim \mathcal{N}( \bm{X\beta} + \bm{Zb},\ \bm{R} ),
\end{equation}
%
and leads to the level-1 residuals, commonly referred to as the error terms, which are defined as
%
\begin{equation}\label{eq:lev1resid}
	\widehat{\bm{\varepsilon}} = \bm{y} - \bm{X \widehat{\beta}} + \bm{Z \widehat{b}} = \bm{R P y}
\end{equation}
%
and reveal the deviations of the observations from the conditional model. The variance of the level-1 residuals is given by $\var(\widehat{\bm{\varepsilon}}) = \bm{ R P R }$, so studentized level-1 residuals can be obtained by 
%
\begin{equation}\label{eq:lev1-std}
\bm{z}_{\varepsilon} =  \text{diag} \left(\bm{RPR} \right)^{-1/2} \widehat{\bm{\varepsilon}}
\end{equation}
%
which have been recommended for distributional assessment \citep{Nobre:2007ej}. An alternative approach is recommended by \citet[Section 4.3]{Pinhiero:2000vf}, who suggest use of the Pearson residuals, which are obtained by dividing the predicted residuals by the estimated within-group standard deviation, $\widehat{\sigma}_\varepsilon$. 



\paragraph{Level-2 residuals.}
%-----------------------------
The final type of residual we consider is the the best linear unbiased predictor (BLUP) of the random effects (i.e., predicted random effects), providing insight into the differences between the marginal (global) and conditional models. By definition, the BLUP of $\bm{b}$ is
%
\begin{equation}\label{eq:lev2resid}
\widehat{\bm{b}} = \bm{D Z\trans V\inv} \left( \bm{y} - \bm{X \widehat{\beta}} \right) = \bm{D Z\trans P y}
\end{equation}
%
which has variance $\var(\widehat{\bm{b}}) = \bm{DZ\trans P ZD}$. 
%Rewriting $\var(\widehat{\bm{b}})$ as
%%
%\begin{equation}
%\bm{DZ\trans P ZD} = \bm{DZ\trans} \bm{V\inv} \left( \bm{V} - \bm{ X} \left( \bm{X\trans V\inv X} \right) \bm{X \trans} \right) \bm{V\inv} \bm{ZD}
%\end{equation}
%%
%leads to two 
For distributional assessment of the BLUPs it makes sense to examine each random effect individually, though \cite{Lange:1989uu} suggest the examination of linear combinations of standardized BLUPs. Rewritting the definition of $\var(\widehat{\bm{b}})$
%
\begin{equation}
\bm{DZ\trans P ZD} = \bm{DZ\trans} \bm{V\inv} \left( \bm{V} - \bm{ X} \left( \bm{X\trans V\inv X} \right) \bm{X \trans} \right) \bm{V\inv} \bm{ZD}
\end{equation}
%
leads to two similar standardizations of the BLUPs. The first utilizes the fact that when the number of groups is large $\bm{ X} \left( \bm{X\trans V\inv X} \right)$ will be small \citep{Goldstein:2011}, so for a large number of groups standardized BLUPs can be calculated by
%
\begin{equation}\label{eq:lev2-std1}
\bm{z}_{b} = \text{diag} \left(\bm{DZ\trans V\inv ZD}\right)^{-1/2} \widehat{\bm{b}}
\end{equation}
%
This formulation is the same used by \cite{Lange:1989uu} (discussed below).
%, who suggest creating weighted Q-Q plots \citep{Dempster:1985tr} to assess the distributional assumptions. Note that standardized linear combinations of the BLUPs follow in the usual way. 
The second standardization applies for all sample sizes and is given by
%
\begin{equation}\label{eq:lev2-std2}
\bm{z}_{b} = \text{diag} \left(\bm{DZ\trans P ZD}\right)^{-1/2} \widehat{\bm{b}}
\end{equation}
%


\subsection{Weighted Q-Q plots}
%------------------------------

As an alternative to Q-Q plots constructed from the BLUPs \cite{Lange:1989uu} propose using weighted Q-Q plots of standardized linear combinations of the BLUPs, $\bm{C\trans}\widehat{\bm{b}}$,
%
\begin{equation}
z_{b} = \text{diag} \left(\bm{C\trans DZ\trans V\inv ZDC}\right)^{-1/2} \bm{C\trans}\widehat{\bm{b}}
\end{equation}
%
The specific form of $\bm{C}$ chosen highlights different departures from distributional assumptions---for example, $\bm{C}$s can be chosen to extract the random slope and the random intercept terms individually. When the random effects may be correlated, \citeauthor{Lange:1989uu} suggest examining a range of additional linear combinations in-between the two marginal random effects either through manual specification of $\bm{C}$ or projection pursuit.
After choosing $\bm{C}$ a weighted Q-Q plot is constructed by comparing the weighted empirical cumulative distribution function
%
\begin{equation}
F_m^*(x) = \sum_{i=1}^{m} I(x - z_{b_i} \geq 0) w_i \bigg/ \sum_{i=1}^{m} w_i, 
\end{equation}
%
to $\Phi\inv \left ( F_m^*(z_{b_i}) \right)$. Here, $w_i$ is the $i$th element of $\bm{C\trans DZ\trans V\inv ZDC}$.
For balanced group sizes this simplifies to the unweighted Q-Q plot of $\bm{z}_b$.
%

\subsection{Simulation-based approaches}
%---------------------------------------

All of the above approaches to checking the distributional assumptions rely on the use of interrelated residuals, which has been reported to be problematic \citep{HildenMinton:1995wh, Verbeke:1996va}.  
One alternative that has been proposed to overcome this problem is the use of the parametric bootstrap to develop point-wise and simultaneous confidence bands for Q-Q plots. We evaluate the potential of this method using bootstrap tests of normality.


\subsection{Simulation study}\label{supp:simstudy}
%--------------------------------------------------

To evaluate the above proposals we carried out a simulation study under the same settings as in the paper, with the only difference being that the original $\bm{Z}$ was used for data generation. To evaluate the bootstrap tests of normality, a null distribution of 5000 simulated test statistics for each situation was used.

Tables~\ref{tab:eval1}--\ref{tab:evalmarginal} present the results of using standard normality tests to assess the distributional assumptions of the residuals from a hierarchical model. The gray background on the table indicates which simulation settings estimate type I error, with the other rows estimating power. Tables~\ref{tab:boot1}--\ref{tab:bootmarginal} present the results of the bootstrap tests for normality. Table~\ref{tab:langeryan} presents the results of using a weighted CDF to evaluate the normality of the random effects, in this case the null distribution was obtained using the parametric bootstrap.

Based on the simulation results it is clear that none of the residual-based diagnostics for assessing distributional assumptions are appropriate in all situations. The error terms can be targeted either by the use of studentized residuals or a parametric bootstrap; however, the assessment of this assumption is less critical. The random effects, on which predictive inference relies, cannot be targeted by the current methods when the residual variance is larger than the variance component associated with the random effects---that is, situations with higher degrees of shrinkage.  Such situations are often encountered in practice. Additionally, use of the parametric bootstrap---to construct simulation envelopes for Q-Q plots, for example---does not appear to remedy this situation based on the performance of the bootstrap tests. Finally, we have shown \citeauthor{Lange:1989uu}'s weighted Q-Q plots cannot target the random effects distribution when the residual variance is large, as the distribution of the error terms overly influences tests for the random slope, resulting in inflated type I error rates for both random effects.

%&&&&&&&&&&&&&&&&&&&&&&&&&&&&&&&&&&&&&&&&&&&&&&&&&&&&&&&&&&&&&&&&&&&&&&&&&&&&&&
% Correct model
%&&&&&&&&&&&&&&&&&&&&&&&&&&&&&&&&&&&&&&&&&&&&&&&&&&&&&&&&&&&&&&&&&&&&&&&&&&&&&&


%------------------------------------------------
% Naive tests
%------------------------------------------------

%%% Level-1 residuals
\begin{table}[ht]
\centering
\caption{\label{tab:eval1} Proportion of tests rejecting the null hypothesis of normality of the error terms.}
\begin{scriptsize}
\begin{tabular}{ll p{.1cm} c p{.1cm} rrr p{.1cm} rrr p{.1cm} rrr}
  \hline
  \multicolumn{2}{c}{Distributions}& & Nominal & &  \multicolumn{3}{c}{Raw residuals} & & \multicolumn{3}{c}{Pearson residuals} & & \multicolumn{3}{c}{Studentized residuals}\\ \cline{1-2} \cline{6-8} \cline{10-12} \cline{14-16}
  Errors & Random effects & & $\alpha$ & & AD & CVM & KS & & AD & CVM & KS & & AD & CVM & KS \\ 
   \hline
& && && \multicolumn{9}{c}{$\sigma_{\varepsilon}^2 = 4$, \ \ $\sigma_{b_0}^2 = \sigma_{b_1}^2 = 1$} \\ \cline{6-16}

\rowcolor{gray!20} Normal       & Normal       && 0.05 && 0.07 & 0.06 & 0.06 && 0.07 & 0.06 & 0.06 && 0.04 & 0.04 & 0.04 \\ 
\rowcolor{gray!20}             &              && 0.10 && 0.13 & 0.12 & 0.12 && 0.13 & 0.12 & 0.12 && 0.09 & 0.09 & 0.10 \\ 
\rowcolor{gray!20}             & Heavy tailed && 0.05 && 0.07 & 0.08 & 0.06 && 0.07 & 0.08 & 0.06 && 0.05 & 0.05 & 0.05 \\ 
\rowcolor{gray!20}             &              && 0.10 && 0.14 & 0.13 & 0.14 && 0.14 & 0.13 & 0.14 && 0.11 & 0.11 & 0.10 \\ 
\rowcolor{gray!20}             & Skewed       && 0.05 && 0.07 & 0.06 & 0.06 && 0.07 & 0.06 & 0.06 && 0.04 & 0.04 & 0.05 \\ 
\rowcolor{gray!20}             &              && 0.10 && 0.13 & 0.12 & 0.13 && 0.13 & 0.12 & 0.13 && 0.09 & 0.09 & 0.10 \\ 
             &&&&&&&&&&&&&&&\\
Heavy tailed & Normal       && 0.05 && 1.00 & 1.00 & 1.00 && 1.00 & 1.00 & 1.00 && 1.00 & 1.00 & 1.00 \\ 
             &              && 0.10 && 1.00 & 1.00 & 1.00 && 1.00 & 1.00 & 1.00 && 1.00 & 1.00 & 1.00 \\ 
             & Heavy tailed && 0.05 && 1.00 & 1.00 & 1.00 && 1.00 & 1.00 & 1.00 && 1.00 & 1.00 & 1.00 \\ 
             &              && 0.10 && 1.00 & 1.00 & 1.00 && 1.00 & 1.00 & 1.00 && 1.00 & 1.00 & 1.00 \\ 
             & Skewed       && 0.05 && 1.00 & 1.00 & 1.00 && 1.00 & 1.00 & 1.00 && 1.00 & 1.00 & 1.00 \\ 
             &              && 0.10 && 1.00 & 1.00 & 1.00 && 1.00 & 1.00 & 1.00 && 1.00 & 1.00 & 1.00 \\ 
             &&&&&&&&&&&&&&&\\
Skewed       & Normal       && 0.05 && 1.00 & 1.00 & 1.00 && 1.00 & 1.00 & 1.00 && 1.00 & 1.00 & 1.00 \\ 
             &              && 0.10 && 1.00 & 1.00 & 1.00 && 1.00 & 1.00 & 1.00 && 1.00 & 1.00 & 1.00 \\ 
             & Heavy tailed && 0.05 && 1.00 & 1.00 & 1.00 && 1.00 & 1.00 & 1.00 && 1.00 & 1.00 & 1.00 \\ 
             &              && 0.10 && 1.00 & 1.00 & 1.00 && 1.00 & 1.00 & 1.00 && 1.00 & 1.00 & 1.00 \\ 
             & Skewed       && 0.05 && 1.00 & 1.00 & 1.00 && 1.00 & 1.00 & 1.00 && 1.00 & 1.00 & 1.00 \\ 
             &              && 0.10 && 1.00 & 1.00 & 1.00 && 1.00 & 1.00 & 1.00 && 1.00 & 1.00 & 1.00 \\ 

&&&&&&&&&&&&&&&\\
& && && \multicolumn{9}{c}{$\sigma_{\varepsilon}^2 = 1$, \ \ $\sigma_{b_0}^2 = \sigma_{b_1}^2 = 1$} \\ \cline{6-16}

\rowcolor{gray!20} Normal       & Normal       && 0.05 &&  0.05 & 0.05 & 0.04 && 0.05 & 0.05 & 0.04 && 0.04 & 0.04 & 0.04 \\ 
\rowcolor{gray!20}              &              && 0.10 &&  0.11 & 0.10 & 0.09 && 0.11 & 0.10 & 0.09 && 0.09 & 0.09 & 0.08 \\ 
\rowcolor{gray!20}              & Heavy tailed && 0.05 &&  0.07 & 0.06 & 0.06 && 0.07 & 0.06 & 0.06 && 0.05 & 0.06 & 0.05 \\ 
\rowcolor{gray!20}              &              && 0.10 &&  0.13 & 0.12 & 0.11 && 0.13 & 0.12 & 0.11 && 0.11 & 0.11 & 0.10 \\ 
\rowcolor{gray!20}              & Skewed       && 0.05 &&  0.05 & 0.05 & 0.05 && 0.05 & 0.05 & 0.05 && 0.04 & 0.04 & 0.04 \\ 
\rowcolor{gray!20}              &              && 0.10 &&  0.10 & 0.09 & 0.11 && 0.10 & 0.09 & 0.11 && 0.08 & 0.09 & 0.10 \\ 
              &&&&&&&&&&&&&&&\\
 Heavy tailed & Normal       && 0.05 &&  1.00 & 1.00 & 1.00 && 1.00 & 1.00 & 1.00 && 1.00 & 1.00 & 1.00 \\ 
              &              && 0.10 &&  1.00 & 1.00 & 1.00 && 1.00 & 1.00 & 1.00 && 1.00 & 1.00 & 1.00 \\ 
              & Heavy tailed && 0.05 &&  1.00 & 1.00 & 1.00 && 1.00 & 1.00 & 1.00 && 1.00 & 1.00 & 1.00 \\ 
              &              && 0.10 &&  1.00 & 1.00 & 1.00 && 1.00 & 1.00 & 1.00 && 1.00 & 1.00 & 1.00 \\ 
              & Skewed       && 0.05 &&  1.00 & 1.00 & 1.00 && 1.00 & 1.00 & 1.00 && 1.00 & 1.00 & 1.00 \\ 
              &              && 0.10 &&  1.00 & 1.00 & 1.00 && 1.00 & 1.00 & 1.00 && 1.00 & 1.00 & 1.00 \\ 
              &&&&&&&&&&&&&&&\\
 Skewed       & Normal       && 0.05 &&  1.00 & 1.00 & 1.00 && 1.00 & 1.00 & 1.00 && 1.00 & 1.00 & 1.00 \\ 
              &              && 0.10 &&  1.00 & 1.00 & 1.00 && 1.00 & 1.00 & 1.00 && 1.00 & 1.00 & 1.00 \\ 
              & Heavy tailed && 0.05 &&  1.00 & 1.00 & 1.00 && 1.00 & 1.00 & 1.00 && 1.00 & 1.00 & 1.00 \\ 
              &              && 0.10 &&  1.00 & 1.00 & 1.00 && 1.00 & 1.00 & 1.00 && 1.00 & 1.00 & 1.00 \\ 
              & Skewed       && 0.05 &&  1.00 & 1.00 & 1.00 && 1.00 & 1.00 & 1.00 && 1.00 & 1.00 & 1.00 \\ 
              &              && 0.10 &&  1.00 & 1.00 & 1.00 && 1.00 & 1.00 & 1.00 && 1.00 & 1.00 & 1.00 \\ 

&&&&&&&&&&&&&&&\\
& && && \multicolumn{9}{c}{$\sigma_{\varepsilon}^2 = 1$, \ \ $\sigma_{b_0}^2 = \sigma_{b_1}^2 = 4$} \\ \cline{6-16}

\rowcolor{gray!20} Normal       & Normal       && 0.05 &&  0.10 & 0.09 & 0.07 && 0.10 & 0.09 & 0.07 && 0.05 & 0.04 & 0.04 \\ 
\rowcolor{gray!20}             &              && 0.10 &&  0.17 & 0.17 & 0.15 && 0.17 & 0.17 & 0.15 && 0.09 & 0.09 & 0.09 \\ 
\rowcolor{gray!20}             & Heavy tailed && 0.05 &&  0.11 & 0.11 & 0.10 && 0.11 & 0.11 & 0.10 && 0.06 & 0.06 & 0.05 \\ 
\rowcolor{gray!20}             &              && 0.10 &&  0.19 & 0.19 & 0.19 && 0.19 & 0.19 & 0.19 && 0.12 & 0.11 & 0.12 \\ 
\rowcolor{gray!20}             & Skewed       && 0.05 &&  0.10 & 0.10 & 0.09 && 0.10 & 0.10 & 0.09 && 0.05 & 0.05 & 0.06 \\ 
\rowcolor{gray!20}             &              && 0.10 &&  0.18 & 0.18 & 0.17 && 0.18 & 0.18 & 0.17 && 0.11 & 0.11 & 0.11 \\ 
             &&&&&&&&&&&&&&&\\
Heavy tailed & Normal       && 0.05 &&  1.00 & 1.00 & 1.00 && 1.00 & 1.00 & 1.00 && 1.00 & 1.00 & 1.00 \\ 
             &              && 0.10 &&  1.00 & 1.00 & 1.00 && 1.00 & 1.00 & 1.00 && 1.00 & 1.00 & 1.00 \\ 
             & Heavy tailed && 0.05 &&  1.00 & 1.00 & 1.00 && 1.00 & 1.00 & 1.00 && 1.00 & 1.00 & 1.00 \\ 
             &              && 0.10 &&  1.00 & 1.00 & 1.00 && 1.00 & 1.00 & 1.00 && 1.00 & 1.00 & 1.00 \\ 
             & Skewed       && 0.05 &&  1.00 & 1.00 & 1.00 && 1.00 & 1.00 & 1.00 && 1.00 & 1.00 & 1.00 \\ 
             &              && 0.10 &&  1.00 & 1.00 & 1.00 && 1.00 & 1.00 & 1.00 && 1.00 & 1.00 & 1.00 \\ 
             &&&&&&&&&&&&&&&\\
Skewed       & Normal       && 0.05 &&  1.00 & 1.00 & 1.00 && 1.00 & 1.00 & 1.00 && 1.00 & 1.00 & 1.00 \\ 
             &              && 0.10 &&  1.00 & 1.00 & 1.00 && 1.00 & 1.00 & 1.00 && 1.00 & 1.00 & 1.00 \\ 
             & Heavy tailed && 0.05 &&  1.00 & 1.00 & 1.00 && 1.00 & 1.00 & 1.00 && 1.00 & 1.00 & 1.00 \\ 
             &              && 0.10 &&  1.00 & 1.00 & 1.00 && 1.00 & 1.00 & 1.00 && 1.00 & 1.00 & 1.00 \\ 
             & Skewed       && 0.05 &&  1.00 & 1.00 & 1.00 && 1.00 & 1.00 & 1.00 && 1.00 & 1.00 & 1.00 \\ 
             &              && 0.10 &&  1.00 & 1.00 & 1.00 && 1.00 & 1.00 & 1.00 && 1.00 & 1.00 & 1.00 \\ 


   \hline
\end{tabular}
\end{scriptsize}
\end{table}

%%% Level-2 residuals
\begin{table}[ht]
\centering
\caption{\label{tab:evalb0} Proportion of tests rejecting the null hypothesis of normality of the random intercept.}
\begin{scriptsize}
\begin{tabular}{ll p{.1cm} c p{.1cm} rrr p{.1cm} rrr p{.1cm} rrr}
  \hline
  \multicolumn{2}{c}{Distributions}& & Nominal & &  \multicolumn{3}{c}{Raw residuals} & & \multicolumn{3}{c}{Pearson residuals} & & \multicolumn{3}{c}{Studentized residuals}\\ \cline{1-2} \cline{6-8} \cline{10-12} \cline{14-16}
  Random effects & Errors & & $\alpha$ & & AD & CVM & KS & & AD & CVM & KS & & AD & CVM & KS \\ 
   \hline
& && && \multicolumn{9}{c}{$\sigma_{\varepsilon}^2 = 4$, \ \ $\sigma_{b_0}^2 = \sigma_{b_1}^2 = 1$} \\ \cline{6-16}

\rowcolor{gray!20} Normal       & Normal       && 0.05 &&   0.05 & 0.05 & 0.05 && 0.05 & 0.05 & 0.06 && 0.05 & 0.05 & 0.06 \\ 
\rowcolor{gray!20}             &              && 0.10 &&   0.10 & 0.10 & 0.10 && 0.11 & 0.10 & 0.12 && 0.11 & 0.10 & 0.12 \\ 
\rowcolor{gray!20}             & Heavy tailed && 0.05 &&   0.15 & 0.13 & 0.12 && 0.17 & 0.15 & 0.13 && 0.17 & 0.15 & 0.13 \\ 
\rowcolor{gray!20}             &              && 0.10 &&   0.22 & 0.20 & 0.20 && 0.26 & 0.23 & 0.21 && 0.26 & 0.23 & 0.20 \\ 
\rowcolor{gray!20}             & Skewed       && 0.05 &&   0.16 & 0.15 & 0.13 && 0.18 & 0.17 & 0.14 && 0.18 & 0.17 & 0.14 \\ 
\rowcolor{gray!20}             &              && 0.10 &&   0.25 & 0.23 & 0.21 && 0.28 & 0.25 & 0.22 && 0.28 & 0.25 & 0.21 \\ 
             &&&&&&&&&&&&&&&\\
Heavy tailed & Normal       && 0.05 &&   0.27 & 0.24 & 0.19 && 0.28 & 0.26 & 0.21 && 0.28 & 0.26 & 0.21 \\ 
             &              && 0.10 &&   0.35 & 0.31 & 0.28 && 0.36 & 0.34 & 0.30 && 0.36 & 0.33 & 0.30 \\ 
             & Heavy tailed && 0.05 &&   0.49 & 0.45 & 0.35 && 0.51 & 0.46 & 0.36 && 0.50 & 0.46 & 0.36 \\ 
             &              && 0.10 &&   0.58 & 0.54 & 0.46 && 0.60 & 0.55 & 0.47 && 0.60 & 0.55 & 0.47 \\ 
             & Skewed       && 0.05 &&   0.52 & 0.48 & 0.36 && 0.55 & 0.50 & 0.40 && 0.55 & 0.50 & 0.40 \\ 
             &              && 0.10 &&   0.62 & 0.59 & 0.51 && 0.65 & 0.60 & 0.53 && 0.65 & 0.60 & 0.52 \\
             &&&&&&&&&&&&&&&\\ 
Skewed       & Normal       && 0.05 &&   0.51 & 0.48 & 0.39 && 0.51 & 0.49 & 0.38 && 0.51 & 0.49 & 0.39 \\ 
             &              && 0.10 &&   0.61 & 0.58 & 0.51 && 0.61 & 0.59 & 0.52 && 0.60 & 0.58 & 0.52 \\ 
             & Heavy tailed && 0.05 &&   0.73 & 0.69 & 0.58 && 0.73 & 0.70 & 0.59 && 0.73 & 0.70 & 0.59 \\ 
             &              && 0.10 &&   0.80 & 0.77 & 0.69 && 0.81 & 0.78 & 0.70 && 0.80 & 0.78 & 0.70 \\ 
             & Skewed       && 0.05 &&   0.87 & 0.83 & 0.70 && 0.87 & 0.82 & 0.69 && 0.86 & 0.83 & 0.69 \\ 
             &              && 0.10 &&   0.92 & 0.89 & 0.80 && 0.91 & 0.88 & 0.80 && 0.91 & 0.88 & 0.80 \\ 

&&&&&&&&&&&&&&&\\
& && && \multicolumn{9}{c}{$\sigma_{\varepsilon}^2 = 1$, \ \ $\sigma_{b_0}^2 = \sigma_{b_1}^2 = 1$} \\ \cline{6-16}
\rowcolor{gray!20} Normal       & Normal       && 0.05 &&   0.05 & 0.05 & 0.04 && 0.05 & 0.05 & 0.04 && 0.05 & 0.05 & 0.04 \\ 
\rowcolor{gray!20}             &              && 0.10 &&   0.09 & 0.09 & 0.08 && 0.09 & 0.08 & 0.08 && 0.08 & 0.08 & 0.08 \\ 
\rowcolor{gray!20}             & Heavy tailed && 0.05 &&   0.07 & 0.07 & 0.06 && 0.07 & 0.07 & 0.06 && 0.07 & 0.07 & 0.06 \\ 
\rowcolor{gray!20}             &              && 0.10 &&   0.12 & 0.12 & 0.11 && 0.12 & 0.11 & 0.11 && 0.12 & 0.11 & 0.11 \\ 
\rowcolor{gray!20}             & Skewed       && 0.05 &&   0.06 & 0.06 & 0.07 && 0.06 & 0.06 & 0.06 && 0.06 & 0.06 & 0.06 \\ 
\rowcolor{gray!20}             &              && 0.10 &&   0.11 & 0.11 & 0.12 && 0.12 & 0.12 & 0.12 && 0.12 & 0.11 & 0.12 \\
             &&&&&&&&&&&&&&&\\  
Heavy tailed & Normal       && 0.05 &&   0.55 & 0.50 & 0.42 && 0.52 & 0.48 & 0.40 && 0.52 & 0.48 & 0.40 \\ 
             &              && 0.10 &&   0.63 & 0.60 & 0.51 && 0.60 & 0.56 & 0.51 && 0.60 & 0.56 & 0.52 \\ 
             & Heavy tailed && 0.05 &&   0.63 & 0.58 & 0.49 && 0.60 & 0.56 & 0.47 && 0.60 & 0.56 & 0.47 \\ 
             &              && 0.10 &&   0.71 & 0.68 & 0.60 && 0.68 & 0.63 & 0.56 && 0.68 & 0.63 & 0.56 \\ 
             & Skewed       && 0.05 &&   0.62 & 0.57 & 0.47 && 0.61 & 0.55 & 0.46 && 0.61 & 0.55 & 0.46 \\ 
             &              && 0.10 &&   0.71 & 0.66 & 0.58 && 0.69 & 0.65 & 0.57 && 0.69 & 0.64 & 0.57 \\ 
             &&&&&&&&&&&&&&&\\ 
Skewed       & Normal       && 0.05 &&   0.93 & 0.92 & 0.86 && 0.93 & 0.91 & 0.86 && 0.93 & 0.91 & 0.85 \\ 
             &              && 0.10 &&   0.96 & 0.94 & 0.91 && 0.96 & 0.94 & 0.90 && 0.95 & 0.94 & 0.90 \\ 
             & Heavy tailed && 0.05 &&   0.97 & 0.96 & 0.89 && 0.97 & 0.96 & 0.88 && 0.97 & 0.96 & 0.88 \\ 
             &              && 0.10 &&   0.99 & 0.98 & 0.94 && 0.99 & 0.98 & 0.94 && 0.99 & 0.98 & 0.94 \\ 
             & Skewed       && 0.05 &&   0.98 & 0.96 & 0.90 && 0.98 & 0.97 & 0.90 && 0.98 & 0.97 & 0.91 \\ 
             &              && 0.10 &&   0.99 & 0.97 & 0.95 && 0.99 & 0.98 & 0.94 && 0.99 & 0.98 & 0.95 \\ 


&&&&&&&&&&&&&&&\\
& && && \multicolumn{9}{c}{$\sigma_{\varepsilon}^2 = 1$, \ \ $\sigma_{b_0}^2 = \sigma_{b_1}^2 = 4$} \\ \cline{6-16}

\rowcolor{gray!20} Normal       & Normal       && 0.05 &&  0.05 & 0.05 & 0.05 && 0.05 & 0.05 & 0.04 && 0.05 & 0.05 & 0.04 \\ 
\rowcolor{gray!20}             &              && 0.10 &&  0.10 & 0.09 & 0.09 && 0.10 & 0.09 & 0.09 && 0.10 & 0.09 & 0.09 \\ 
\rowcolor{gray!20}             & Heavy tailed && 0.05 &&  0.04 & 0.04 & 0.05 && 0.05 & 0.05 & 0.05 && 0.05 & 0.05 & 0.05 \\ 
\rowcolor{gray!20}             &              && 0.10 &&  0.09 & 0.09 & 0.09 && 0.09 & 0.09 & 0.09 && 0.09 & 0.09 & 0.09 \\ 
\rowcolor{gray!20}             & Skewed       && 0.05 &&  0.04 & 0.04 & 0.03 && 0.04 & 0.04 & 0.03 && 0.04 & 0.04 & 0.03 \\ 
\rowcolor{gray!20}             &              && 0.10 &&  0.09 & 0.09 & 0.08 && 0.10 & 0.09 & 0.08 && 0.09 & 0.09 & 0.08 \\ 
             &&&&&&&&&&&&&&&\\
Heavy tailed & Normal       && 0.05 &&  0.68 & 0.63 & 0.54 && 0.68 & 0.63 & 0.54 && 0.68 & 0.63 & 0.54 \\ 
             &              && 0.10 &&  0.75 & 0.71 & 0.63 && 0.75 & 0.70 & 0.64 && 0.75 & 0.71 & 0.64 \\ 
             & Heavy tailed && 0.05 &&  0.71 & 0.67 & 0.57 && 0.71 & 0.67 & 0.58 && 0.71 & 0.67 & 0.57 \\ 
             &              && 0.10 &&  0.78 & 0.76 & 0.68 && 0.79 & 0.76 & 0.67 && 0.79 & 0.75 & 0.67 \\ 
             & Skewed       && 0.05 &&  0.70 & 0.68 & 0.57 && 0.70 & 0.67 & 0.56 && 0.70 & 0.67 & 0.56 \\ 
             &              && 0.10 &&  0.78 & 0.74 & 0.68 && 0.78 & 0.74 & 0.68 && 0.78 & 0.74 & 0.67 \\ 
             &&&&&&&&&&&&&&&\\
Skewed       & Normal       && 0.05 &&  1.00 & 0.99 & 0.97 && 1.00 & 0.99 & 0.97 && 1.00 & 0.99 & 0.97 \\ 
             &              && 0.10 &&  1.00 & 1.00 & 0.99 && 1.00 & 1.00 & 0.99 && 1.00 & 1.00 & 0.99 \\ 
             & Heavy tailed && 0.05 &&  1.00 & 1.00 & 0.98 && 1.00 & 1.00 & 0.98 && 1.00 & 1.00 & 0.98 \\ 
             &              && 0.10 &&  1.00 & 1.00 & 0.99 && 1.00 & 1.00 & 0.99 && 1.00 & 1.00 & 0.99 \\ 
             & Skewed       && 0.05 &&  1.00 & 1.00 & 0.98 && 1.00 & 1.00 & 0.98 && 1.00 & 1.00 & 0.98 \\ 
             &              && 0.10 &&  1.00 & 1.00 & 0.99 && 1.00 & 1.00 & 0.99 && 1.00 & 1.00 & 0.99 \\ 


   \hline
\end{tabular}
\end{scriptsize}
\end{table}


\begin{table}[ht]
\centering
\caption{\label{tab:evalb1} Proportion of tests rejecting the null hypothesis of normality of the random slope.}
\begin{scriptsize}
\begin{tabular}{ll p{.1cm} c p{.1cm} rrr p{.1cm} rrr p{.1cm} rrr}
  \hline
  \multicolumn{2}{c}{Distributions}& & Nominal & &  \multicolumn{3}{c}{Raw residuals} & & \multicolumn{3}{c}{Pearson residuals} & & \multicolumn{3}{c}{Studentized residuals}\\ \cline{1-2} \cline{6-8} \cline{10-12} \cline{14-16}
  Random effects & Errors & & $\alpha$ & & AD & CVM & KS & & AD & CVM & KS & & AD & CVM & KS \\ 
   \hline
& && && \multicolumn{9}{c}{$\sigma_{\varepsilon}^2 = 4$, \ \ $\sigma_{b_0}^2 = \sigma_{b_1}^2 = 1$} \\ \cline{6-16}

\rowcolor{gray!20} Normal       & Normal       && 0.05 &&   1.00 & 1.00 & 1.00 && 0.05 & 0.05 & 0.06 && 0.05 & 0.05 & 0.06 \\ 
\rowcolor{gray!20}             &              && 0.10 &&   1.00 & 1.00 & 1.00 && 0.10 & 0.10 & 0.10 && 0.10 & 0.10 & 0.10 \\ 
\rowcolor{gray!20}             & Heavy tailed && 0.05 &&   1.00 & 1.00 & 1.00 && 0.26 & 0.24 & 0.19 && 0.26 & 0.24 & 0.19 \\ 
\rowcolor{gray!20}             &              && 0.10 &&   1.00 & 1.00 & 1.00 && 0.35 & 0.32 & 0.27 && 0.35 & 0.32 & 0.27 \\ 
\rowcolor{gray!20}             & Skewed       && 0.05 &&   1.00 & 1.00 & 1.00 && 0.33 & 0.31 & 0.24 && 0.33 & 0.31 & 0.24 \\ 
\rowcolor{gray!20}             &              && 0.10 &&   1.00 & 1.00 & 1.00 && 0.41 & 0.38 & 0.34 && 0.41 & 0.38 & 0.34 \\ 
             &&&&&&&&&&&&&&&\\
Heavy tailed & Normal       && 0.05 &&   1.00 & 1.00 & 1.00 && 0.13 & 0.12 & 0.09 && 0.13 & 0.12 & 0.09 \\ 
             &              && 0.10 &&   1.00 & 1.00 & 1.00 && 0.18 & 0.19 & 0.17 && 0.18 & 0.19 & 0.17 \\ 
             & Heavy tailed && 0.05 &&   1.00 & 1.00 & 1.00 && 0.40 & 0.36 & 0.29 && 0.40 & 0.36 & 0.29 \\ 
             &              && 0.10 &&   1.00 & 1.00 & 1.00 && 0.49 & 0.44 & 0.37 && 0.49 & 0.45 & 0.37 \\ 
             & Skewed       && 0.05 &&   1.00 & 1.00 & 1.00 && 0.49 & 0.46 & 0.37 && 0.49 & 0.46 & 0.37 \\ 
             &              && 0.10 &&   1.00 & 1.00 & 1.00 && 0.59 & 0.56 & 0.50 && 0.59 & 0.56 & 0.49 \\
             &&&&&&&&&&&&&&&\\ 
Skewed       & Normal       && 0.05 &&   1.00 & 1.00 & 1.00 && 0.12 & 0.11 & 0.10 && 0.12 & 0.11 & 0.10 \\ 
             &              && 0.10 &&   1.00 & 1.00 & 1.00 && 0.17 & 0.16 & 0.16 && 0.18 & 0.16 & 0.16 \\ 
             & Heavy tailed && 0.05 &&   1.00 & 1.00 & 1.00 && 0.41 & 0.37 & 0.30 && 0.40 & 0.37 & 0.30 \\ 
             &              && 0.10 &&   1.00 & 1.00 & 1.00 && 0.51 & 0.47 & 0.39 && 0.51 & 0.47 & 0.39 \\ 
             & Skewed       && 0.05 &&   1.00 & 1.00 & 1.00 && 0.59 & 0.56 & 0.46 && 0.59 & 0.56 & 0.46 \\ 
             &              && 0.10 &&   1.00 & 1.00 & 1.00 && 0.70 & 0.66 & 0.58 && 0.70 & 0.66 & 0.58 \\ 

&&&&&&&&&&&&&&&\\
& && && \multicolumn{9}{c}{$\sigma_{\varepsilon}^2 = 1$, \ \ $\sigma_{b_0}^2 = \sigma_{b_1}^2 = 1$} \\ \cline{6-16}

\rowcolor{gray!20} Normal       & Normal       && 0.05 &&  1.00 & 1.00 & 1.00 && 0.05 & 0.05 & 0.06 && 0.05 & 0.05 & 0.06 \\ 
\rowcolor{gray!20}             &              && 0.10 &&  1.00 & 1.00 & 1.00 && 0.12 & 0.11 & 0.11 && 0.12 & 0.11 & 0.11 \\ 
\rowcolor{gray!20}             & Heavy tailed && 0.05 &&  1.00 & 1.00 & 1.00 && 0.13 & 0.12 & 0.11 && 0.13 & 0.12 & 0.11 \\ 
\rowcolor{gray!20}             &              && 0.10 &&  1.00 & 1.00 & 1.00 && 0.22 & 0.20 & 0.17 && 0.22 & 0.20 & 0.17 \\ 
\rowcolor{gray!20}             & Skewed       && 0.05 &&  1.00 & 1.00 & 1.00 && 0.14 & 0.12 & 0.10 && 0.14 & 0.12 & 0.11 \\ 
\rowcolor{gray!20}             &              && 0.10 &&  1.00 & 1.00 & 1.00 && 0.22 & 0.20 & 0.17 && 0.22 & 0.20 & 0.17 \\
             &&&&&&&&&&&&&&&\\ 
Heavy tailed & Normal       && 0.05 &&  1.00 & 1.00 & 1.00 && 0.27 & 0.24 & 0.19 && 0.27 & 0.24 & 0.19 \\ 
             &              && 0.10 &&  1.00 & 1.00 & 1.00 && 0.35 & 0.32 & 0.27 && 0.35 & 0.31 & 0.27 \\ 
             & Heavy tailed && 0.05 &&  1.00 & 1.00 & 1.00 && 0.44 & 0.40 & 0.33 && 0.44 & 0.40 & 0.33 \\ 
             &              && 0.10 &&  1.00 & 1.00 & 1.00 && 0.50 & 0.48 & 0.42 && 0.50 & 0.48 & 0.42 \\ 
             & Skewed       && 0.05 &&  1.00 & 1.00 & 1.00 && 0.41 & 0.38 & 0.31 && 0.41 & 0.38 & 0.31 \\ 
             &              && 0.10 &&  1.00 & 1.00 & 1.00 && 0.51 & 0.48 & 0.42 && 0.51 & 0.48 & 0.42 \\ 
             &&&&&&&&&&&&&&&\\
Skewed       & Normal       && 0.05 &&  1.00 & 1.00 & 1.00 && 0.46 & 0.42 & 0.34 && 0.46 & 0.42 & 0.34 \\ 
             &              && 0.10 &&  1.00 & 1.00 & 1.00 && 0.57 & 0.52 & 0.46 && 0.57 & 0.52 & 0.46 \\ 
             & Heavy tailed && 0.05 &&  1.00 & 1.00 & 1.00 && 0.65 & 0.60 & 0.51 && 0.65 & 0.60 & 0.51 \\ 
             &              && 0.10 &&  1.00 & 1.00 & 1.00 && 0.73 & 0.69 & 0.62 && 0.73 & 0.69 & 0.62 \\ 
             & Skewed       && 0.05 &&  1.00 & 1.00 & 1.00 && 0.75 & 0.70 & 0.57 && 0.75 & 0.70 & 0.57 \\ 
             &              && 0.10 &&  1.00 & 1.00 & 1.00 && 0.83 & 0.78 & 0.70 && 0.83 & 0.78 & 0.70 \\ 


&&&&&&&&&&&&&&&\\
& && && \multicolumn{9}{c}{$\sigma_{\varepsilon}^2 = 1$, \ \ $\sigma_{b_0}^2 = \sigma_{b_1}^2 = 4$} \\ \cline{6-16}

\rowcolor{gray!20} Normal       & Normal       && 0.05 &&  1.00 & 1.00 & 1.00 && 0.04 & 0.05 & 0.05 && 0.04 & 0.05 & 0.05 \\ 
\rowcolor{gray!20}             &              && 0.10 &&  1.00 & 1.00 & 1.00 && 0.11 & 0.10 & 0.10 && 0.11 & 0.10 & 0.10 \\ 
\rowcolor{gray!20}             & Heavy tailed && 0.05 &&  1.00 & 1.00 & 1.00 && 0.07 & 0.07 & 0.07 && 0.07 & 0.07 & 0.07 \\ 
\rowcolor{gray!20}             &              && 0.10 &&  1.00 & 1.00 & 1.00 && 0.12 & 0.12 & 0.13 && 0.12 & 0.12 & 0.13 \\ 
\rowcolor{gray!20}             & Skewed       && 0.05 &&  1.00 & 1.00 & 1.00 && 0.06 & 0.06 & 0.05 && 0.06 & 0.06 & 0.05 \\ 
\rowcolor{gray!20}             &              && 0.10 &&  1.00 & 1.00 & 1.00 && 0.11 & 0.11 & 0.11 && 0.11 & 0.11 & 0.11 \\ 
             &&&&&&&&&&&&&&&\\
Heavy tailed & Normal       && 0.05 &&  1.00 & 1.00 & 1.00 && 0.46 & 0.41 & 0.34 && 0.46 & 0.41 & 0.34 \\ 
             &              && 0.10 &&  1.00 & 1.00 & 1.00 && 0.56 & 0.52 & 0.44 && 0.56 & 0.52 & 0.44 \\ 
             & Heavy tailed && 0.05 &&  1.00 & 1.00 & 1.00 && 0.55 & 0.50 & 0.43 && 0.55 & 0.50 & 0.42 \\ 
             &              && 0.10 &&  1.00 & 1.00 & 1.00 && 0.63 & 0.60 & 0.53 && 0.63 & 0.60 & 0.53 \\ 
             & Skewed       && 0.05 &&  1.00 & 1.00 & 1.00 && 0.48 & 0.46 & 0.37 && 0.48 & 0.46 & 0.37 \\ 
             &              && 0.10 &&  1.00 & 1.00 & 1.00 && 0.57 & 0.53 & 0.48 && 0.57 & 0.53 & 0.48 \\ 
             &&&&&&&&&&&&&&&\\
Skewed       & Normal       && 0.05 &&  1.00 & 1.00 & 1.00 && 0.90 & 0.87 & 0.74 && 0.90 & 0.87 & 0.74 \\ 
             &              && 0.10 &&  1.00 & 1.00 & 1.00 && 0.94 & 0.93 & 0.83 && 0.94 & 0.93 & 0.84 \\ 
             & Heavy tailed && 0.05 &&  1.00 & 1.00 & 1.00 && 0.92 & 0.91 & 0.80 && 0.93 & 0.91 & 0.80 \\ 
             &              && 0.10 &&  1.00 & 1.00 & 1.00 && 0.96 & 0.95 & 0.90 && 0.96 & 0.95 & 0.90 \\ 
             & Skewed       && 0.05 &&  1.00 & 1.00 & 1.00 && 0.92 & 0.90 & 0.79 && 0.92 & 0.90 & 0.79 \\ 
             &              && 0.10 &&  1.00 & 1.00 & 1.00 && 0.95 & 0.93 & 0.89 && 0.95 & 0.93 & 0.89 \\ 

   \hline
\end{tabular}
\end{scriptsize}
\end{table}

%%% Marginal residuals
\begin{table}[ht]
\centering
\caption{\label{tab:evalmarginal} Proportion of tests rejecting the null hypothesis of normality of the marginal residuals.}
\begin{scriptsize}
\begin{tabular}{ll p{.1cm} c p{.1cm} rrr p{.1cm} rrr}
  \hline
  \multicolumn{2}{c}{Distributions}& & Nominal & &  \multicolumn{3}{c}{Raw residuals} & & \multicolumn{3}{c}{Cholesky residuals} \\ \cline{1-2} \cline{6-8} \cline{10-12}
  Errors & Random effects & & $\alpha$ & & AD & CVM & KS & & AD & CVM & KS \\ 
   \hline
& && && \multicolumn{6}{c}{$\sigma_{\varepsilon}^2 = 4$, \ \ $\sigma_{b_0}^2 = \sigma_{b_1}^2 = 1$} \\ \cline{6-12}
\rowcolor{gray!20} Normal       & Normal       && 0.05 &&  0.05 & 0.05 & 0.06 && 0.06 & 0.06 & 0.06 \\ 
\rowcolor{gray!20}             &              && 0.10 &&  0.12 & 0.12 & 0.13 && 0.11 & 0.11 & 0.12 \\ 
\rowcolor{gray!20}             & Heavy tailed && 0.05 &&  0.20 & 0.19 & 0.14 && 0.09 & 0.08 & 0.05 \\ 
\rowcolor{gray!20}             &              && 0.10 &&  0.28 & 0.24 & 0.22 && 0.14 & 0.13 & 0.12 \\ 
\rowcolor{gray!20}             & Skewed       && 0.05 &&  0.30 & 0.27 & 0.22 && 0.05 & 0.04 & 0.05 \\ 
\rowcolor{gray!20}             &              && 0.10 &&  0.39 & 0.36 & 0.32 && 0.10 & 0.10 & 0.10 \\ 
             &&&&&&&&&&&\\
Heavy tailed & Normal       && 0.05 &&  1.00 & 1.00 & 0.99 && 1.00 & 1.00 & 1.00 \\ 
             &              && 0.10 &&  1.00 & 1.00 & 0.99 && 1.00 & 1.00 & 1.00 \\ 
             & Heavy tailed && 0.05 &&  1.00 & 1.00 & 1.00 && 1.00 & 1.00 & 1.00 \\ 
             &              && 0.10 &&  1.00 & 1.00 & 1.00 && 1.00 & 1.00 & 1.00 \\ 
             & Skewed       && 0.05 &&  1.00 & 1.00 & 1.00 && 1.00 & 1.00 & 1.00 \\ 
             &              && 0.10 &&  1.00 & 1.00 & 1.00 && 1.00 & 1.00 & 1.00 \\ 
             &&&&&&&&&&&\\
Skewed       & Normal       && 0.05 &&  1.00 & 1.00 & 1.00 && 1.00 & 1.00 & 1.00 \\ 
             &              && 0.10 &&  1.00 & 1.00 & 1.00 && 1.00 & 1.00 & 1.00 \\ 
             & Heavy tailed && 0.05 &&  1.00 & 1.00 & 1.00 && 1.00 & 1.00 & 1.00 \\ 
             &              && 0.10 &&  1.00 & 1.00 & 1.00 && 1.00 & 1.00 & 1.00 \\ 
             & Skewed       && 0.05 &&  1.00 & 1.00 & 1.00 && 1.00 & 1.00 & 1.00 \\ 
             &              && 0.10 &&  1.00 & 1.00 & 1.00 && 1.00 & 1.00 & 1.00 \\ 

&&&&&&&&&&&\\
& && && \multicolumn{6}{c}{$\sigma_{\varepsilon}^2 = 1$, \ \ $\sigma_{b_0}^2 = \sigma_{b_1}^2 = 1$} \\ \cline{6-12}

\rowcolor{gray!20} Normal       & Normal       && 0.05 &&   0.32 & 0.30 & 0.25 && 0.05 & 0.05 & 0.04 \\ 
\rowcolor{gray!20}             &              && 0.10 &&   0.41 & 0.38 & 0.34 && 0.09 & 0.10 & 0.10 \\ 
\rowcolor{gray!20}             & Heavy tailed && 0.05 &&   0.65 & 0.61 & 0.52 && 0.10 & 0.09 & 0.06 \\ 
\rowcolor{gray!20}             &              && 0.10 &&   0.72 & 0.68 & 0.61 && 0.16 & 0.14 & 0.13 \\ 
\rowcolor{gray!20}             & Skewed       && 0.05 &&   0.93 & 0.90 & 0.87 && 0.11 & 0.10 & 0.09 \\ 
\rowcolor{gray!20}             &              && 0.10 &&   0.94 & 0.93 & 0.91 && 0.18 & 0.17 & 0.16 \\ 
             &&&&&&&&&&&\\
Heavy tailed & Normal       && 0.05 &&   0.95 & 0.91 & 0.85 && 1.00 & 1.00 & 1.00 \\ 
             &              && 0.10 &&   0.97 & 0.94 & 0.90 && 1.00 & 1.00 & 1.00 \\ 
             & Heavy tailed && 0.05 &&   1.00 & 1.00 & 0.99 && 1.00 & 1.00 & 1.00 \\ 
             &              && 0.10 &&   1.00 & 1.00 & 1.00 && 1.00 & 1.00 & 1.00 \\ 
             & Skewed       && 0.05 &&   1.00 & 1.00 & 1.00 && 1.00 & 1.00 & 1.00 \\ 
             &              && 0.10 &&   1.00 & 1.00 & 1.00 && 1.00 & 1.00 & 1.00 \\
             &&&&&&&&&&&\\ 
Skewed       & Normal       && 0.05 &&   1.00 & 0.99 & 0.98 && 1.00 & 1.00 & 1.00 \\ 
             &              && 0.10 &&   1.00 & 1.00 & 0.99 && 1.00 & 1.00 & 1.00 \\ 
             & Heavy tailed && 0.05 &&   1.00 & 1.00 & 1.00 && 1.00 & 1.00 & 1.00 \\ 
             &              && 0.10 &&   1.00 & 1.00 & 1.00 && 1.00 & 1.00 & 1.00 \\ 
             & Skewed       && 0.05 &&   1.00 & 1.00 & 1.00 && 1.00 & 1.00 & 1.00 \\ 
             &              && 0.10 &&   1.00 & 1.00 & 1.00 && 1.00 & 1.00 & 1.00 \\ 


&&&&&&&&&&&\\
& && && \multicolumn{6}{c}{$\sigma_{\varepsilon}^2 = 1$, \ \ $\sigma_{b_0}^2 = \sigma_{b_1}^2 = 4$} \\ \cline{6-12}

\rowcolor{gray!20} Normal       & Normal       && 0.05 &&   0.82 & 0.80 & 0.71 && 0.05 & 0.05 & 0.04 \\ 
\rowcolor{gray!20}             &              && 0.10 &&   0.87 & 0.85 & 0.80 && 0.10 & 0.09 & 0.08 \\ 
\rowcolor{gray!20}             & Heavy tailed && 0.05 &&   0.96 & 0.94 & 0.91 && 0.16 & 0.14 & 0.10 \\ 
\rowcolor{gray!20}             &              && 0.10 &&   0.97 & 0.95 & 0.93 && 0.26 & 0.23 & 0.20 \\ 
\rowcolor{gray!20}             & Skewed       && 0.05 &&   1.00 & 1.00 & 0.99 && 0.31 & 0.29 & 0.23 \\ 
\rowcolor{gray!20}             &              && 0.10 &&   1.00 & 1.00 & 1.00 && 0.43 & 0.40 & 0.34 \\ 
             &&&&&&&&&&&\\
Heavy tailed & Normal       && 0.05 &&   0.98 & 0.96 & 0.92 && 1.00 & 1.00 & 1.00 \\ 
             &              && 0.10 &&   0.99 & 0.97 & 0.96 && 1.00 & 1.00 & 1.00 \\ 
             & Heavy tailed && 0.05 &&   1.00 & 0.99 & 0.98 && 1.00 & 1.00 & 1.00 \\ 
             &              && 0.10 &&   1.00 & 0.99 & 0.99 && 1.00 & 1.00 & 1.00 \\ 
             & Skewed       && 0.05 &&   1.00 & 1.00 & 1.00 && 1.00 & 1.00 & 1.00 \\ 
             &              && 0.10 &&   1.00 & 1.00 & 1.00 && 1.00 & 1.00 & 1.00 \\ 
             &&&&&&&&&&&\\
Skewed       & Normal       && 0.05 &&   0.98 & 0.97 & 0.95 && 1.00 & 1.00 & 1.00 \\ 
             &              && 0.10 &&   0.99 & 0.98 & 0.98 && 1.00 & 1.00 & 1.00 \\ 
             & Heavy tailed && 0.05 &&   1.00 & 1.00 & 1.00 && 1.00 & 1.00 & 1.00 \\ 
             &              && 0.10 &&   1.00 & 1.00 & 1.00 && 1.00 & 1.00 & 1.00 \\ 
             & Skewed       && 0.05 &&   1.00 & 1.00 & 1.00 && 1.00 & 1.00 & 1.00 \\ 
             &              && 0.10 &&   1.00 & 1.00 & 1.00 && 1.00 & 1.00 & 1.00 \\ 


   \hline
\end{tabular}
\end{scriptsize}
\end{table}


%------------------------------------------------
% Bootstrap tests
%------------------------------------------------

%%% Level-1
\begin{table}[ht]
\centering
\caption{\label{tab:boot1} Proportion of bootstrap tests rejecting the null hypothesis of normality of the error terms.}
\begin{scriptsize}
\begin{tabular}{ll p{.1cm} c p{.1cm} rrr p{.1cm} rrr p{.1cm} rrr}
  \hline
  \multicolumn{2}{c}{Distributions}& & Nominal & &  \multicolumn{3}{c}{Raw residuals} & & \multicolumn{3}{c}{Pearson residuals} & & \multicolumn{3}{c}{Studentized residuals}\\ \cline{1-2} \cline{6-8} \cline{10-12} \cline{14-16}
  Errors & Random effects & & $\alpha$ & & AD & CVM & KS & & AD & CVM & KS & & AD & CVM & KS \\ 
   \hline
& && && \multicolumn{9}{c}{$\sigma_{\varepsilon}^2 = 4$, \ \ $\sigma_{b_0}^2 = \sigma_{b_1}^2 = 1$} \\ \cline{6-16}

\rowcolor{gray!20} Normal       & Normal       && 0.05 &&  0.05 & 0.05 & 0.05 && 0.05 & 0.05 & 0.05 && 0.04 & 0.04 & 0.04 \\ 
\rowcolor{gray!20}              &              && 0.10 &&  0.10 & 0.10 & 0.09 && 0.10 & 0.10 & 0.09 && 0.09 & 0.09 & 0.09 \\ 
\rowcolor{gray!20}              & Heavy tailed && 0.05 &&  0.05 & 0.05 & 0.05 && 0.05 & 0.05 & 0.05 && 0.05 & 0.05 & 0.05 \\ 
\rowcolor{gray!20}              &              && 0.10 &&  0.11 & 0.10 & 0.11 && 0.11 & 0.10 & 0.11 && 0.11 & 0.11 & 0.09 \\ 
\rowcolor{gray!20}              & Skewed       && 0.05 &&  0.05 & 0.04 & 0.05 && 0.05 & 0.04 & 0.05 && 0.04 & 0.04 & 0.05 \\ 
\rowcolor{gray!20}              &              && 0.10 &&  0.10 & 0.09 & 0.09 && 0.10 & 0.09 & 0.09 && 0.09 & 0.09 & 0.10 \\
              &&&&&&&&&&&&&&&\\  
 Heavy tailed & Normal       && 0.05 &&  1.00 & 1.00 & 1.00 && 1.00 & 1.00 & 1.00 && 1.00 & 1.00 & 1.00 \\ 
              &              && 0.10 &&  1.00 & 1.00 & 1.00 && 1.00 & 1.00 & 1.00 && 1.00 & 1.00 & 1.00 \\ 
              & Heavy tailed && 0.05 &&  1.00 & 1.00 & 1.00 && 1.00 & 1.00 & 1.00 && 1.00 & 1.00 & 1.00 \\ 
              &              && 0.10 &&  1.00 & 1.00 & 1.00 && 1.00 & 1.00 & 1.00 && 1.00 & 1.00 & 1.00 \\ 
              & Skewed       && 0.05 &&  1.00 & 1.00 & 1.00 && 1.00 & 1.00 & 1.00 && 1.00 & 1.00 & 1.00 \\ 
              &              && 0.10 &&  1.00 & 1.00 & 1.00 && 1.00 & 1.00 & 1.00 && 1.00 & 1.00 & 1.00 \\ 
              &&&&&&&&&&&&&&&\\ 
 Skewed       & Normal       && 0.05 &&  1.00 & 1.00 & 1.00 && 1.00 & 1.00 & 1.00 && 1.00 & 1.00 & 1.00 \\ 
              &              && 0.10 &&  1.00 & 1.00 & 1.00 && 1.00 & 1.00 & 1.00 && 1.00 & 1.00 & 1.00 \\ 
              & Heavy tailed && 0.05 &&  1.00 & 1.00 & 1.00 && 1.00 & 1.00 & 1.00 && 1.00 & 1.00 & 1.00 \\ 
              &              && 0.10 &&  1.00 & 1.00 & 1.00 && 1.00 & 1.00 & 1.00 && 1.00 & 1.00 & 1.00 \\ 
              & Skewed       && 0.05 &&  1.00 & 1.00 & 1.00 && 1.00 & 1.00 & 1.00 && 1.00 & 1.00 & 1.00 \\ 
              &              && 0.10 &&  1.00 & 1.00 & 1.00 && 1.00 & 1.00 & 1.00 && 1.00 & 1.00 & 1.00 \\ 

&&&&&&&&&&&&&&&\\
& && && \multicolumn{9}{c}{$\sigma_{\varepsilon}^2 = 1$, \ \ $\sigma_{b_0}^2 = \sigma_{b_1}^2 = 1$} \\ \cline{6-16}

\rowcolor{gray!20}Normal       & Normal       && 0.05 &&  0.05 & 0.04 & 0.04 && 0.05 & 0.04 & 0.04 && 0.04 & 0.04 & 0.04 \\ 
\rowcolor{gray!20}             &              && 0.10 &&  0.10 & 0.09 & 0.08 && 0.10 & 0.09 & 0.08 && 0.09 & 0.09 & 0.08 \\ 
\rowcolor{gray!20}             & Heavy tailed && 0.05 &&  0.06 & 0.06 & 0.06 && 0.06 & 0.06 & 0.06 && 0.06 & 0.06 & 0.05 \\ 
\rowcolor{gray!20}             &              && 0.10 &&  0.12 & 0.12 & 0.11 && 0.12 & 0.12 & 0.11 && 0.11 & 0.12 & 0.09 \\ 
\rowcolor{gray!20}             & Skewed       && 0.05 &&  0.04 & 0.05 & 0.05 && 0.04 & 0.05 & 0.05 && 0.04 & 0.05 & 0.04 \\ 
\rowcolor{gray!20}             &              && 0.10 &&  0.09 & 0.08 & 0.09 && 0.09 & 0.08 & 0.09 && 0.08 & 0.09 & 0.09 \\ 
             &&&&&&&&&&&&&&&\\
Heavy tailed & Normal       && 0.05 &&  1.00 & 1.00 & 1.00 && 1.00 & 1.00 & 1.00 && 1.00 & 1.00 & 1.00 \\ 
             &              && 0.10 &&  1.00 & 1.00 & 1.00 && 1.00 & 1.00 & 1.00 && 1.00 & 1.00 & 1.00 \\ 
             & Heavy tailed && 0.05 &&  1.00 & 1.00 & 1.00 && 1.00 & 1.00 & 1.00 && 1.00 & 1.00 & 1.00 \\ 
             &              && 0.10 &&  1.00 & 1.00 & 1.00 && 1.00 & 1.00 & 1.00 && 1.00 & 1.00 & 1.00 \\ 
             & Skewed       && 0.05 &&  1.00 & 1.00 & 1.00 && 1.00 & 1.00 & 1.00 && 1.00 & 1.00 & 1.00 \\ 
             &              && 0.10 &&  1.00 & 1.00 & 1.00 && 1.00 & 1.00 & 1.00 && 1.00 & 1.00 & 1.00 \\
             &&&&&&&&&&&&&&&\\ 
Skewed       & Normal       && 0.05 &&  1.00 & 1.00 & 1.00 && 1.00 & 1.00 & 1.00 && 1.00 & 1.00 & 1.00 \\ 
             &              && 0.10 &&  1.00 & 1.00 & 1.00 && 1.00 & 1.00 & 1.00 && 1.00 & 1.00 & 1.00 \\ 
             & Heavy tailed && 0.05 &&  1.00 & 1.00 & 1.00 && 1.00 & 1.00 & 1.00 && 1.00 & 1.00 & 1.00 \\ 
             &              && 0.10 &&  1.00 & 1.00 & 1.00 && 1.00 & 1.00 & 1.00 && 1.00 & 1.00 & 1.00 \\ 
             & Skewed       && 0.05 &&  1.00 & 1.00 & 1.00 && 1.00 & 1.00 & 1.00 && 1.00 & 1.00 & 1.00 \\ 
             &              && 0.10 &&  1.00 & 1.00 & 1.00 && 1.00 & 1.00 & 1.00 && 1.00 & 1.00 & 1.00 \\ 


&&&&&&&&&&&&&&&\\
& && && \multicolumn{9}{c}{$\sigma_{\varepsilon}^2 = 1$, \ \ $\sigma_{b_0}^2 = \sigma_{b_1}^2 = 4$} \\ \cline{6-16}

\rowcolor{gray!20}Normal       & Normal       && 0.05 &&   0.05 & 0.05 & 0.04 && 0.05 & 0.05 & 0.04 && 0.05 & 0.05 & 0.05 \\ 
\rowcolor{gray!20}             &              && 0.10 &&   0.10 & 0.10 & 0.10 && 0.10 & 0.10 & 0.10 && 0.09 & 0.09 & 0.09 \\ 
\rowcolor{gray!20}             & Heavy tailed && 0.05 &&   0.07 & 0.06 & 0.07 && 0.07 & 0.06 & 0.07 && 0.06 & 0.06 & 0.05 \\ 
\rowcolor{gray!20}             &              && 0.10 &&   0.12 & 0.13 & 0.14 && 0.12 & 0.13 & 0.14 && 0.13 & 0.12 & 0.11 \\ 
\rowcolor{gray!20}             & Skewed       && 0.05 &&   0.06 & 0.06 & 0.06 && 0.06 & 0.06 & 0.06 && 0.06 & 0.06 & 0.06 \\ 
\rowcolor{gray!20}             &              && 0.10 &&   0.11 & 0.10 & 0.12 && 0.11 & 0.10 & 0.12 && 0.11 & 0.11 & 0.11 \\ 
             &&&&&&&&&&&&&&&\\
Heavy tailed & Normal       && 0.05 &&   1.00 & 1.00 & 1.00 && 1.00 & 1.00 & 1.00 && 1.00 & 1.00 & 1.00 \\ 
             &              && 0.10 &&   1.00 & 1.00 & 1.00 && 1.00 & 1.00 & 1.00 && 1.00 & 1.00 & 1.00 \\ 
             & Heavy tailed && 0.05 &&   1.00 & 1.00 & 1.00 && 1.00 & 1.00 & 1.00 && 1.00 & 1.00 & 1.00 \\ 
             &              && 0.10 &&   1.00 & 1.00 & 1.00 && 1.00 & 1.00 & 1.00 && 1.00 & 1.00 & 1.00 \\ 
             & Skewed       && 0.05 &&   1.00 & 1.00 & 1.00 && 1.00 & 1.00 & 1.00 && 1.00 & 1.00 & 1.00 \\ 
             &              && 0.10 &&   1.00 & 1.00 & 1.00 && 1.00 & 1.00 & 1.00 && 1.00 & 1.00 & 1.00 \\ 
             &&&&&&&&&&&&&&&\\
Skewed       & Normal       && 0.05 &&   1.00 & 1.00 & 1.00 && 1.00 & 1.00 & 1.00 && 1.00 & 1.00 & 1.00 \\ 
             &              && 0.10 &&   1.00 & 1.00 & 1.00 && 1.00 & 1.00 & 1.00 && 1.00 & 1.00 & 1.00 \\ 
             & Heavy tailed && 0.05 &&   1.00 & 1.00 & 1.00 && 1.00 & 1.00 & 1.00 && 1.00 & 1.00 & 1.00 \\ 
             &              && 0.10 &&   1.00 & 1.00 & 1.00 && 1.00 & 1.00 & 1.00 && 1.00 & 1.00 & 1.00 \\ 
             & Skewed       && 0.05 &&   1.00 & 1.00 & 1.00 && 1.00 & 1.00 & 1.00 && 1.00 & 1.00 & 1.00 \\ 
             &              && 0.10 &&   1.00 & 1.00 & 1.00 && 1.00 & 1.00 & 1.00 && 1.00 & 1.00 & 1.00 \\ 

   \hline
\end{tabular}
\end{scriptsize}
\end{table}


%%% Level-2

\begin{table}[ht]
\centering
\caption{\label{tab:bootb0} Proportion of bootstrap tests rejecting the null hypothesis of normality of the random intercept.}
\begin{scriptsize}
\begin{tabular}{ll p{.1cm} c p{.1cm} rrr p{.1cm} rrr p{.1cm} rrr}
  \hline
  \multicolumn{2}{c}{Distributions}& & Nominal & &  \multicolumn{3}{c}{Raw residuals} & & \multicolumn{3}{c}{Pearson residuals} & & \multicolumn{3}{c}{Studentized residuals}\\ \cline{1-2} \cline{6-8} \cline{10-12} \cline{14-16}
  Random effects & Errors & & $\alpha$ & & AD & CVM & KS & & AD & CVM & KS & & AD & CVM & KS \\ 
   \hline
& && && \multicolumn{9}{c}{$\sigma_{\varepsilon}^2 = 4$, \ \ $\sigma_{b_0}^2 = \sigma_{b_1}^2 = 1$} \\ \cline{6-16}

\rowcolor{gray!20}Normal       & Normal       && 0.05 &&   0.05 & 0.05 & 0.05 && 0.05 & 0.04 & 0.05 && 0.05 & 0.04 & 0.05 \\ 
\rowcolor{gray!20}             &              && 0.10 &&   0.11 & 0.10 & 0.09 && 0.10 & 0.10 & 0.11 && 0.10 & 0.10 & 0.11 \\ 
\rowcolor{gray!20}             & Heavy tailed && 0.05 &&   0.15 & 0.13 & 0.11 && 0.16 & 0.14 & 0.12 && 0.16 & 0.14 & 0.12 \\ 
\rowcolor{gray!20}             &              && 0.10 &&   0.22 & 0.20 & 0.18 && 0.25 & 0.23 & 0.20 && 0.24 & 0.23 & 0.19 \\ 
\rowcolor{gray!20}             & Skewed       && 0.05 &&   0.16 & 0.15 & 0.13 && 0.18 & 0.16 & 0.13 && 0.18 & 0.16 & 0.13 \\ 
\rowcolor{gray!20}             &              && 0.10 &&   0.26 & 0.23 & 0.20 && 0.27 & 0.24 & 0.20 && 0.27 & 0.24 & 0.21 \\ 
             &&&&&&&&&&&&&&&\\
Heavy tailed & Normal       && 0.05 &&   0.27 & 0.24 & 0.19 && 0.28 & 0.25 & 0.20 && 0.28 & 0.25 & 0.20 \\ 
             &              && 0.10 &&   0.36 & 0.31 & 0.26 && 0.35 & 0.33 & 0.28 && 0.35 & 0.33 & 0.27 \\ 
             & Heavy tailed && 0.05 &&   0.49 & 0.45 & 0.34 && 0.50 & 0.45 & 0.35 && 0.49 & 0.45 & 0.35 \\ 
             &              && 0.10 &&   0.59 & 0.54 & 0.45 && 0.59 & 0.55 & 0.45 && 0.58 & 0.55 & 0.45 \\ 
             & Skewed       && 0.05 &&   0.52 & 0.48 & 0.35 && 0.54 & 0.48 & 0.38 && 0.54 & 0.48 & 0.38 \\ 
             &              && 0.10 &&   0.63 & 0.59 & 0.49 && 0.64 & 0.60 & 0.51 && 0.64 & 0.60 & 0.51 \\
             &&&&&&&&&&&&&&&\\ 
Skewed       & Normal       && 0.05 &&   0.51 & 0.48 & 0.38 && 0.51 & 0.47 & 0.37 && 0.51 & 0.47 & 0.37 \\ 
             &              && 0.10 &&   0.62 & 0.59 & 0.49 && 0.59 & 0.58 & 0.50 && 0.59 & 0.58 & 0.50 \\ 
             & Heavy tailed && 0.05 &&   0.73 & 0.70 & 0.57 && 0.73 & 0.69 & 0.57 && 0.73 & 0.69 & 0.57 \\ 
             &              && 0.10 &&   0.80 & 0.77 & 0.67 && 0.80 & 0.77 & 0.69 && 0.80 & 0.77 & 0.69 \\ 
             & Skewed       && 0.05 &&   0.87 & 0.83 & 0.68 && 0.86 & 0.81 & 0.67 && 0.86 & 0.82 & 0.67 \\ 
             &              && 0.10 &&   0.92 & 0.89 & 0.79 && 0.91 & 0.87 & 0.79 && 0.90 & 0.88 & 0.80 \\ 

&&&&&&&&&&&&&&&\\
& && && \multicolumn{9}{c}{$\sigma_{\varepsilon}^2 = 1$, \ \ $\sigma_{b_0}^2 = \sigma_{b_1}^2 = 1$} \\ \cline{6-16}

\rowcolor{gray!20}Normal       & Normal       && 0.05 &&   0.05 & 0.04 & 0.04 && 0.06 & 0.05 & 0.04 && 0.05 & 0.05 & 0.04 \\ 
\rowcolor{gray!20}             &              && 0.10 &&   0.09 & 0.09 & 0.08 && 0.09 & 0.09 & 0.08 && 0.09 & 0.09 & 0.08 \\ 
\rowcolor{gray!20}             & Heavy tailed && 0.05 &&   0.07 & 0.07 & 0.06 && 0.07 & 0.07 & 0.06 && 0.07 & 0.07 & 0.06 \\ 
\rowcolor{gray!20}             &              && 0.10 &&   0.12 & 0.11 & 0.11 && 0.13 & 0.12 & 0.10 && 0.13 & 0.12 & 0.10 \\ 
\rowcolor{gray!20}             & Skewed       && 0.05 &&   0.06 & 0.06 & 0.07 && 0.07 & 0.06 & 0.06 && 0.06 & 0.06 & 0.06 \\ 
\rowcolor{gray!20}             &              && 0.10 &&   0.11 & 0.11 & 0.12 && 0.12 & 0.12 & 0.12 && 0.12 & 0.12 & 0.11 \\ 
             &&&&&&&&&&&&&&&\\
Heavy tailed & Normal       && 0.05 &&   0.54 & 0.49 & 0.41 && 0.52 & 0.48 & 0.40 && 0.52 & 0.48 & 0.40 \\ 
             &              && 0.10 &&   0.63 & 0.59 & 0.50 && 0.61 & 0.57 & 0.51 && 0.61 & 0.56 & 0.51 \\ 
             & Heavy tailed && 0.05 &&   0.62 & 0.57 & 0.48 && 0.60 & 0.55 & 0.46 && 0.60 & 0.55 & 0.47 \\ 
             &              && 0.10 &&   0.70 & 0.67 & 0.59 && 0.68 & 0.64 & 0.56 && 0.68 & 0.64 & 0.55 \\ 
             & Skewed       && 0.05 &&   0.61 & 0.56 & 0.46 && 0.61 & 0.55 & 0.45 && 0.61 & 0.55 & 0.45 \\ 
             &              && 0.10 &&   0.71 & 0.65 & 0.57 && 0.70 & 0.65 & 0.57 && 0.70 & 0.65 & 0.56 \\ 
             &&&&&&&&&&&&&&&\\
Skewed       & Normal       && 0.05 &&   0.92 & 0.91 & 0.85 && 0.93 & 0.91 & 0.85 && 0.93 & 0.91 & 0.85 \\ 
             &              && 0.10 &&   0.95 & 0.94 & 0.91 && 0.96 & 0.94 & 0.90 && 0.96 & 0.94 & 0.90 \\ 
             & Heavy tailed && 0.05 &&   0.97 & 0.95 & 0.89 && 0.97 & 0.96 & 0.88 && 0.97 & 0.96 & 0.88 \\ 
             &              && 0.10 &&   0.99 & 0.98 & 0.94 && 0.99 & 0.98 & 0.94 && 0.99 & 0.98 & 0.94 \\ 
             & Skewed       && 0.05 &&   0.98 & 0.96 & 0.90 && 0.98 & 0.97 & 0.90 && 0.98 & 0.97 & 0.91 \\ 
             &              && 0.10 &&   0.99 & 0.97 & 0.95 && 0.99 & 0.98 & 0.94 && 0.99 & 0.98 & 0.94 \\

&&&&&&&&&&&&&&&\\
& && && \multicolumn{9}{c}{$\sigma_{\varepsilon}^2 = 1$, \ \ $\sigma_{b_0}^2 = \sigma_{b_1}^2 = 4$} \\ \cline{6-16}

\rowcolor{gray!20}Normal       & Normal       && 0.05 &&   0.05 & 0.05 & 0.05 && 0.05 & 0.05 & 0.05 && 0.05 & 0.06 & 0.05 \\ 
\rowcolor{gray!20}             &              && 0.10 &&   0.10 & 0.10 & 0.09 && 0.10 & 0.09 & 0.09 && 0.10 & 0.09 & 0.09 \\ 
\rowcolor{gray!20}             & Heavy tailed && 0.05 &&   0.04 & 0.04 & 0.05 && 0.05 & 0.05 & 0.05 && 0.05 & 0.05 & 0.05 \\ 
\rowcolor{gray!20}             &              && 0.10 &&   0.09 & 0.09 & 0.08 && 0.09 & 0.09 & 0.09 && 0.09 & 0.09 & 0.09 \\ 
\rowcolor{gray!20}             & Skewed       && 0.05 &&   0.04 & 0.04 & 0.03 && 0.04 & 0.04 & 0.04 && 0.04 & 0.04 & 0.03 \\ 
\rowcolor{gray!20}             &              && 0.10 &&   0.09 & 0.09 & 0.08 && 0.10 & 0.09 & 0.08 && 0.10 & 0.09 & 0.08 \\ 
             &&&&&&&&&&&&&&&\\
Heavy tailed & Normal       && 0.05 &&   0.68 & 0.63 & 0.55 && 0.68 & 0.63 & 0.55 && 0.68 & 0.63 & 0.55 \\ 
             &              && 0.10 &&   0.76 & 0.72 & 0.63 && 0.75 & 0.71 & 0.65 && 0.75 & 0.71 & 0.64 \\ 
             & Heavy tailed && 0.05 &&   0.72 & 0.67 & 0.58 && 0.71 & 0.67 & 0.58 && 0.71 & 0.67 & 0.58 \\ 
             &              && 0.10 &&   0.79 & 0.77 & 0.68 && 0.79 & 0.76 & 0.67 && 0.79 & 0.76 & 0.68 \\ 
             & Skewed       && 0.05 &&   0.71 & 0.68 & 0.57 && 0.70 & 0.67 & 0.57 && 0.70 & 0.68 & 0.57 \\ 
             &              && 0.10 &&   0.78 & 0.75 & 0.68 && 0.78 & 0.74 & 0.68 && 0.78 & 0.74 & 0.68 \\ 
             &&&&&&&&&&&&&&&\\
Skewed       & Normal       && 0.05 &&   1.00 & 0.99 & 0.97 && 1.00 & 0.99 & 0.97 && 1.00 & 0.99 & 0.98 \\ 
             &              && 0.10 &&   1.00 & 1.00 & 0.99 && 1.00 & 1.00 & 0.99 && 1.00 & 1.00 & 0.99 \\ 
             & Heavy tailed && 0.05 &&   1.00 & 1.00 & 0.98 && 1.00 & 1.00 & 0.98 && 1.00 & 1.00 & 0.98 \\ 
             &              && 0.10 &&   1.00 & 1.00 & 0.99 && 1.00 & 1.00 & 0.99 && 1.00 & 1.00 & 0.99 \\ 
             & Skewed       && 0.05 &&   1.00 & 1.00 & 0.98 && 1.00 & 1.00 & 0.98 && 1.00 & 1.00 & 0.98 \\ 
             &              && 0.10 &&   1.00 & 1.00 & 0.99 && 1.00 & 1.00 & 0.99 && 1.00 & 1.00 & 0.99 \\ 


   \hline
\end{tabular}
\end{scriptsize}
\end{table}


\begin{table}[ht]
\centering
\caption{\label{tab:bootb1} Proportion of bootstrap tests rejecting the null hypothesis of normality of the random slope.}
\begin{scriptsize}
\begin{tabular}{ll p{.1cm} c p{.1cm} rrr p{.1cm} rrr p{.1cm} rrr}
  \hline
  \multicolumn{2}{c}{Distributions}& & Nominal & &  \multicolumn{3}{c}{Raw residuals} & & \multicolumn{3}{c}{Pearson residuals} & & \multicolumn{3}{c}{Studentized residuals}\\ \cline{1-2} \cline{6-8} \cline{10-12} \cline{14-16}
  Random effects & Errors & & $\alpha$ & & AD & CVM & KS & & AD & CVM & KS & & AD & CVM & KS \\ 
   \hline
& && && \multicolumn{9}{c}{$\sigma_{\varepsilon}^2 = 4$, \ \ $\sigma_{b_0}^2 = \sigma_{b_1}^2 = 1$} \\ \cline{6-16}

\rowcolor{gray!20}Normal       & Normal       && 0.05 &&   0.06 & 0.06 & 0.06 && 0.05 & 0.06 & 0.06 && 0.06 & 0.06 & 0.06 \\ 
\rowcolor{gray!20}             &              && 0.10 &&   0.10 & 0.11 & 0.10 && 0.10 & 0.10 & 0.10 && 0.10 & 0.10 & 0.10 \\ 
\rowcolor{gray!20}             & Heavy tailed && 0.05 &&   0.31 & 0.29 & 0.12 && 0.27 & 0.25 & 0.19 && 0.26 & 0.25 & 0.18 \\ 
\rowcolor{gray!20}             &              && 0.10 &&   0.41 & 0.38 & 0.19 && 0.35 & 0.32 & 0.27 && 0.35 & 0.32 & 0.27 \\ 
\rowcolor{gray!20}             & Skewed       && 0.05 &&   0.23 & 0.20 & 0.21 && 0.33 & 0.32 & 0.24 && 0.33 & 0.31 & 0.24 \\ 
\rowcolor{gray!20}             &              && 0.10 &&   0.34 & 0.32 & 0.31 && 0.41 & 0.38 & 0.34 && 0.41 & 0.38 & 0.34 \\ 
             &&&&&&&&&&&&&&&\\
Heavy tailed & Normal       && 0.05 &&   0.11 & 0.10 & 0.09 && 0.13 & 0.13 & 0.09 && 0.13 & 0.13 & 0.09 \\ 
             &              && 0.10 &&   0.18 & 0.17 & 0.15 && 0.18 & 0.18 & 0.16 && 0.18 & 0.18 & 0.16 \\ 
             & Heavy tailed && 0.05 &&   0.49 & 0.45 & 0.20 && 0.40 & 0.37 & 0.29 && 0.40 & 0.37 & 0.29 \\ 
             &              && 0.10 &&   0.61 & 0.58 & 0.28 && 0.49 & 0.44 & 0.37 && 0.49 & 0.44 & 0.37 \\ 
             & Skewed       && 0.05 &&   0.41 & 0.35 & 0.31 && 0.49 & 0.48 & 0.36 && 0.49 & 0.48 & 0.36 \\ 
             &              && 0.10 &&   0.54 & 0.50 & 0.44 && 0.59 & 0.55 & 0.49 && 0.59 & 0.55 & 0.49 \\
             &&&&&&&&&&&&&&&\\ 
Skewed       & Normal       && 0.05 &&   0.10 & 0.10 & 0.10 && 0.12 & 0.12 & 0.10 && 0.12 & 0.12 & 0.10 \\ 
             &              && 0.10 &&   0.18 & 0.17 & 0.16 && 0.17 & 0.16 & 0.16 && 0.17 & 0.16 & 0.16 \\ 
             & Heavy tailed && 0.05 &&   0.41 & 0.38 & 0.22 && 0.41 & 0.37 & 0.30 && 0.41 & 0.37 & 0.30 \\ 
             &              && 0.10 &&   0.55 & 0.50 & 0.31 && 0.51 & 0.47 & 0.38 && 0.51 & 0.47 & 0.38 \\ 
             & Skewed       && 0.05 &&   0.31 & 0.25 & 0.41 && 0.59 & 0.57 & 0.46 && 0.60 & 0.57 & 0.46 \\ 
             &              && 0.10 &&   0.43 & 0.38 & 0.53 && 0.70 & 0.66 & 0.57 && 0.70 & 0.65 & 0.57 \\ 

&&&&&&&&&&&&&&&\\
& && && \multicolumn{9}{c}{$\sigma_{\varepsilon}^2 = 1$, \ \ $\sigma_{b_0}^2 = \sigma_{b_1}^2 = 1$} \\ \cline{6-16}

\rowcolor{gray!20}Normal       & Normal       && 0.05 &&  0.06 & 0.06 & 0.05 && 0.06 & 0.06 & 0.06 && 0.06 & 0.06 & 0.06 \\ 
\rowcolor{gray!20}             &              && 0.10 &&  0.12 & 0.12 & 0.10 && 0.12 & 0.12 & 0.11 && 0.12 & 0.12 & 0.11 \\ 
\rowcolor{gray!20}             & Heavy tailed && 0.05 &&  0.17 & 0.16 & 0.08 && 0.15 & 0.14 & 0.11 && 0.15 & 0.14 & 0.12 \\ 
\rowcolor{gray!20}             &              && 0.10 &&  0.25 & 0.23 & 0.14 && 0.23 & 0.20 & 0.18 && 0.23 & 0.21 & 0.18 \\ 
\rowcolor{gray!20}             & Skewed       && 0.05 &&  0.12 & 0.12 & 0.10 && 0.16 & 0.14 & 0.11 && 0.16 & 0.14 & 0.11 \\ 
\rowcolor{gray!20}             &              && 0.10 &&  0.20 & 0.19 & 0.16 && 0.23 & 0.21 & 0.18 && 0.23 & 0.21 & 0.18 \\ 
             &&&&&&&&&&&&&&&\\
Heavy tailed & Normal       && 0.05 &&  0.28 & 0.25 & 0.13 && 0.28 & 0.24 & 0.20 && 0.28 & 0.24 & 0.20 \\ 
             &              && 0.10 &&  0.36 & 0.33 & 0.20 && 0.36 & 0.32 & 0.28 && 0.36 & 0.32 & 0.28 \\ 
             & Heavy tailed && 0.05 &&  0.50 & 0.47 & 0.20 && 0.45 & 0.41 & 0.34 && 0.45 & 0.41 & 0.34 \\ 
             &              && 0.10 &&  0.58 & 0.56 & 0.28 && 0.51 & 0.49 & 0.43 && 0.51 & 0.49 & 0.43 \\ 
             & Skewed       && 0.05 &&  0.44 & 0.40 & 0.20 && 0.43 & 0.40 & 0.32 && 0.43 & 0.40 & 0.32 \\ 
             &              && 0.10 &&  0.52 & 0.50 & 0.30 && 0.53 & 0.48 & 0.43 && 0.53 & 0.48 & 0.43 \\
             &&&&&&&&&&&&&&&\\ 
Skewed       & Normal       && 0.05 &&  0.30 & 0.25 & 0.30 && 0.48 & 0.43 & 0.35 && 0.48 & 0.43 & 0.35 \\ 
             &              && 0.10 &&  0.39 & 0.35 & 0.39 && 0.58 & 0.54 & 0.47 && 0.58 & 0.54 & 0.47 \\ 
             & Heavy tailed && 0.05 &&  0.46 & 0.39 & 0.45 && 0.66 & 0.61 & 0.52 && 0.66 & 0.62 & 0.52 \\ 
             &              && 0.10 &&  0.53 & 0.49 & 0.56 && 0.74 & 0.71 & 0.63 && 0.74 & 0.71 & 0.63 \\ 
             & Skewed       && 0.05 &&  0.35 & 0.28 & 0.51 && 0.76 & 0.71 & 0.59 && 0.76 & 0.71 & 0.59 \\ 
             &              && 0.10 &&  0.45 & 0.37 & 0.63 && 0.83 & 0.79 & 0.70 && 0.83 & 0.79 & 0.70 \\ 


&&&&&&&&&&&&&&&\\
& && && \multicolumn{9}{c}{$\sigma_{\varepsilon}^2 = 1$, \ \ $\sigma_{b_0}^2 = \sigma_{b_1}^2 = 4$} \\ \cline{6-16}

\rowcolor{gray!20} Normal       & Normal       && 0.05 &&  0.05 & 0.05 & 0.05 && 0.04 & 0.05 & 0.05 && 0.04 & 0.05 & 0.05 \\ 
\rowcolor{gray!20}              &              && 0.10 &&  0.11 & 0.11 & 0.11 && 0.10 & 0.10 & 0.10 && 0.10 & 0.10 & 0.10 \\ 
\rowcolor{gray!20}              & Heavy tailed && 0.05 &&  0.07 & 0.07 & 0.05 && 0.07 & 0.07 & 0.07 && 0.07 & 0.07 & 0.07 \\ 
\rowcolor{gray!20}              &              && 0.10 &&  0.12 & 0.12 & 0.11 && 0.11 & 0.12 & 0.13 && 0.11 & 0.12 & 0.13 \\ 
\rowcolor{gray!20}              & Skewed       && 0.05 &&  0.06 & 0.06 & 0.04 && 0.06 & 0.06 & 0.05 && 0.06 & 0.06 & 0.05 \\ 
\rowcolor{gray!20}              &              && 0.10 &&  0.10 & 0.11 & 0.09 && 0.11 & 0.11 & 0.11 && 0.11 & 0.11 & 0.11 \\
              &&&&&&&&&&&&&&&\\
 Heavy tailed & Normal       && 0.05 &&  0.49 & 0.45 & 0.18 && 0.46 & 0.41 & 0.34 && 0.46 & 0.41 & 0.34 \\ 
              &              && 0.10 &&  0.58 & 0.54 & 0.28 && 0.56 & 0.51 & 0.44 && 0.56 & 0.51 & 0.44 \\ 
              & Heavy tailed && 0.05 &&  0.57 & 0.54 & 0.24 && 0.54 & 0.50 & 0.43 && 0.54 & 0.50 & 0.43 \\ 
              &              && 0.10 &&  0.66 & 0.62 & 0.32 && 0.63 & 0.60 & 0.52 && 0.63 & 0.60 & 0.52 \\ 
              & Skewed       && 0.05 &&  0.52 & 0.48 & 0.22 && 0.49 & 0.46 & 0.37 && 0.48 & 0.46 & 0.37 \\ 
              &              && 0.10 &&  0.61 & 0.57 & 0.32 && 0.56 & 0.53 & 0.47 && 0.56 & 0.53 & 0.47 \\ 
              &&&&&&&&&&&&&&&\\
 Skewed       & Normal       && 0.05 &&  0.58 & 0.49 & 0.69 && 0.90 & 0.87 & 0.74 && 0.90 & 0.87 & 0.74 \\ 
              &              && 0.10 &&  0.68 & 0.59 & 0.79 && 0.94 & 0.92 & 0.83 && 0.94 & 0.92 & 0.83 \\ 
              & Heavy tailed && 0.05 &&  0.61 & 0.52 & 0.76 && 0.92 & 0.91 & 0.80 && 0.93 & 0.91 & 0.80 \\ 
              &              && 0.10 &&  0.71 & 0.61 & 0.84 && 0.96 & 0.95 & 0.89 && 0.96 & 0.95 & 0.89 \\ 
              & Skewed       && 0.05 &&  0.59 & 0.47 & 0.72 && 0.92 & 0.90 & 0.79 && 0.92 & 0.90 & 0.79 \\ 
              &              && 0.10 &&  0.69 & 0.58 & 0.83 && 0.95 & 0.93 & 0.88 && 0.95 & 0.93 & 0.88 \\ 


   \hline
\end{tabular}
\end{scriptsize}
\end{table}


%%% Marginal residuals
\begin{table}[ht]
\centering
\caption{\label{tab:bootmarginal}Bootstrap tests for normality of marginal residuals.}
\begin{scriptsize}
\begin{tabular}{ll p{.1cm} c p{.1cm} rrr p{.1cm} rrr}
  \hline
  \multicolumn{2}{c}{Distributions}& & Nominal & &  \multicolumn{3}{c}{Raw residuals} & & \multicolumn{3}{c}{Cholesky residuals} \\ \cline{1-2} \cline{6-8} \cline{10-12}
  Errors & Random effects & & $\alpha$ & & AD & CVM & KS & & AD & CVM & KS \\ 
   \hline
& && && \multicolumn{6}{c}{$\sigma_{\varepsilon}^2 = 4$, \ \ $\sigma_{b_0}^2 = \sigma_{b_1}^2 = 1$} \\ \cline{6-12}

\rowcolor{gray!20}Normal       & Normal       && 0.05 &&  0.04 & 0.04 & 0.05 && 0.05 & 0.06 & 0.05 \\ 
\rowcolor{gray!20}             &              && 0.10 &&  0.09 & 0.10 & 0.10 && 0.11 & 0.10 & 0.11 \\ 
\rowcolor{gray!20}             & Heavy tailed && 0.05 &&  0.20 & 0.17 & 0.13 && 0.09 & 0.07 & 0.05 \\ 
\rowcolor{gray!20}             &              && 0.10 &&  0.26 & 0.23 & 0.19 && 0.14 & 0.12 & 0.10 \\ 
\rowcolor{gray!20}             & Skewed       && 0.05 &&  0.28 & 0.25 & 0.21 && 0.04 & 0.04 & 0.05 \\ 
\rowcolor{gray!20}             &              && 0.10 &&  0.36 & 0.34 & 0.30 && 0.09 & 0.09 & 0.09 \\ 
             &&&&&&&&&&&\\
Heavy tailed & Normal       && 0.05 &&  1.00 & 1.00 & 0.99 && 1.00 & 1.00 & 1.00 \\ 
             &              && 0.10 &&  1.00 & 1.00 & 0.99 && 1.00 & 1.00 & 1.00 \\ 
             & Heavy tailed && 0.05 &&  1.00 & 1.00 & 1.00 && 1.00 & 1.00 & 1.00 \\ 
             &              && 0.10 &&  1.00 & 1.00 & 1.00 && 1.00 & 1.00 & 1.00 \\ 
             & Skewed       && 0.05 &&  1.00 & 1.00 & 1.00 && 1.00 & 1.00 & 1.00 \\ 
             &              && 0.10 &&  1.00 & 1.00 & 1.00 && 1.00 & 1.00 & 1.00 \\ 
             &&&&&&&&&&&\\
Skewed       & Normal       && 0.05 &&  1.00 & 1.00 & 1.00 && 1.00 & 1.00 & 1.00 \\ 
             &              && 0.10 &&  1.00 & 1.00 & 1.00 && 1.00 & 1.00 & 1.00 \\ 
             & Heavy tailed && 0.05 &&  1.00 & 1.00 & 1.00 && 1.00 & 1.00 & 1.00 \\ 
             &              && 0.10 &&  1.00 & 1.00 & 1.00 && 1.00 & 1.00 & 1.00 \\ 
             & Skewed       && 0.05 &&  1.00 & 1.00 & 1.00 && 1.00 & 1.00 & 1.00 \\ 
             &              && 0.10 &&  1.00 & 1.00 & 1.00 && 1.00 & 1.00 & 1.00 \\ 

&&&&&&&&&&&\\
& && && \multicolumn{6}{c}{$\sigma_{\varepsilon}^2 = 1$, \ \ $\sigma_{b_0}^2 = \sigma_{b_1}^2 = 1$} \\ \cline{6-12}

\rowcolor{gray!20}Normal       & Normal       && 0.05 &&   0.04 & 0.04 & 0.04 && 0.05 & 0.05 & 0.04 \\ 
\rowcolor{gray!20}             &              && 0.10 &&   0.10 & 0.11 & 0.11 && 0.10 & 0.10 & 0.09 \\ 
\rowcolor{gray!20}             & Heavy tailed && 0.05 &&   0.72 & 0.66 & 0.58 && 1.00 & 1.00 & 1.00 \\ 
\rowcolor{gray!20}             &              && 0.10 &&   0.81 & 0.77 & 0.71 && 1.00 & 1.00 & 1.00 \\ 
\rowcolor{gray!20}             & Skewed       && 0.05 &&   0.94 & 0.90 & 0.86 && 1.00 & 1.00 & 1.00 \\ 
\rowcolor{gray!20}             &              && 0.10 &&   0.97 & 0.95 & 0.94 && 1.00 & 1.00 & 1.00 \\ 
             &&&&&&&&&&&\\
Heavy tailed & Normal       && 0.05 &&   0.35 & 0.31 & 0.27 && 0.10 & 0.09 & 0.07 \\ 
             &              && 0.10 &&   0.44 & 0.40 & 0.36 && 0.16 & 0.14 & 0.13 \\ 
             & Heavy tailed && 0.05 &&   0.97 & 0.96 & 0.92 && 1.00 & 1.00 & 1.00 \\ 
             &              && 0.10 &&   0.99 & 0.98 & 0.96 && 1.00 & 1.00 & 1.00 \\ 
             & Skewed       && 0.05 &&   1.00 & 1.00 & 1.00 && 1.00 & 1.00 & 1.00 \\ 
             &              && 0.10 &&   1.00 & 1.00 & 1.00 && 1.00 & 1.00 & 1.00 \\
             &&&&&&&&&&&\\ 
Skewed       & Normal       && 0.05 &&   0.72 & 0.70 & 0.64 && 0.12 & 0.10 & 0.09 \\ 
             &              && 0.10 &&   0.81 & 0.78 & 0.74 && 0.19 & 0.17 & 0.16 \\ 
             & Heavy tailed && 0.05 &&   1.00 & 0.99 & 0.97 && 1.00 & 1.00 & 1.00 \\ 
             &              && 0.10 &&   1.00 & 1.00 & 0.99 && 1.00 & 1.00 & 1.00 \\ 
             & Skewed       && 0.05 &&   1.00 & 1.00 & 1.00 && 1.00 & 1.00 & 1.00 \\ 
             &              && 0.10 &&   1.00 & 1.00 & 1.00 && 1.00 & 1.00 & 1.00 \\ 



&&&&&&&&&&&\\
& && && \multicolumn{6}{c}{$\sigma_{\varepsilon}^2 = 1$, \ \ $\sigma_{b_0}^2 = \sigma_{b_1}^2 = 4$} \\ \cline{6-12}

\rowcolor{gray!20}Normal       & Normal       && 0.05 &&   0.04 & 0.04 & 0.04 && 0.05 & 0.05 & 0.04 \\ 
\rowcolor{gray!20}             &              && 0.10 &&   0.10 & 0.10 & 0.10 && 0.10 & 0.09 & 0.08 \\ 
\rowcolor{gray!20}             & Heavy tailed && 0.05 &&   0.40 & 0.36 & 0.33 && 0.17 & 0.14 & 0.11 \\ 
\rowcolor{gray!20}             &              && 0.10 &&   0.49 & 0.44 & 0.41 && 0.26 & 0.24 & 0.20 \\ 
\rowcolor{gray!20}             & Skewed       && 0.05 &&   0.81 & 0.79 & 0.74 && 0.33 & 0.29 & 0.24 \\ 
\rowcolor{gray!20}             &              && 0.10 &&   0.88 & 0.86 & 0.83 && 0.44 & 0.40 & 0.34 \\ 
             &&&&&&&&&&&\\
Heavy tailed & Normal       && 0.05 &&   0.17 & 0.15 & 0.16 && 1.00 & 1.00 & 1.00 \\ 
             &              && 0.10 &&   0.27 & 0.27 & 0.30 && 1.00 & 1.00 & 1.00 \\ 
             & Heavy tailed && 0.05 &&   0.64 & 0.60 & 0.54 && 1.00 & 1.00 & 1.00 \\ 
             &              && 0.10 &&   0.74 & 0.70 & 0.66 && 1.00 & 1.00 & 1.00 \\ 
             & Skewed       && 0.05 &&   0.93 & 0.91 & 0.89 && 1.00 & 1.00 & 1.00 \\ 
             &              && 0.10 &&   0.96 & 0.95 & 0.94 && 1.00 & 1.00 & 1.00 \\ 
             &&&&&&&&&&&\\
Skewed       & Normal       && 0.05 &&   0.15 & 0.16 & 0.18 && 1.00 & 1.00 & 1.00 \\ 
             &              && 0.10 &&   0.26 & 0.27 & 0.30 && 1.00 & 1.00 & 1.00 \\ 
             & Heavy tailed && 0.05 &&   0.66 & 0.62 & 0.61 && 1.00 & 1.00 & 1.00 \\ 
             &              && 0.10 &&   0.77 & 0.75 & 0.74 && 1.00 & 1.00 & 1.00 \\ 
             & Skewed       && 0.05 &&   0.96 & 0.95 & 0.93 && 1.00 & 1.00 & 1.00 \\ 
             &              && 0.10 &&   0.98 & 0.97 & 0.97 && 1.00 & 1.00 & 1.00 \\ 


   \hline
\end{tabular}
\end{scriptsize}
\end{table}

%------------------------------------------------
% Results of Lange and Ryan's method
%------------------------------------------------


\begin{table}[ht]
\centering
\caption{\label{tab:langeryan}Bootstrap tests of the weighted Q-Q plots for the random effects.}
\begin{scriptsize}
\begin{tabular}{ll p{.1cm} c p{.1cm} cc}
  \hline
  \multicolumn{2}{c}{Distributions} & & Nominal & & & \\ \cline{1-2}
  Random effects & Errors && $\alpha$ && $b_0$ & $b_1$ \\ 
  \hline
  & && && \multicolumn{2}{c}{$\sigma_{\varepsilon}^2 = 4$, \ \ $\sigma_{b_0}^2 = \sigma_{b_1}^2 = 1$} \\ \cline{6-7}
\rowcolor{gray!20}Normal       & Normal       && 0.05 &&  0.06 & 0.05 \\ 
\rowcolor{gray!20}             &              && 0.10 &&  0.10 & 0.11 \\ 
\rowcolor{gray!20}             & Heavy tailed && 0.05 &&  0.12 & 0.13 \\ 
\rowcolor{gray!20}             &              && 0.10 &&  0.18 & 0.18 \\ 
\rowcolor{gray!20}             & Skewed       && 0.05 &&  0.15 & 0.14 \\ 
\rowcolor{gray!20}             &              && 0.10 &&  0.23 & 0.22 \\ 
             &&&&&&\\
Heavy tailed & Normal       && 0.05 &&  0.20 & 0.06 \\ 
             &              && 0.10 &&  0.30 & 0.13 \\ 
             & Heavy tailed && 0.05 &&  0.34 & 0.18 \\ 
             &              && 0.10 &&  0.45 & 0.26 \\ 
             & Skewed       && 0.05 &&  0.41 & 0.22 \\ 
             &              && 0.10 &&  0.53 & 0.32 \\ 
             &&&&&&\\
Skewed       & Normal       && 0.05 &&  0.46 & 0.07 \\ 
             &              && 0.10 &&  0.59 & 0.14 \\ 
             & Heavy tailed && 0.05 &&  0.64 & 0.15 \\ 
             &              && 0.10 &&  0.73 & 0.26 \\ 
             & Skewed       && 0.05 &&  0.72 & 0.24 \\ 
             &              && 0.10 &&  0.82 & 0.36 \\

&&&&&&\\             
  & && && \multicolumn{2}{c}{$\sigma_{\varepsilon}^2 = 1$, \ \ $\sigma_{b_0}^2 = \sigma_{b_1}^2 = 1$} \\ \cline{6-7}
\rowcolor{gray!20}Normal       & Normal       && 0.05 &&  0.04 & 0.04 \\ 
\rowcolor{gray!20}             &              && 0.10 &&  0.09 & 0.10 \\ 
\rowcolor{gray!20}             & Heavy tailed && 0.05 &&  0.06 & 0.07 \\ 
\rowcolor{gray!20}             &              && 0.10 &&  0.10 & 0.12 \\ 
\rowcolor{gray!20}             & Skewed       && 0.05 &&  0.07 & 0.08 \\ 
\rowcolor{gray!20}             &              && 0.10 &&  0.13 & 0.14 \\ 
             &&&&&&\\
Heavy tailed & Normal       && 0.05 &&  0.40 & 0.12 \\ 
             &              && 0.10 &&  0.48 & 0.19 \\ 
             & Heavy tailed && 0.05 &&  0.46 & 0.18 \\ 
             &              && 0.10 &&  0.55 & 0.27 \\ 
             & Skewed       && 0.05 &&  0.45 & 0.18 \\ 
             &              && 0.10 &&  0.56 & 0.27 \\ 
             &&&&&&\\
Skewed       & Normal       && 0.05 &&  0.89 & 0.20 \\ 
             &              && 0.10 &&  0.93 & 0.30 \\ 
             & Heavy tailed && 0.05 &&  0.92 & 0.27 \\ 
             &              && 0.10 &&  0.96 & 0.38 \\ 
             & Skewed       && 0.05 &&  0.93 & 0.31 \\ 
             &              && 0.10 &&  0.96 & 0.41 \\ 

&&&&&&\\
  & && && \multicolumn{2}{c}{$\sigma_{\varepsilon}^2 = 1$, \ \ $\sigma_{b_0}^2 = \sigma_{b_1}^2 = 4$} \\ \cline{6-7}
\rowcolor{gray!20}Normal       & Normal       && 0.05 &&  0.05 & 0.05 \\ 
\rowcolor{gray!20}             &              && 0.10 &&  0.10 & 0.10 \\ 
\rowcolor{gray!20}             & Heavy tailed && 0.05 &&  0.05 & 0.06 \\ 
\rowcolor{gray!20}             &              && 0.10 &&  0.09 & 0.10 \\ 
\rowcolor{gray!20}             & Skewed       && 0.05 &&  0.04 & 0.06 \\ 
\rowcolor{gray!20}             &              && 0.10 &&  0.09 & 0.10 \\ 
             &&&&&&\\
Heavy tailed & Normal       && 0.05 &&  0.55 & 0.19 \\ 
             &              && 0.10 &&  0.63 & 0.28 \\ 
             & Heavy tailed && 0.05 &&  0.58 & 0.24 \\ 
             &              && 0.10 &&  0.68 & 0.35 \\ 
             & Skewed       && 0.05 &&  0.58 & 0.23 \\ 
             &              && 0.10 &&  0.66 & 0.33 \\ 
             &&&&&&\\
Skewed       & Normal       && 0.05 &&  0.99 & 0.46 \\ 
             &              && 0.10 &&  1.00 & 0.59 \\ 
             & Heavy tailed && 0.05 &&  0.99 & 0.49 \\ 
             &              && 0.10 &&  1.00 & 0.64 \\ 
             & Skewed       && 0.05 &&  0.99 & 0.47 \\ 
             &              && 0.10 &&  1.00 & 0.60 \\ 


   \hline
\end{tabular}
\end{scriptsize}
\end{table}


\section{Full results from the simulation study}\label{supp:simstudy}
%--------------------------------------------------

In the paper we described a simulation study and only presented results for the the Anderson-Darling test under one variance structure ($\sigma^2_\varepsilon = 4$ and $\sigma^2_{b_0} = \sigma^2_{b_1} = 1$). In this section we present the results from the full simulation study. Tables~\ref{tab:simb0sB}--\ref{tab:simb030} present the results for the rotated random intercept and Tables~\ref{tab:simb1sB}--\ref{tab:simb130} present the results for the rotated random slope. We use a gray background to highlight the simulation settings under which the tests should fail to reject the null hypothesis of normality.


\begin{table}[ht]
\centering
\caption{\label{tab:simb0sB} Proportion of tests rejecting normality of the random intercept using two rotations and $s = \rank(\bm{B})$.}
\begin{scriptsize}
\begin{tabular}{ll p{.1cm} c p{.1cm} rrr p{.1cm} rrr}
  \hline
  \multicolumn{2}{c}{Distributions}& & Nominal & &  \multicolumn{3}{c}{Rotation} & & \multicolumn{3}{c}{Varimax rotation} \\ \cline{1-2} \cline{6-8} \cline{10-12}   
  Random effects & Errors & & $\alpha$ & & AD & CVM & KS & & AD & CVM & KS \\ 
   \hline
& && && \multicolumn{7}{c}{$\sigma_{\varepsilon}^2 = 4$, \ \ $\sigma_{b_0}^2 = \sigma_{b_1}^2 = 1$} \\ \cline{6-12}
\rowcolor{gray!20}Normal       & Normal       && 0.05 &&   0.04 & 0.04 & 0.04 && 0.05 & 0.05 & 0.06 \\ 
\rowcolor{gray!20}             &              && 0.10 &&   0.10 & 0.10 & 0.10 && 0.11 & 0.11 & 0.11 \\ 
\rowcolor{gray!20}             & Heavy tailed && 0.05 &&   0.07 & 0.07 & 0.07 && 0.09 & 0.08 & 0.07 \\ 
\rowcolor{gray!20}             &              && 0.10 &&   0.13 & 0.13 & 0.13 && 0.16 & 0.15 & 0.14 \\ 
\rowcolor{gray!20}             & Skewed       && 0.05 &&   0.05 & 0.05 & 0.05 && 0.05 & 0.05 & 0.05 \\ 
\rowcolor{gray!20}             &              && 0.10 &&   0.10 & 0.09 & 0.09 && 0.10 & 0.10 & 0.11 \\ 
             &&&&&&&&&&&\\
Heavy tailed & Normal       && 0.05 &&   0.14 & 0.13 & 0.10 && 0.22 & 0.20 & 0.17 \\ 
             &              && 0.10 &&   0.20 & 0.19 & 0.18 && 0.31 & 0.28 & 0.24 \\ 
             & Heavy tailed && 0.05 &&   0.19 & 0.17 & 0.15 && 0.34 & 0.32 & 0.26 \\ 
             &              && 0.10 &&   0.26 & 0.24 & 0.21 && 0.44 & 0.41 & 0.35 \\ 
             & Skewed       && 0.05 &&   0.15 & 0.14 & 0.13 && 0.28 & 0.23 & 0.20 \\ 
             &              && 0.10 &&   0.21 & 0.18 & 0.18 && 0.37 & 0.32 & 0.28 \\ 
             &&&&&&&&&&&\\
Skewed       & Normal       && 0.05 &&   0.10 & 0.09 & 0.08 && 0.20 & 0.17 & 0.12 \\ 
             &              && 0.10 &&   0.17 & 0.15 & 0.15 && 0.29 & 0.25 & 0.19 \\ 
             & Heavy tailed && 0.05 &&   0.13 & 0.11 & 0.11 && 0.30 & 0.24 & 0.19 \\ 
             &              && 0.10 &&   0.22 & 0.19 & 0.17 && 0.39 & 0.34 & 0.28 \\ 
             & Skewed       && 0.05 &&   0.13 & 0.12 & 0.09 && 0.22 & 0.19 & 0.15 \\ 
             &              && 0.10 &&   0.19 & 0.17 & 0.16 && 0.33 & 0.28 & 0.25 \\ 

&&&&&&&&&&&\\
& && && \multicolumn{7}{c}{$\sigma_{\varepsilon}^2 = 1$, \ \ $\sigma_{b_0}^2 = \sigma_{b_1}^2 = 1$} \\ \cline{6-12}
\rowcolor{gray!20}Normal       & Normal       && 0.05 &&   0.05 & 0.05 & 0.04 && 0.05 & 0.06 & 0.05 \\ 
\rowcolor{gray!20}             &              && 0.10 &&   0.09 & 0.09 & 0.09 && 0.11 & 0.11 & 0.12 \\ 
\rowcolor{gray!20}             & Heavy tailed && 0.05 &&   0.06 & 0.07 & 0.06 && 0.06 & 0.06 & 0.05 \\ 
\rowcolor{gray!20}             &              && 0.10 &&   0.11 & 0.10 & 0.11 && 0.11 & 0.11 & 0.11 \\ 
\rowcolor{gray!20}             & Skewed       && 0.05 &&   0.05 & 0.04 & 0.04 && 0.05 & 0.05 & 0.04 \\ 
\rowcolor{gray!20}             &              && 0.10 &&   0.09 & 0.09 & 0.09 && 0.10 & 0.10 & 0.09 \\ 
             &&&&&&&&&&&\\
Heavy tailed & Normal       && 0.05 &&   0.21 & 0.20 & 0.16 && 0.40 & 0.35 & 0.28 \\ 
             &              && 0.10 &&   0.29 & 0.27 & 0.23 && 0.49 & 0.45 & 0.39 \\ 
             & Heavy tailed && 0.05 &&   0.25 & 0.22 & 0.17 && 0.47 & 0.42 & 0.34 \\ 
             &              && 0.10 &&   0.32 & 0.30 & 0.25 && 0.54 & 0.50 & 0.44 \\ 
             & Skewed       && 0.05 &&   0.25 & 0.22 & 0.17 && 0.42 & 0.38 & 0.30 \\ 
             &              && 0.10 &&   0.33 & 0.30 & 0.27 && 0.50 & 0.46 & 0.40 \\ 
             &&&&&&&&&&&\\
Skewed       & Normal       && 0.05 &&   0.17 & 0.16 & 0.13 && 0.36 & 0.28 & 0.21 \\ 
             &              && 0.10 &&   0.26 & 0.24 & 0.20 && 0.47 & 0.38 & 0.31 \\ 
             & Heavy tailed && 0.05 &&   0.20 & 0.18 & 0.14 && 0.42 & 0.34 & 0.25 \\ 
             &              && 0.10 &&   0.28 & 0.26 & 0.23 && 0.51 & 0.44 & 0.35 \\ 
             & Skewed       && 0.05 &&   0.18 & 0.17 & 0.12 && 0.40 & 0.30 & 0.21 \\ 
             &              && 0.10 &&   0.26 & 0.23 & 0.22 && 0.51 & 0.39 & 0.33 \\ 

&&&&&&&&&&&\\
& && && \multicolumn{7}{c}{$\sigma_{\varepsilon}^2 = 1$, \ \ $\sigma_{b_0}^2 = \sigma_{b_1}^2 = 4$} \\ \cline{6-12}
\rowcolor{gray!20}Normal       & Normal       && 0.05 &&  0.05 & 0.05 & 0.04 && 0.05 & 0.05 & 0.04 \\ 
\rowcolor{gray!20}             &              && 0.10 &&  0.10 & 0.09 & 0.10 && 0.10 & 0.10 & 0.10 \\ 
\rowcolor{gray!20}             & Heavy tailed && 0.05 &&  0.06 & 0.06 & 0.06 && 0.05 & 0.06 & 0.05 \\ 
\rowcolor{gray!20}             &              && 0.10 &&  0.12 & 0.12 & 0.11 && 0.10 & 0.10 & 0.10 \\ 
\rowcolor{gray!20}             & Skewed       && 0.05 &&  0.04 & 0.05 & 0.04 && 0.04 & 0.04 & 0.04 \\ 
\rowcolor{gray!20}             &              && 0.10 &&  0.10 & 0.10 & 0.11 && 0.08 & 0.08 & 0.10 \\ 
             &&&&&&&&&&&\\
Heavy tailed & Normal       && 0.05 &&  0.28 & 0.25 & 0.21 && 0.43 & 0.38 & 0.29 \\ 
             &              && 0.10 &&  0.34 & 0.32 & 0.29 && 0.53 & 0.49 & 0.41 \\ 
             & Heavy tailed && 0.05 &&  0.30 & 0.28 & 0.21 && 0.44 & 0.40 & 0.33 \\ 
             &              && 0.10 &&  0.37 & 0.34 & 0.29 && 0.54 & 0.50 & 0.42 \\ 
             & Skewed       && 0.05 &&  0.28 & 0.26 & 0.21 && 0.44 & 0.40 & 0.31 \\ 
             &              && 0.10 &&  0.37 & 0.34 & 0.29 && 0.53 & 0.47 & 0.40 \\ 
             &&&&&&&&&&&\\
Skewed       & Normal       && 0.05 &&  0.24 & 0.22 & 0.15 && 0.37 & 0.30 & 0.23 \\ 
             &              && 0.10 &&  0.33 & 0.30 & 0.24 && 0.49 & 0.41 & 0.33 \\ 
             & Heavy tailed && 0.05 &&  0.26 & 0.23 & 0.18 && 0.38 & 0.29 & 0.22 \\ 
             &              && 0.10 &&  0.33 & 0.33 & 0.26 && 0.48 & 0.39 & 0.32 \\ 
             & Skewed       && 0.05 &&  0.24 & 0.21 & 0.15 && 0.38 & 0.29 & 0.23 \\ 
             &              && 0.10 &&  0.33 & 0.31 & 0.26 && 0.50 & 0.39 & 0.33 \\ 

\hline
\end{tabular}
\end{scriptsize}
\end{table}


\begin{table}[ht]
\centering
\caption{\label{tab:simb0s55} Proportion of tests rejecting normality of the random intercept using two rotations and $s = 55$.}
\begin{scriptsize}
\begin{tabular}{ll p{.1cm} c p{.1cm} rrr p{.1cm} rrr}
  \hline
  \multicolumn{2}{c}{Distributions}& & Nominal & &  \multicolumn{3}{c}{Rotation} & & \multicolumn{3}{c}{Varimax rotation} \\ \cline{1-2} \cline{6-8} \cline{10-12}   
  Random effects & Errors & & $\alpha$ & & AD & CVM & KS & & AD & CVM & KS \\ 
   \hline
& && && \multicolumn{7}{c}{$\sigma_{\varepsilon}^2 = 4$, \ \ $\sigma_{b_0}^2 = \sigma_{b_1}^2 = 1$} \\ \cline{6-12}
\rowcolor{gray!20}Normal       & Normal       && 0.05 &&   0.04 & 0.04 & 0.04 && 0.05 & 0.06 & 0.05 \\ 
\rowcolor{gray!20}             &              && 0.10 &&   0.09 & 0.09 & 0.10 && 0.10 & 0.10 & 0.11 \\ 
\rowcolor{gray!20}             & Heavy tailed && 0.05 &&   0.08 & 0.08 & 0.06 && 0.09 & 0.08 & 0.08 \\ 
\rowcolor{gray!20}             &              && 0.10 &&   0.13 & 0.14 & 0.12 && 0.17 & 0.15 & 0.13 \\ 
\rowcolor{gray!20}             & Skewed       && 0.05 &&   0.05 & 0.05 & 0.05 && 0.05 & 0.05 & 0.05 \\ 
\rowcolor{gray!20}             &              && 0.10 &&   0.09 & 0.09 & 0.10 && 0.09 & 0.09 & 0.11 \\ 
             &&&&&&&&&&&\\
Heavy tailed & Normal       && 0.05 &&   0.14 & 0.12 & 0.11 && 0.22 & 0.20 & 0.17 \\ 
             &              && 0.10 &&   0.20 & 0.20 & 0.18 && 0.30 & 0.27 & 0.23 \\ 
             & Heavy tailed && 0.05 &&   0.19 & 0.17 & 0.14 && 0.33 & 0.32 & 0.25 \\ 
             &              && 0.10 &&   0.25 & 0.23 & 0.21 && 0.45 & 0.40 & 0.34 \\ 
             & Skewed       && 0.05 &&   0.15 & 0.14 & 0.11 && 0.27 & 0.21 & 0.17 \\ 
             &              && 0.10 &&   0.21 & 0.19 & 0.19 && 0.36 & 0.32 & 0.27 \\ 
             &&&&&&&&&&&\\
Skewed       & Normal       && 0.05 &&   0.09 & 0.08 & 0.08 && 0.21 & 0.17 & 0.12 \\ 
             &              && 0.10 &&   0.16 & 0.14 & 0.14 && 0.29 & 0.24 & 0.20 \\ 
             & Heavy tailed && 0.05 &&   0.12 & 0.11 & 0.10 && 0.28 & 0.24 & 0.18 \\ 
             &              && 0.10 &&   0.20 & 0.18 & 0.17 && 0.38 & 0.33 & 0.27 \\ 
             & Skewed       && 0.05 &&   0.13 & 0.12 & 0.10 && 0.23 & 0.19 & 0.15 \\ 
             &              && 0.10 &&   0.18 & 0.17 & 0.15 && 0.32 & 0.28 & 0.24 \\ 

&&&&&&&&&&&\\
& && && \multicolumn{7}{c}{$\sigma_{\varepsilon}^2 = 1$, \ \ $\sigma_{b_0}^2 = \sigma_{b_1}^2 = 1$} \\ \cline{6-12}
\rowcolor{gray!20}Normal       & Normal       && 0.05 &&   0.04 & 0.05 & 0.04 && 0.04 & 0.05 & 0.05 \\ 
\rowcolor{gray!20}             &              && 0.10 &&   0.10 & 0.09 & 0.09 && 0.10 & 0.10 & 0.10 \\ 
\rowcolor{gray!20}             & Heavy tailed && 0.05 &&   0.06 & 0.06 & 0.05 && 0.06 & 0.05 & 0.04 \\ 
\rowcolor{gray!20}             &              && 0.10 &&   0.10 & 0.10 & 0.11 && 0.11 & 0.10 & 0.10 \\ 
\rowcolor{gray!20}             & Skewed       && 0.05 &&   0.04 & 0.03 & 0.03 && 0.05 & 0.05 & 0.04 \\ 
\rowcolor{gray!20}             &              && 0.10 &&   0.09 & 0.08 & 0.07 && 0.10 & 0.09 & 0.09 \\ 
             &&&&&&&&&&&\\
Heavy tailed & Normal       && 0.05 &&   0.19 & 0.18 & 0.15 && 0.39 & 0.36 & 0.30 \\ 
             &              && 0.10 &&   0.28 & 0.25 & 0.23 && 0.49 & 0.45 & 0.39 \\ 
             & Heavy tailed && 0.05 &&   0.24 & 0.23 & 0.18 && 0.44 & 0.40 & 0.33 \\ 
             &              && 0.10 &&   0.31 & 0.30 & 0.26 && 0.53 & 0.49 & 0.42 \\ 
             & Skewed       && 0.05 &&   0.24 & 0.21 & 0.17 && 0.41 & 0.36 & 0.28 \\ 
             &              && 0.10 &&   0.32 & 0.30 & 0.25 && 0.49 & 0.45 & 0.38 \\ 
             &&&&&&&&&&&\\
Skewed       & Normal       && 0.05 &&   0.17 & 0.17 & 0.14 && 0.36 & 0.28 & 0.22 \\ 
             &              && 0.10 &&   0.27 & 0.24 & 0.20 && 0.47 & 0.38 & 0.32 \\ 
             & Heavy tailed && 0.05 &&   0.20 & 0.17 & 0.14 && 0.41 & 0.34 & 0.25 \\ 
             &              && 0.10 &&   0.29 & 0.27 & 0.22 && 0.51 & 0.42 & 0.34 \\ 
             & Skewed       && 0.05 &&   0.18 & 0.16 & 0.12 && 0.40 & 0.30 & 0.21 \\ 
             &              && 0.10 &&   0.24 & 0.22 & 0.21 && 0.52 & 0.40 & 0.33 \\ 


&&&&&&&&&&&\\
& && && \multicolumn{7}{c}{$\sigma_{\varepsilon}^2 = 1$, \ \ $\sigma_{b_0}^2 = \sigma_{b_1}^2 = 4$} \\ \cline{6-12}
\rowcolor{gray!20}Normal       & Normal       && 0.05 &&   0.05 & 0.05 & 0.05 && 0.05 & 0.05 & 0.03 \\ 
\rowcolor{gray!20}             &              && 0.10 &&   0.10 & 0.11 & 0.11 && 0.09 & 0.10 & 0.10 \\ 
\rowcolor{gray!20}             & Heavy tailed && 0.05 &&   0.07 & 0.07 & 0.06 && 0.05 & 0.05 & 0.06 \\ 
\rowcolor{gray!20}             &              && 0.10 &&   0.13 & 0.12 & 0.11 && 0.10 & 0.10 & 0.10 \\ 
\rowcolor{gray!20}             & Skewed       && 0.05 &&   0.05 & 0.04 & 0.04 && 0.05 & 0.05 & 0.05 \\ 
\rowcolor{gray!20}             &              && 0.10 &&   0.10 & 0.10 & 0.10 && 0.09 & 0.09 & 0.10 \\ 
             &&&&&&&&&&&\\
Heavy tailed & Normal       && 0.05 &&   0.27 & 0.24 & 0.20 && 0.42 & 0.37 & 0.29 \\ 
             &              && 0.10 &&   0.33 & 0.31 & 0.28 && 0.51 & 0.47 & 0.39 \\ 
             & Heavy tailed && 0.05 &&   0.28 & 0.25 & 0.22 && 0.43 & 0.39 & 0.31 \\ 
             &              && 0.10 &&   0.37 & 0.35 & 0.28 && 0.52 & 0.48 & 0.42 \\ 
             & Skewed       && 0.05 &&   0.27 & 0.24 & 0.20 && 0.42 & 0.37 & 0.28 \\ 
             &              && 0.10 &&   0.35 & 0.32 & 0.28 && 0.51 & 0.47 & 0.39 \\ 
             &&&&&&&&&&&\\
Skewed       & Normal       && 0.05 &&   0.23 & 0.21 & 0.15 && 0.37 & 0.29 & 0.21 \\ 
             &              && 0.10 &&   0.31 & 0.29 & 0.24 && 0.46 & 0.38 & 0.31 \\ 
             & Heavy tailed && 0.05 &&   0.23 & 0.21 & 0.17 && 0.35 & 0.27 & 0.21 \\ 
             &              && 0.10 &&   0.33 & 0.32 & 0.26 && 0.44 & 0.37 & 0.30 \\ 
             & Skewed       && 0.05 &&   0.23 & 0.21 & 0.16 && 0.38 & 0.28 & 0.21 \\ 
             &              && 0.10 &&   0.32 & 0.30 & 0.26 && 0.47 & 0.38 & 0.31 \\ 

\hline
\end{tabular}
\end{scriptsize}
\end{table}


\begin{table}[ht]
\centering
\caption{\label{tab:simb050} Proportion of tests rejecting normality of the random intercept using two rotations and $s = 50$.}
\begin{scriptsize}
\begin{tabular}{ll p{.1cm} c p{.1cm} rrr p{.1cm} rrr}
  \hline
  \multicolumn{2}{c}{Distributions}& & Nominal & &  \multicolumn{3}{c}{Rotation} & & \multicolumn{3}{c}{Varimax rotation} \\ \cline{1-2} \cline{6-8} \cline{10-12}   
  Random effects & Errors & & $\alpha$ & & AD & CVM & KS & & AD & CVM & KS \\ 
   \hline
& && && \multicolumn{7}{c}{$\sigma_{\varepsilon}^2 = 4$, \ \ $\sigma_{b_0}^2 = \sigma_{b_1}^2 = 1$} \\ \cline{6-12}
\rowcolor{gray!20}Normal       & Normal       && 0.05 &&   0.05 & 0.05 & 0.05 && 0.05 & 0.06 & 0.06 \\ 
\rowcolor{gray!20}             &              && 0.10 &&   0.11 & 0.10 & 0.10 && 0.11 & 0.11 & 0.11 \\ 
\rowcolor{gray!20}             & Heavy tailed && 0.05 &&   0.07 & 0.07 & 0.07 && 0.08 & 0.08 & 0.07 \\ 
\rowcolor{gray!20}             &              && 0.10 &&   0.12 & 0.13 & 0.12 && 0.14 & 0.13 & 0.13 \\ 
\rowcolor{gray!20}             & Skewed       && 0.05 &&   0.06 & 0.06 & 0.05 && 0.04 & 0.04 & 0.04 \\ 
\rowcolor{gray!20}             &              && 0.10 &&   0.11 & 0.10 & 0.10 && 0.09 & 0.09 & 0.10 \\ 
             &&&&&&&&&&&\\
Heavy tailed & Normal       && 0.05 &&   0.13 & 0.12 & 0.10 && 0.23 & 0.20 & 0.15 \\ 
             &              && 0.10 &&   0.20 & 0.18 & 0.17 && 0.31 & 0.27 & 0.24 \\ 
             & Heavy tailed && 0.05 &&   0.17 & 0.16 & 0.13 && 0.32 & 0.28 & 0.22 \\ 
             &              && 0.10 &&   0.24 & 0.23 & 0.20 && 0.41 & 0.38 & 0.32 \\ 
             & Skewed       && 0.05 &&   0.14 & 0.13 & 0.12 && 0.26 & 0.22 & 0.18 \\ 
             &              && 0.10 &&   0.21 & 0.19 & 0.18 && 0.35 & 0.31 & 0.28 \\ 
             &&&&&&&&&&&\\
Skewed       & Normal       && 0.05 &&   0.10 & 0.08 & 0.08 && 0.21 & 0.17 & 0.12 \\ 
             &              && 0.10 &&   0.16 & 0.15 & 0.14 && 0.31 & 0.27 & 0.19 \\ 
             & Heavy tailed && 0.05 &&   0.12 & 0.11 & 0.09 && 0.27 & 0.22 & 0.17 \\ 
             &              && 0.10 &&   0.20 & 0.17 & 0.16 && 0.37 & 0.31 & 0.25 \\ 
             & Skewed       && 0.05 &&   0.12 & 0.11 & 0.09 && 0.22 & 0.18 & 0.14 \\ 
             &              && 0.10 &&   0.19 & 0.17 & 0.15 && 0.31 & 0.26 & 0.23 \\ 

&&&&&&&&&&&\\
& && && \multicolumn{7}{c}{$\sigma_{\varepsilon}^2 = 1$, \ \ $\sigma_{b_0}^2 = \sigma_{b_1}^2 = 1$} \\ \cline{6-12}
\rowcolor{gray!20}Normal       & Normal       && 0.05 &&   0.04 & 0.05 & 0.04 && 0.04 & 0.05 & 0.05 \\ 
\rowcolor{gray!20}             &              && 0.10 &&   0.09 & 0.09 & 0.08 && 0.10 & 0.10 & 0.10 \\ 
\rowcolor{gray!20}             & Heavy tailed && 0.05 &&   0.06 & 0.06 & 0.05 && 0.06 & 0.06 & 0.06 \\ 
\rowcolor{gray!20}             &              && 0.10 &&   0.10 & 0.11 & 0.11 && 0.12 & 0.12 & 0.10 \\ 
\rowcolor{gray!20}             & Skewed       && 0.05 &&   0.04 & 0.04 & 0.03 && 0.05 & 0.04 & 0.04 \\ 
\rowcolor{gray!20}             &              && 0.10 &&   0.09 & 0.08 & 0.08 && 0.08 & 0.08 & 0.09 \\ 
             &&&&&&&&&&&\\
Heavy tailed & Normal       && 0.05 &&   0.18 & 0.17 & 0.14 && 0.39 & 0.36 & 0.30 \\ 
             &              && 0.10 &&   0.27 & 0.24 & 0.23 && 0.47 & 0.43 & 0.40 \\ 
             & Heavy tailed && 0.05 &&   0.23 & 0.21 & 0.17 && 0.43 & 0.38 & 0.31 \\ 
             &              && 0.10 &&   0.31 & 0.28 & 0.25 && 0.52 & 0.48 & 0.41 \\ 
             & Skewed       && 0.05 &&   0.24 & 0.21 & 0.16 && 0.41 & 0.37 & 0.29 \\ 
             &              && 0.10 &&   0.31 & 0.29 & 0.24 && 0.50 & 0.45 & 0.38 \\ 
             &&&&&&&&&&&\\
Skewed       & Normal       && 0.05 &&   0.18 & 0.16 & 0.12 && 0.34 & 0.28 & 0.20 \\ 
             &              && 0.10 &&   0.25 & 0.23 & 0.21 && 0.46 & 0.38 & 0.31 \\ 
             & Heavy tailed && 0.05 &&   0.19 & 0.18 & 0.15 && 0.40 & 0.31 & 0.24 \\ 
             &              && 0.10 &&   0.28 & 0.26 & 0.23 && 0.49 & 0.42 & 0.35 \\ 
             & Skewed       && 0.05 &&   0.17 & 0.15 & 0.11 && 0.37 & 0.27 & 0.22 \\ 
             &              && 0.10 &&   0.24 & 0.22 & 0.19 && 0.47 & 0.38 & 0.33 \\ 


&&&&&&&&&&&\\
& && && \multicolumn{7}{c}{$\sigma_{\varepsilon}^2 = 1$, \ \ $\sigma_{b_0}^2 = \sigma_{b_1}^2 = 4$} \\ \cline{6-12}
\rowcolor{gray!20}Normal       & Normal       && 0.05 &&  0.06 & 0.06 & 0.06 && 0.04 & 0.04 & 0.04 \\ 
\rowcolor{gray!20}             &              && 0.10 &&  0.10 & 0.10 & 0.10 && 0.10 & 0.10 & 0.10 \\ 
\rowcolor{gray!20}             & Heavy tailed && 0.05 &&  0.06 & 0.06 & 0.06 && 0.05 & 0.05 & 0.04 \\ 
\rowcolor{gray!20}             &              && 0.10 &&  0.11 & 0.11 & 0.10 && 0.11 & 0.10 & 0.10 \\ 
\rowcolor{gray!20}             & Skewed       && 0.05 &&  0.05 & 0.05 & 0.06 && 0.06 & 0.06 & 0.06 \\ 
\rowcolor{gray!20}             &              && 0.10 &&  0.11 & 0.11 & 0.11 && 0.10 & 0.11 & 0.11 \\ 
             &&&&&&&&&&&\\
Heavy tailed & Normal       && 0.05 &&  0.25 & 0.23 & 0.20 && 0.38 & 0.34 & 0.26 \\ 
             &              && 0.10 &&  0.33 & 0.31 & 0.26 && 0.47 & 0.43 & 0.37 \\ 
             & Heavy tailed && 0.05 &&  0.27 & 0.25 & 0.19 && 0.42 & 0.39 & 0.31 \\ 
             &              && 0.10 &&  0.35 & 0.33 & 0.28 && 0.51 & 0.46 & 0.41 \\ 
             & Skewed       && 0.05 &&  0.25 & 0.24 & 0.19 && 0.39 & 0.34 & 0.26 \\ 
             &              && 0.10 &&  0.33 & 0.31 & 0.27 && 0.47 & 0.44 & 0.36 \\ 
             &&&&&&&&&&&\\
Skewed       & Normal       && 0.05 &&  0.23 & 0.20 & 0.15 && 0.36 & 0.28 & 0.23 \\ 
             &              && 0.10 &&  0.31 & 0.29 & 0.24 && 0.47 & 0.38 & 0.32 \\ 
             & Heavy tailed && 0.05 &&  0.22 & 0.21 & 0.17 && 0.35 & 0.27 & 0.21 \\ 
             &              && 0.10 &&  0.32 & 0.32 & 0.25 && 0.46 & 0.37 & 0.30 \\ 
             & Skewed       && 0.05 &&  0.23 & 0.21 & 0.16 && 0.36 & 0.27 & 0.21 \\ 
             &              && 0.10 &&  0.31 & 0.29 & 0.26 && 0.46 & 0.37 & 0.30 \\
\hline
\end{tabular}
\end{scriptsize}
\end{table}

\begin{table}[ht]
\centering
\caption{\label{tab:simb045} Proportion of tests rejecting normality of the random intercept using two rotations and $s = 45$.}
\begin{scriptsize}
\begin{tabular}{ll p{.1cm} c p{.1cm} rrr p{.1cm} rrr}
  \hline
  \multicolumn{2}{c}{Distributions}& & Nominal & &  \multicolumn{3}{c}{Rotation} & & \multicolumn{3}{c}{Varimax rotation} \\ \cline{1-2} \cline{6-8} \cline{10-12}   
  Random effects & Errors & & $\alpha$ & & AD & CVM & KS & & AD & CVM & KS \\ 
   \hline
& && && \multicolumn{7}{c}{$\sigma_{\varepsilon}^2 = 4$, \ \ $\sigma_{b_0}^2 = \sigma_{b_1}^2 = 1$} \\ \cline{6-12}
\rowcolor{gray!20}Normal       & Normal       && 0.05 &&   0.04 & 0.04 & 0.04 && 0.05 & 0.06 & 0.05 \\ 
\rowcolor{gray!20}             &              && 0.10 &&   0.10 & 0.10 & 0.10 && 0.11 & 0.10 & 0.10 \\ 
\rowcolor{gray!20}             & Heavy tailed && 0.05 &&   0.06 & 0.06 & 0.06 && 0.07 & 0.07 & 0.06 \\ 
\rowcolor{gray!20}             &              && 0.10 &&   0.12 & 0.12 & 0.12 && 0.13 & 0.12 & 0.10 \\ 
\rowcolor{gray!20}             & Skewed       && 0.05 &&   0.06 & 0.06 & 0.05 && 0.06 & 0.05 & 0.04 \\ 
\rowcolor{gray!20}             &              && 0.10 &&   0.10 & 0.10 & 0.09 && 0.11 & 0.10 & 0.09 \\ 
             &&&&&&&&&&&\\
Heavy tailed & Normal       && 0.05 &&   0.13 & 0.12 & 0.10 && 0.23 & 0.21 & 0.15 \\ 
             &              && 0.10 &&   0.20 & 0.18 & 0.18 && 0.30 & 0.27 & 0.24 \\ 
             & Heavy tailed && 0.05 &&   0.16 & 0.15 & 0.13 && 0.32 & 0.27 & 0.22 \\ 
             &              && 0.10 &&   0.23 & 0.22 & 0.19 && 0.40 & 0.37 & 0.32 \\ 
             & Skewed       && 0.05 &&   0.14 & 0.13 & 0.11 && 0.27 & 0.24 & 0.18 \\ 
             &              && 0.10 &&   0.21 & 0.19 & 0.17 && 0.34 & 0.31 & 0.28 \\ 
             &&&&&&&&&&&\\
Skewed       & Normal       && 0.05 &&   0.10 & 0.10 & 0.08 && 0.22 & 0.20 & 0.14 \\ 
             &              && 0.10 &&   0.18 & 0.16 & 0.14 && 0.31 & 0.26 & 0.22 \\ 
             & Heavy tailed && 0.05 &&   0.11 & 0.10 & 0.10 && 0.25 & 0.20 & 0.16 \\ 
             &              && 0.10 &&   0.18 & 0.17 & 0.15 && 0.35 & 0.30 & 0.25 \\ 
             & Skewed       && 0.05 &&   0.12 & 0.11 & 0.10 && 0.23 & 0.20 & 0.14 \\ 
             &              && 0.10 &&   0.19 & 0.18 & 0.16 && 0.32 & 0.27 & 0.20 \\ 

&&&&&&&&&&&\\
& && && \multicolumn{7}{c}{$\sigma_{\varepsilon}^2 = 1$, \ \ $\sigma_{b_0}^2 = \sigma_{b_1}^2 = 1$} \\ \cline{6-12}
\rowcolor{gray!20}Normal       & Normal       && 0.05 &&   0.04 & 0.04 & 0.04 && 0.04 & 0.05 & 0.05 \\ 
\rowcolor{gray!20}             &              && 0.10 &&   0.09 & 0.09 & 0.09 && 0.10 & 0.10 & 0.11 \\ 
\rowcolor{gray!20}             & Heavy tailed && 0.05 &&   0.06 & 0.06 & 0.05 && 0.05 & 0.06 & 0.05 \\ 
\rowcolor{gray!20}             &              && 0.10 &&   0.10 & 0.10 & 0.10 && 0.12 & 0.11 & 0.10 \\ 
\rowcolor{gray!20}             & Skewed       && 0.05 &&   0.04 & 0.04 & 0.04 && 0.06 & 0.06 & 0.05 \\ 
\rowcolor{gray!20}             &              && 0.10 &&   0.09 & 0.08 & 0.08 && 0.10 & 0.10 & 0.11 \\ 
             &&&&&&&&&&&\\
Heavy tailed & Normal       && 0.05 &&   0.18 & 0.17 & 0.14 && 0.37 & 0.33 & 0.28 \\ 
             &              && 0.10 &&   0.26 & 0.24 & 0.22 && 0.45 & 0.41 & 0.37 \\ 
             & Heavy tailed && 0.05 &&   0.21 & 0.20 & 0.14 && 0.40 & 0.36 & 0.27 \\ 
             &              && 0.10 &&   0.28 & 0.26 & 0.23 && 0.49 & 0.45 & 0.38 \\ 
             & Skewed       && 0.05 &&   0.22 & 0.19 & 0.16 && 0.39 & 0.34 & 0.27 \\ 
             &              && 0.10 &&   0.28 & 0.26 & 0.23 && 0.47 & 0.43 & 0.37 \\ 
             &&&&&&&&&&&\\
Skewed       & Normal       && 0.05 &&   0.17 & 0.15 & 0.12 && 0.36 & 0.28 & 0.23 \\ 
             &              && 0.10 &&   0.25 & 0.24 & 0.19 && 0.45 & 0.38 & 0.31 \\ 
             & Heavy tailed && 0.05 &&   0.19 & 0.17 & 0.14 && 0.38 & 0.32 & 0.24 \\ 
             &              && 0.10 &&   0.27 & 0.24 & 0.22 && 0.48 & 0.41 & 0.34 \\ 
             & Skewed       && 0.05 &&   0.15 & 0.13 & 0.10 && 0.34 & 0.26 & 0.20 \\ 
             &              && 0.10 &&   0.22 & 0.20 & 0.16 && 0.47 & 0.38 & 0.30 \\ 


&&&&&&&&&&&\\
& && && \multicolumn{7}{c}{$\sigma_{\varepsilon}^2 = 1$, \ \ $\sigma_{b_0}^2 = \sigma_{b_1}^2 = 4$} \\ \cline{6-12}
\rowcolor{gray!20}Normal       & Normal       && 0.05 &&   0.05 & 0.05 & 0.05 && 0.05 & 0.06 & 0.06 \\ 
\rowcolor{gray!20}             &              && 0.10 &&   0.11 & 0.11 & 0.09 && 0.11 & 0.10 & 0.11 \\ 
\rowcolor{gray!20}             & Heavy tailed && 0.05 &&   0.06 & 0.06 & 0.06 && 0.05 & 0.05 & 0.04 \\ 
\rowcolor{gray!20}             &              && 0.10 &&   0.12 & 0.12 & 0.11 && 0.09 & 0.09 & 0.09 \\ 
\rowcolor{gray!20}             & Skewed       && 0.05 &&   0.05 & 0.05 & 0.05 && 0.06 & 0.06 & 0.05 \\ 
\rowcolor{gray!20}             &              && 0.10 &&   0.10 & 0.10 & 0.11 && 0.11 & 0.11 & 0.11 \\ 
             &&&&&&&&&&&\\
Heavy tailed & Normal       && 0.05 &&   0.22 & 0.20 & 0.18 && 0.36 & 0.32 & 0.25 \\ 
             &              && 0.10 &&   0.30 & 0.28 & 0.24 && 0.48 & 0.43 & 0.35 \\ 
             & Heavy tailed && 0.05 &&   0.24 & 0.23 & 0.17 && 0.37 & 0.35 & 0.29 \\ 
             &              && 0.10 &&   0.32 & 0.31 & 0.26 && 0.46 & 0.43 & 0.37 \\ 
             & Skewed       && 0.05 &&   0.24 & 0.23 & 0.17 && 0.37 & 0.34 & 0.26 \\ 
             &              && 0.10 &&   0.33 & 0.30 & 0.26 && 0.46 & 0.44 & 0.35 \\ 
             &&&&&&&&&&&\\
Skewed       & Normal       && 0.05 &&   0.21 & 0.20 & 0.14 && 0.34 & 0.27 & 0.21 \\ 
             &              && 0.10 &&   0.30 & 0.30 & 0.23 && 0.44 & 0.36 & 0.29 \\ 
             & Heavy tailed && 0.05 &&   0.22 & 0.22 & 0.16 && 0.33 & 0.25 & 0.20 \\ 
             &              && 0.10 &&   0.31 & 0.30 & 0.26 && 0.43 & 0.36 & 0.28 \\ 
             & Skewed       && 0.05 &&   0.21 & 0.21 & 0.17 && 0.33 & 0.24 & 0.19 \\ 
             &              && 0.10 &&   0.29 & 0.27 & 0.25 && 0.43 & 0.34 & 0.28 \\ 

\hline
\end{tabular}
\end{scriptsize}
\end{table}

\begin{table}[ht]
\centering
\caption{\label{tab:simb040} Proportion of tests rejecting normality of the random intercept using two rotations and $s = 40$.}
\begin{scriptsize}
\begin{tabular}{ll p{.1cm} c p{.1cm} rrr p{.1cm} rrr}
  \hline
  \multicolumn{2}{c}{Distributions}& & Nominal & &  \multicolumn{3}{c}{Rotation} & & \multicolumn{3}{c}{Varimax rotation} \\ \cline{1-2} \cline{6-8} \cline{10-12}   
  Random effects & Errors & & $\alpha$ & & AD & CVM & KS & & AD & CVM & KS \\ 
   \hline
& && && \multicolumn{7}{c}{$\sigma_{\varepsilon}^2 = 4$, \ \ $\sigma_{b_0}^2 = \sigma_{b_1}^2 = 1$} \\ \cline{6-12}
\rowcolor{gray!20}Normal       & Normal       && 0.05 &&   0.04 & 0.05 & 0.04 && 0.05 & 0.05 & 0.06 \\ 
\rowcolor{gray!20}             &              && 0.10 &&   0.11 & 0.10 & 0.09 && 0.12 & 0.12 & 0.11 \\ 
\rowcolor{gray!20}             & Heavy tailed && 0.05 &&   0.05 & 0.05 & 0.06 && 0.06 & 0.06 & 0.04 \\ 
\rowcolor{gray!20}             &              && 0.10 &&   0.11 & 0.11 & 0.12 && 0.12 & 0.12 & 0.10 \\ 
\rowcolor{gray!20}             & Skewed       && 0.05 &&   0.06 & 0.06 & 0.05 && 0.06 & 0.06 & 0.06 \\ 
\rowcolor{gray!20}             &              && 0.10 &&   0.11 & 0.11 & 0.10 && 0.12 & 0.11 & 0.11 \\ 
             &&&&&&&&&&&\\
Heavy tailed & Normal       && 0.05 &&   0.13 & 0.12 & 0.10 && 0.23 & 0.20 & 0.17 \\ 
             &              && 0.10 &&   0.21 & 0.19 & 0.18 && 0.31 & 0.29 & 0.25 \\ 
             & Heavy tailed && 0.05 &&   0.16 & 0.15 & 0.11 && 0.30 & 0.27 & 0.22 \\ 
             &              && 0.10 &&   0.23 & 0.21 & 0.20 && 0.38 & 0.35 & 0.30 \\ 
             & Skewed       && 0.05 &&   0.13 & 0.12 & 0.10 && 0.24 & 0.22 & 0.17 \\ 
             &              && 0.10 &&   0.20 & 0.17 & 0.16 && 0.31 & 0.29 & 0.26 \\ 
             &&&&&&&&&&&\\
Skewed       & Normal       && 0.05 &&   0.10 & 0.08 & 0.08 && 0.23 & 0.18 & 0.15 \\ 
             &              && 0.10 &&   0.18 & 0.17 & 0.14 && 0.32 & 0.27 & 0.23 \\ 
             & Heavy tailed && 0.05 &&   0.10 & 0.09 & 0.08 && 0.26 & 0.21 & 0.16 \\ 
             &              && 0.10 &&   0.18 & 0.16 & 0.14 && 0.35 & 0.29 & 0.25 \\ 
             & Skewed       && 0.05 &&   0.11 & 0.11 & 0.10 && 0.21 & 0.17 & 0.11 \\ 
             &              && 0.10 &&   0.18 & 0.17 & 0.16 && 0.31 & 0.25 & 0.19 \\ 

&&&&&&&&&&&\\
& && && \multicolumn{7}{c}{$\sigma_{\varepsilon}^2 = 1$, \ \ $\sigma_{b_0}^2 = \sigma_{b_1}^2 = 1$} \\ \cline{6-12}
\rowcolor{gray!20}Normal       & Normal       && 0.05 &&   0.04 & 0.05 & 0.04 && 0.07 & 0.06 & 0.05 \\ 
\rowcolor{gray!20}             &              && 0.10 &&   0.09 & 0.11 & 0.09 && 0.11 & 0.11 & 0.10 \\ 
\rowcolor{gray!20}             & Heavy tailed && 0.05 &&   0.05 & 0.05 & 0.05 && 0.05 & 0.05 & 0.05 \\ 
\rowcolor{gray!20}             &              && 0.10 &&   0.10 & 0.10 & 0.10 && 0.12 & 0.10 & 0.10 \\ 
\rowcolor{gray!20}             & Skewed       && 0.05 &&   0.05 & 0.05 & 0.03 && 0.05 & 0.04 & 0.05 \\ 
\rowcolor{gray!20}             &              && 0.10 &&   0.09 & 0.09 & 0.09 && 0.10 & 0.10 & 0.10 \\ 
             &&&&&&&&&&&\\
Heavy tailed & Normal       && 0.05 &&   0.17 & 0.16 & 0.14 && 0.36 & 0.33 & 0.26 \\ 
             &              && 0.10 &&   0.24 & 0.22 & 0.21 && 0.45 & 0.41 & 0.35 \\ 
             & Heavy tailed && 0.05 &&   0.19 & 0.17 & 0.14 && 0.38 & 0.34 & 0.28 \\ 
             &              && 0.10 &&   0.26 & 0.24 & 0.20 && 0.46 & 0.42 & 0.37 \\ 
             & Skewed       && 0.05 &&   0.21 & 0.19 & 0.13 && 0.36 & 0.33 & 0.26 \\ 
             &              && 0.10 &&   0.27 & 0.25 & 0.21 && 0.46 & 0.41 & 0.35 \\ 
             &&&&&&&&&&&\\
Skewed       & Normal       && 0.05 &&   0.16 & 0.15 & 0.10 && 0.33 & 0.27 & 0.21 \\ 
             &              && 0.10 &&   0.25 & 0.23 & 0.18 && 0.44 & 0.36 & 0.32 \\ 
             & Heavy tailed && 0.05 &&   0.17 & 0.17 & 0.13 && 0.35 & 0.29 & 0.22 \\ 
             &              && 0.10 &&   0.26 & 0.24 & 0.21 && 0.46 & 0.39 & 0.33 \\ 
             & Skewed       && 0.05 &&   0.15 & 0.14 & 0.10 && 0.35 & 0.28 & 0.20 \\ 
             &              && 0.10 &&   0.22 & 0.20 & 0.17 && 0.43 & 0.37 & 0.30 \\ 


&&&&&&&&&&&\\
& && && \multicolumn{7}{c}{$\sigma_{\varepsilon}^2 = 1$, \ \ $\sigma_{b_0}^2 = \sigma_{b_1}^2 = 4$} \\ \cline{6-12}
\rowcolor{gray!20}Normal       & Normal       && 0.05 &&  0.05 & 0.05 & 0.05 && 0.06 & 0.06 & 0.06 \\ 
\rowcolor{gray!20}             &              && 0.10 &&  0.10 & 0.10 & 0.10 && 0.12 & 0.12 & 0.10 \\ 
\rowcolor{gray!20}             & Heavy tailed && 0.05 &&  0.06 & 0.05 & 0.05 && 0.05 & 0.05 & 0.05 \\ 
\rowcolor{gray!20}             &              && 0.10 &&  0.11 & 0.11 & 0.11 && 0.11 & 0.11 & 0.10 \\ 
\rowcolor{gray!20}             & Skewed       && 0.05 &&  0.05 & 0.05 & 0.06 && 0.05 & 0.06 & 0.05 \\ 
\rowcolor{gray!20}             &              && 0.10 &&  0.11 & 0.11 & 0.11 && 0.10 & 0.10 & 0.10 \\ 
             &&&&&&&&&&&\\
Heavy tailed & Normal       && 0.05 &&  0.20 & 0.18 & 0.16 && 0.33 & 0.29 & 0.24 \\ 
             &              && 0.10 &&  0.28 & 0.26 & 0.23 && 0.42 & 0.39 & 0.33 \\ 
             & Heavy tailed && 0.05 &&  0.23 & 0.20 & 0.16 && 0.36 & 0.34 & 0.26 \\ 
             &              && 0.10 &&  0.31 & 0.27 & 0.24 && 0.45 & 0.41 & 0.36 \\ 
             & Skewed       && 0.05 &&  0.21 & 0.20 & 0.17 && 0.35 & 0.31 & 0.22 \\ 
             &              && 0.10 &&  0.29 & 0.27 & 0.25 && 0.43 & 0.41 & 0.33 \\ 
             &&&&&&&&&&&\\
Skewed       & Normal       && 0.05 &&  0.20 & 0.19 & 0.12 && 0.32 & 0.25 & 0.20 \\ 
             &              && 0.10 &&  0.27 & 0.27 & 0.21 && 0.44 & 0.35 & 0.29 \\ 
             & Heavy tailed && 0.05 &&  0.19 & 0.19 & 0.15 && 0.30 & 0.25 & 0.19 \\ 
             &              && 0.10 &&  0.29 & 0.28 & 0.23 && 0.42 & 0.35 & 0.29 \\ 
             & Skewed       && 0.05 &&  0.21 & 0.19 & 0.17 && 0.29 & 0.21 & 0.19 \\ 
             &              && 0.10 &&  0.29 & 0.27 & 0.25 && 0.38 & 0.31 & 0.27 \\ 


\hline
\end{tabular}
\end{scriptsize}
\end{table}

\begin{table}[ht]
\centering
\caption{\label{tab:simb035}Proportion of tests rejecting normality of the random intercept using two rotations and $s = 35$.}
\begin{scriptsize}
\begin{tabular}{ll p{.1cm} c p{.1cm} rrr p{.1cm} rrr}
  \hline
  \multicolumn{2}{c}{Distributions}& & Nominal & &  \multicolumn{3}{c}{Rotation} & & \multicolumn{3}{c}{Varimax rotation} \\ \cline{1-2} \cline{6-8} \cline{10-12}   
  Random effects & Errors & & $\alpha$ & & AD & CVM & KS & & AD & CVM & KS \\ 
   \hline
& && && \multicolumn{7}{c}{$\sigma_{\varepsilon}^2 = 4$, \ \ $\sigma_{b_0}^2 = \sigma_{b_1}^2 = 1$} \\ \cline{6-12}
\rowcolor{gray!20}Normal       & Normal       && 0.05 &&  0.05 & 0.05 & 0.05 && 0.05 & 0.05 & 0.06 \\ 
\rowcolor{gray!20}             &              && 0.10 &&  0.11 & 0.11 & 0.10 && 0.11 & 0.11 & 0.11 \\ 
\rowcolor{gray!20}             & Heavy tailed && 0.05 &&  0.04 & 0.04 & 0.06 && 0.05 & 0.06 & 0.05 \\ 
\rowcolor{gray!20}             &              && 0.10 &&  0.10 & 0.10 & 0.10 && 0.11 & 0.11 & 0.10 \\ 
\rowcolor{gray!20}             & Skewed       && 0.05 &&  0.06 & 0.06 & 0.06 && 0.05 & 0.06 & 0.06 \\ 
\rowcolor{gray!20}             &              && 0.10 &&  0.11 & 0.11 & 0.11 && 0.12 & 0.12 & 0.11 \\ 
             &&&&&&&&&&&\\
Heavy tailed & Normal       && 0.05 &&  0.13 & 0.11 & 0.10 && 0.22 & 0.20 & 0.17 \\ 
             &              && 0.10 &&  0.21 & 0.19 & 0.17 && 0.30 & 0.27 & 0.23 \\ 
             & Heavy tailed && 0.05 &&  0.15 & 0.14 & 0.12 && 0.27 & 0.24 & 0.19 \\ 
             &              && 0.10 &&  0.21 & 0.20 & 0.18 && 0.35 & 0.31 & 0.28 \\ 
             & Skewed       && 0.05 &&  0.13 & 0.12 & 0.10 && 0.23 & 0.21 & 0.17 \\ 
             &              && 0.10 &&  0.19 & 0.16 & 0.16 && 0.30 & 0.28 & 0.26 \\ 
             &&&&&&&&&&&\\
Skewed       & Normal       && 0.05 &&  0.11 & 0.10 & 0.07 && 0.21 & 0.17 & 0.13 \\ 
             &              && 0.10 &&  0.17 & 0.16 & 0.15 && 0.30 & 0.26 & 0.22 \\ 
             & Heavy tailed && 0.05 &&  0.11 & 0.10 & 0.09 && 0.25 & 0.21 & 0.16 \\ 
             &              && 0.10 &&  0.18 & 0.18 & 0.15 && 0.33 & 0.29 & 0.24 \\ 
             & Skewed       && 0.05 &&  0.11 & 0.11 & 0.09 && 0.23 & 0.19 & 0.14 \\ 
             &              && 0.10 &&  0.18 & 0.17 & 0.17 && 0.31 & 0.26 & 0.21 \\ 

&&&&&&&&&&&\\
& && && \multicolumn{7}{c}{$\sigma_{\varepsilon}^2 = 1$, \ \ $\sigma_{b_0}^2 = \sigma_{b_1}^2 = 1$} \\ \cline{6-12}
\rowcolor{gray!20}Normal       & Normal       && 0.05 &&  0.04 & 0.05 & 0.04 && 0.06 & 0.06 & 0.05 \\ 
\rowcolor{gray!20}             &              && 0.10 &&  0.10 & 0.10 & 0.09 && 0.11 & 0.11 & 0.10 \\ 
\rowcolor{gray!20}             & Heavy tailed && 0.05 &&  0.05 & 0.05 & 0.04 && 0.04 & 0.04 & 0.05 \\ 
\rowcolor{gray!20}             &              && 0.10 &&  0.09 & 0.10 & 0.08 && 0.09 & 0.09 & 0.09 \\ 
\rowcolor{gray!20}             & Skewed       && 0.05 &&  0.04 & 0.04 & 0.04 && 0.05 & 0.05 & 0.05 \\ 
\rowcolor{gray!20}             &              && 0.10 &&  0.09 & 0.09 & 0.09 && 0.09 & 0.09 & 0.10 \\ 
             &&&&&&&&&&&\\
Heavy tailed & Normal       && 0.05 &&  0.15 & 0.15 & 0.13 && 0.33 & 0.30 & 0.26 \\ 
             &              && 0.10 &&  0.23 & 0.21 & 0.19 && 0.41 & 0.38 & 0.34 \\ 
             & Heavy tailed && 0.05 &&  0.19 & 0.17 & 0.13 && 0.34 & 0.31 & 0.25 \\ 
             &              && 0.10 &&  0.25 & 0.23 & 0.19 && 0.43 & 0.39 & 0.33 \\ 
             & Skewed       && 0.05 &&  0.19 & 0.17 & 0.14 && 0.34 & 0.30 & 0.24 \\ 
             &              && 0.10 &&  0.27 & 0.25 & 0.20 && 0.42 & 0.39 & 0.34 \\ 
             &&&&&&&&&&&\\
Skewed       & Normal       && 0.05 &&  0.15 & 0.13 & 0.10 && 0.33 & 0.28 & 0.21 \\ 
             &              && 0.10 &&  0.23 & 0.20 & 0.18 && 0.42 & 0.36 & 0.31 \\ 
             & Heavy tailed && 0.05 &&  0.15 & 0.14 & 0.12 && 0.35 & 0.29 & 0.23 \\ 
             &              && 0.10 &&  0.23 & 0.21 & 0.19 && 0.45 & 0.39 & 0.32 \\ 
             & Skewed       && 0.05 &&  0.14 & 0.12 & 0.10 && 0.31 & 0.26 & 0.19 \\ 
             &              && 0.10 &&  0.20 & 0.19 & 0.15 && 0.41 & 0.34 & 0.29 \\ 


&&&&&&&&&&&\\
& && && \multicolumn{7}{c}{$\sigma_{\varepsilon}^2 = 1$, \ \ $\sigma_{b_0}^2 = \sigma_{b_1}^2 = 4$} \\ \cline{6-12}
\rowcolor{gray!20}Normal       & Normal       && 0.05 &&  0.06 & 0.05 & 0.05 && 0.04 & 0.05 & 0.06 \\ 
\rowcolor{gray!20}             &              && 0.10 &&  0.11 & 0.11 & 0.10 && 0.10 & 0.10 & 0.10 \\ 
\rowcolor{gray!20}             & Heavy tailed && 0.05 &&  0.06 & 0.06 & 0.06 && 0.05 & 0.05 & 0.04 \\ 
\rowcolor{gray!20}             &              && 0.10 &&  0.11 & 0.11 & 0.12 && 0.10 & 0.11 & 0.11 \\ 
\rowcolor{gray!20}             & Skewed       && 0.05 &&  0.04 & 0.04 & 0.05 && 0.04 & 0.05 & 0.05 \\ 
\rowcolor{gray!20}             &              && 0.10 &&  0.10 & 0.11 & 0.10 && 0.10 & 0.09 & 0.10 \\ 
             &&&&&&&&&&&\\
Heavy tailed & Normal       && 0.05 &&  0.17 & 0.16 & 0.13 && 0.32 & 0.28 & 0.21 \\ 
             &              && 0.10 &&  0.26 & 0.24 & 0.22 && 0.41 & 0.38 & 0.33 \\ 
             & Heavy tailed && 0.05 &&  0.21 & 0.19 & 0.14 && 0.35 & 0.31 & 0.24 \\ 
             &              && 0.10 &&  0.29 & 0.27 & 0.22 && 0.43 & 0.41 & 0.36 \\ 
             & Skewed       && 0.05 &&  0.20 & 0.19 & 0.15 && 0.32 & 0.29 & 0.23 \\ 
             &              && 0.10 &&  0.29 & 0.26 & 0.23 && 0.40 & 0.37 & 0.32 \\ 
             &&&&&&&&&&&\\
Skewed       & Normal       && 0.05 &&  0.17 & 0.16 & 0.12 && 0.29 & 0.22 & 0.18 \\ 
             &              && 0.10 &&  0.26 & 0.24 & 0.21 && 0.40 & 0.33 & 0.28 \\ 
             & Heavy tailed && 0.05 &&  0.20 & 0.20 & 0.15 && 0.31 & 0.25 & 0.18 \\ 
             &              && 0.10 &&  0.27 & 0.26 & 0.22 && 0.40 & 0.34 & 0.30 \\ 
             & Skewed       && 0.05 &&  0.18 & 0.16 & 0.14 && 0.29 & 0.22 & 0.17 \\ 
             &              && 0.10 &&  0.26 & 0.25 & 0.22 && 0.39 & 0.32 & 0.27 \\ 

\hline
\end{tabular}
\end{scriptsize}
\end{table}

\begin{table}[ht]
\centering
\caption{\label{tab:simb030}Proportion of tests rejecting normality of the random intercept using two rotations and $s = 30$.}
\begin{scriptsize}
\begin{tabular}{ll p{.1cm} c p{.1cm} rrr p{.1cm} rrr}
  \hline
  \multicolumn{2}{c}{Distributions}& & Nominal & &  \multicolumn{3}{c}{Rotation} & & \multicolumn{3}{c}{Varimax rotation} \\ \cline{1-2} \cline{6-8} \cline{10-12}   
  Random effects & Errors & & $\alpha$ & & AD & CVM & KS & & AD & CVM & KS \\ 
   \hline
& && && \multicolumn{7}{c}{$\sigma_{\varepsilon}^2 = 4$, \ \ $\sigma_{b_0}^2 = \sigma_{b_1}^2 = 1$} \\ \cline{6-12}
\rowcolor{gray!20}Normal       & Normal       && 0.05 &&  0.06 & 0.06 & 0.05 && 0.06 & 0.05 & 0.05 \\ 
\rowcolor{gray!20}             &              && 0.10 &&  0.11 & 0.10 & 0.12 && 0.11 & 0.10 & 0.10 \\ 
\rowcolor{gray!20}             & Heavy tailed && 0.05 &&  0.05 & 0.04 & 0.05 && 0.05 & 0.05 & 0.05 \\ 
\rowcolor{gray!20}             &              && 0.10 &&  0.10 & 0.09 & 0.09 && 0.11 & 0.10 & 0.10 \\ 
\rowcolor{gray!20}             & Skewed       && 0.05 &&  0.06 & 0.06 & 0.06 && 0.06 & 0.06 & 0.06 \\ 
\rowcolor{gray!20}             &              && 0.10 &&  0.12 & 0.11 & 0.11 && 0.11 & 0.11 & 0.11 \\ 
             &&&&&&&&&&&\\
Heavy tailed & Normal       && 0.05 &&  0.12 & 0.11 & 0.09 && 0.22 & 0.20 & 0.15 \\ 
             &              && 0.10 &&  0.18 & 0.17 & 0.15 && 0.29 & 0.28 & 0.22 \\ 
             & Heavy tailed && 0.05 &&  0.14 & 0.13 & 0.11 && 0.27 & 0.24 & 0.19 \\ 
             &              && 0.10 &&  0.21 & 0.19 & 0.17 && 0.35 & 0.31 & 0.28 \\ 
             & Skewed       && 0.05 &&  0.12 & 0.11 & 0.09 && 0.22 & 0.20 & 0.16 \\ 
             &              && 0.10 &&  0.19 & 0.16 & 0.15 && 0.29 & 0.27 & 0.24 \\ 
             &&&&&&&&&&&\\
Skewed       & Normal       && 0.05 &&  0.10 & 0.09 & 0.08 && 0.22 & 0.18 & 0.12 \\ 
             &              && 0.10 &&  0.17 & 0.15 & 0.14 && 0.30 & 0.27 & 0.21 \\ 
             & Heavy tailed && 0.05 &&  0.11 & 0.09 & 0.09 && 0.24 & 0.21 & 0.16 \\ 
             &              && 0.10 &&  0.17 & 0.17 & 0.16 && 0.32 & 0.29 & 0.24 \\ 
             & Skewed       && 0.05 &&  0.11 & 0.10 & 0.09 && 0.21 & 0.18 & 0.12 \\ 
             &              && 0.10 &&  0.18 & 0.17 & 0.16 && 0.31 & 0.26 & 0.19 \\ 

&&&&&&&&&&&\\
& && && \multicolumn{7}{c}{$\sigma_{\varepsilon}^2 = 1$, \ \ $\sigma_{b_0}^2 = \sigma_{b_1}^2 = 1$} \\ \cline{6-12}
\rowcolor{gray!20}Normal       & Normal       && 0.05 &&  0.04 & 0.04 & 0.05 && 0.05 & 0.05 & 0.05 \\ 
\rowcolor{gray!20}             &              && 0.10 &&  0.10 & 0.10 & 0.09 && 0.10 & 0.11 & 0.10 \\ 
\rowcolor{gray!20}             & Heavy tailed && 0.05 &&  0.05 & 0.05 & 0.04 && 0.06 & 0.06 & 0.06 \\ 
\rowcolor{gray!20}             &              && 0.10 &&  0.09 & 0.10 & 0.08 && 0.12 & 0.12 & 0.11 \\ 
\rowcolor{gray!20}             & Skewed       && 0.05 &&  0.04 & 0.04 & 0.04 && 0.04 & 0.04 & 0.04 \\ 
\rowcolor{gray!20}             &              && 0.10 &&  0.09 & 0.09 & 0.10 && 0.08 & 0.10 & 0.08 \\ 
             &&&&&&&&&&&\\
Heavy tailed & Normal       && 0.05 &&  0.14 & 0.14 & 0.12 && 0.29 & 0.28 & 0.22 \\ 
             &              && 0.10 &&  0.21 & 0.20 & 0.18 && 0.39 & 0.35 & 0.32 \\ 
             & Heavy tailed && 0.05 &&  0.17 & 0.15 & 0.13 && 0.34 & 0.30 & 0.24 \\ 
             &              && 0.10 &&  0.24 & 0.22 & 0.19 && 0.41 & 0.39 & 0.33 \\ 
             & Skewed       && 0.05 &&  0.17 & 0.15 & 0.12 && 0.31 & 0.29 & 0.23 \\ 
             &              && 0.10 &&  0.25 & 0.23 & 0.20 && 0.39 & 0.36 & 0.31 \\ 
             &&&&&&&&&&&\\
Skewed       & Normal       && 0.05 &&  0.14 & 0.12 & 0.09 && 0.30 & 0.25 & 0.20 \\ 
             &              && 0.10 &&  0.22 & 0.21 & 0.17 && 0.39 & 0.33 & 0.28 \\ 
             & Heavy tailed && 0.05 &&  0.14 & 0.13 & 0.12 && 0.32 & 0.26 & 0.21 \\ 
             &              && 0.10 &&  0.22 & 0.21 & 0.18 && 0.44 & 0.36 & 0.32 \\ 
             & Skewed       && 0.05 &&  0.13 & 0.11 & 0.09 && 0.30 & 0.23 & 0.18 \\ 
             &              && 0.10 &&  0.19 & 0.18 & 0.14 && 0.40 & 0.34 & 0.28 \\ 


&&&&&&&&&&&\\
& && && \multicolumn{7}{c}{$\sigma_{\varepsilon}^2 = 1$, \ \ $\sigma_{b_0}^2 = \sigma_{b_1}^2 = 4$} \\ \cline{6-12}
\rowcolor{gray!20}Normal       & Normal       && 0.05 &&  0.04 & 0.04 & 0.04 && 0.05 & 0.05 & 0.05 \\ 
\rowcolor{gray!20}             &              && 0.10 &&  0.11 & 0.10 & 0.09 && 0.10 & 0.10 & 0.10 \\ 
\rowcolor{gray!20}             & Heavy tailed && 0.05 &&  0.05 & 0.05 & 0.06 && 0.05 & 0.05 & 0.05 \\ 
\rowcolor{gray!20}             &              && 0.10 &&  0.10 & 0.11 & 0.11 && 0.10 & 0.11 & 0.11 \\ 
\rowcolor{gray!20}             & Skewed       && 0.05 &&  0.04 & 0.05 & 0.04 && 0.05 & 0.05 & 0.04 \\ 
\rowcolor{gray!20}             &              && 0.10 &&  0.10 & 0.09 & 0.10 && 0.09 & 0.09 & 0.10 \\
             &&&&&&&&&&&\\ 
Heavy tailed & Normal       && 0.05 &&  0.17 & 0.17 & 0.14 && 0.28 & 0.26 & 0.19 \\ 
             &              && 0.10 &&  0.24 & 0.23 & 0.21 && 0.37 & 0.33 & 0.30 \\ 
             & Heavy tailed && 0.05 &&  0.19 & 0.17 & 0.13 && 0.31 & 0.28 & 0.23 \\ 
             &              && 0.10 &&  0.28 & 0.25 & 0.20 && 0.40 & 0.38 & 0.34 \\ 
             & Skewed       && 0.05 &&  0.17 & 0.16 & 0.14 && 0.29 & 0.26 & 0.21 \\ 
             &              && 0.10 &&  0.24 & 0.22 & 0.19 && 0.38 & 0.34 & 0.30 \\ 
             &&&&&&&&&&&\\
Skewed       & Normal       && 0.05 &&  0.16 & 0.16 & 0.13 && 0.28 & 0.21 & 0.15 \\ 
             &              && 0.10 &&  0.26 & 0.24 & 0.19 && 0.39 & 0.32 & 0.27 \\ 
             & Heavy tailed && 0.05 &&  0.19 & 0.18 & 0.13 && 0.29 & 0.23 & 0.17 \\ 
             &              && 0.10 &&  0.26 & 0.25 & 0.22 && 0.39 & 0.33 & 0.27 \\ 
             & Skewed       && 0.05 &&  0.16 & 0.14 & 0.12 && 0.26 & 0.21 & 0.16 \\ 
             &              && 0.10 &&  0.25 & 0.23 & 0.21 && 0.36 & 0.30 & 0.26 \\ 

\hline
\end{tabular}
\end{scriptsize}
\end{table}

\begin{table}[ht]
\centering
\caption{\label{tab:simb1sB}Proportion of tests rejecting normality of the random slope using two rotations and $s = \rank(\bm{B})$.}
\begin{scriptsize}
\begin{tabular}{ll p{.1cm} c p{.1cm} rrr p{.1cm} rrr}
  \hline
  \multicolumn{2}{c}{Distributions}& & Nominal & &  \multicolumn{3}{c}{Rotation} & & \multicolumn{3}{c}{Varimax rotation} \\ \cline{1-2} \cline{6-8} \cline{10-12}   
  Random effects & Errors & & $\alpha$ & & AD & CVM & KS & & AD & CVM & KS \\ 
   \hline
& && && \multicolumn{7}{c}{$\sigma_{\varepsilon}^2 = 4$, \ \ $\sigma_{b_0}^2 = \sigma_{b_1}^2 = 1$} \\ \cline{6-12}
\rowcolor{gray!20}Normal       & Normal       && 0.05 &&  0.03 & 0.03 & 0.04 && 0.05 & 0.05 & 0.05 \\ 
\rowcolor{gray!20}             &              && 0.10 &&  0.08 & 0.08 & 0.10 && 0.09 & 0.10 & 0.10 \\ 
\rowcolor{gray!20}             & Heavy tailed && 0.05 &&  0.13 & 0.12 & 0.11 && 0.22 & 0.20 & 0.16 \\ 
\rowcolor{gray!20}             &              && 0.10 &&  0.20 & 0.19 & 0.17 && 0.29 & 0.27 & 0.23 \\ 
\rowcolor{gray!20}             & Skewed       && 0.05 &&  0.07 & 0.07 & 0.07 && 0.16 & 0.12 & 0.10 \\ 
\rowcolor{gray!20}             &              && 0.10 &&  0.14 & 0.13 & 0.13 && 0.24 & 0.20 & 0.17 \\ 
             &&&&&&&&&&&\\
Heavy tailed & Normal       && 0.05 &&  0.07 & 0.07 & 0.07 && 0.11 & 0.09 & 0.08 \\ 
             &              && 0.10 &&  0.12 & 0.12 & 0.13 && 0.17 & 0.15 & 0.15 \\ 
             & Heavy tailed && 0.05 &&  0.20 & 0.19 & 0.14 && 0.39 & 0.35 & 0.27 \\ 
             &              && 0.10 &&  0.28 & 0.27 & 0.22 && 0.47 & 0.44 & 0.38 \\ 
             & Skewed       && 0.05 &&  0.13 & 0.11 & 0.10 && 0.29 & 0.24 & 0.18 \\ 
             &              && 0.10 &&  0.20 & 0.18 & 0.16 && 0.38 & 0.33 & 0.28 \\ 
             &&&&&&&&&&&\\
Skewed       & Normal       && 0.05 &&  0.06 & 0.06 & 0.06 && 0.08 & 0.07 & 0.06 \\ 
             &              && 0.10 &&  0.12 & 0.11 & 0.12 && 0.15 & 0.14 & 0.11 \\ 
             & Heavy tailed && 0.05 &&  0.16 & 0.15 & 0.13 && 0.33 & 0.28 & 0.23 \\ 
             &              && 0.10 &&  0.22 & 0.21 & 0.20 && 0.42 & 0.37 & 0.33 \\ 
             & Skewed       && 0.05 &&  0.11 & 0.10 & 0.08 && 0.23 & 0.18 & 0.13 \\ 
             &              && 0.10 &&  0.19 & 0.16 & 0.16 && 0.34 & 0.28 & 0.22 \\ 

&&&&&&&&&&&\\
& && && \multicolumn{7}{c}{$\sigma_{\varepsilon}^2 = 1$, \ \ $\sigma_{b_0}^2 = \sigma_{b_1}^2 = 1$} \\ \cline{6-12}
\rowcolor{gray!20}Normal       & Normal       && 0.05 &&  0.05 & 0.05 & 0.05 && 0.04 & 0.04 & 0.04 \\ 
\rowcolor{gray!20}             &              && 0.10 &&  0.10 & 0.10 & 0.11 && 0.08 & 0.09 & 0.09 \\ 
\rowcolor{gray!20}             & Heavy tailed && 0.05 &&  0.09 & 0.09 & 0.09 && 0.14 & 0.12 & 0.10 \\ 
\rowcolor{gray!20}             &              && 0.10 &&  0.16 & 0.15 & 0.14 && 0.20 & 0.18 & 0.16 \\ 
\rowcolor{gray!20}             & Skewed       && 0.05 &&  0.07 & 0.07 & 0.06 && 0.07 & 0.07 & 0.06 \\ 
\rowcolor{gray!20}             &              && 0.10 &&  0.13 & 0.12 & 0.10 && 0.13 & 0.12 & 0.11 \\ 
             &&&&&&&&&&&\\
Heavy tailed & Normal       && 0.05 &&  0.12 & 0.11 & 0.11 && 0.21 & 0.19 & 0.15 \\ 
             &              && 0.10 &&  0.19 & 0.18 & 0.16 && 0.28 & 0.26 & 0.21 \\ 
             & Heavy tailed && 0.05 &&  0.21 & 0.21 & 0.15 && 0.39 & 0.36 & 0.30 \\ 
             &              && 0.10 &&  0.31 & 0.29 & 0.25 && 0.47 & 0.45 & 0.40 \\ 
             & Skewed       && 0.05 &&  0.14 & 0.13 & 0.12 && 0.32 & 0.27 & 0.21 \\ 
             &              && 0.10 &&  0.21 & 0.20 & 0.19 && 0.40 & 0.35 & 0.31 \\ 
             &&&&&&&&&&&\\
Skewed       & Normal       && 0.05 &&  0.10 & 0.09 & 0.07 && 0.19 & 0.16 & 0.11 \\ 
             &              && 0.10 &&  0.18 & 0.16 & 0.14 && 0.27 & 0.23 & 0.19 \\ 
             & Heavy tailed && 0.05 &&  0.18 & 0.16 & 0.12 && 0.33 & 0.28 & 0.21 \\ 
             &              && 0.10 &&  0.26 & 0.24 & 0.20 && 0.44 & 0.38 & 0.30 \\ 
             & Skewed       && 0.05 &&  0.13 & 0.11 & 0.09 && 0.25 & 0.20 & 0.14 \\ 
             &              && 0.10 &&  0.21 & 0.19 & 0.16 && 0.35 & 0.29 & 0.23 \\ 


&&&&&&&&&&&\\
& && && \multicolumn{7}{c}{$\sigma_{\varepsilon}^2 = 1$, \ \ $\sigma_{b_0}^2 = \sigma_{b_1}^2 = 4$} \\ \cline{6-12}
\rowcolor{gray!20}Normal       & Normal       && 0.05 &&  0.06 & 0.05 & 0.05 && 0.05 & 0.05 & 0.05 \\ 
\rowcolor{gray!20}             &              && 0.10 &&  0.11 & 0.12 & 0.10 && 0.10 & 0.09 & 0.09 \\ 
\rowcolor{gray!20}             & Heavy tailed && 0.05 &&  0.08 & 0.07 & 0.07 && 0.08 & 0.07 & 0.07 \\ 
\rowcolor{gray!20}             &              && 0.10 &&  0.13 & 0.13 & 0.12 && 0.13 & 0.12 & 0.12 \\ 
\rowcolor{gray!20}             & Skewed       && 0.05 &&  0.06 & 0.05 & 0.05 && 0.05 & 0.05 & 0.04 \\ 
\rowcolor{gray!20}             &              && 0.10 &&  0.09 & 0.10 & 0.10 && 0.09 & 0.09 & 0.09 \\
             &&&&&&&&&&&\\ 
Heavy tailed & Normal       && 0.05 &&  0.23 & 0.20 & 0.16 && 0.41 & 0.36 & 0.29 \\ 
             &              && 0.10 &&  0.30 & 0.28 & 0.24 && 0.50 & 0.47 & 0.40 \\ 
             & Heavy tailed && 0.05 &&  0.29 & 0.26 & 0.19 && 0.49 & 0.45 & 0.37 \\ 
             &              && 0.10 &&  0.36 & 0.33 & 0.28 && 0.57 & 0.53 & 0.49 \\ 
             & Skewed       && 0.05 &&  0.21 & 0.20 & 0.16 && 0.47 & 0.44 & 0.34 \\ 
             &              && 0.10 &&  0.31 & 0.28 & 0.25 && 0.56 & 0.51 & 0.45 \\ 
             &&&&&&&&&&&\\
Skewed       & Normal       && 0.05 &&  0.20 & 0.17 & 0.12 && 0.38 & 0.29 & 0.22 \\ 
             &              && 0.10 &&  0.30 & 0.26 & 0.21 && 0.49 & 0.39 & 0.32 \\ 
             & Heavy tailed && 0.05 &&  0.25 & 0.22 & 0.17 && 0.45 & 0.36 & 0.26 \\ 
             &              && 0.10 &&  0.36 & 0.33 & 0.27 && 0.55 & 0.46 & 0.39 \\ 
             & Skewed       && 0.05 &&  0.20 & 0.17 & 0.13 && 0.43 & 0.33 & 0.24 \\ 
             &              && 0.10 &&  0.29 & 0.26 & 0.22 && 0.54 & 0.42 & 0.36 \\
\hline
\end{tabular}
\end{scriptsize}
\end{table}

\begin{table}[ht]
\centering
\caption{\label{tab:simb155}Proportion of tests rejecting normality of the random slope using two rotations and $s = 55$.}
\begin{scriptsize}
\begin{tabular}{ll p{.1cm} c p{.1cm} rrr p{.1cm} rrr}
  \hline
  \multicolumn{2}{c}{Distributions}& & Nominal & &  \multicolumn{3}{c}{Rotation} & & \multicolumn{3}{c}{Varimax rotation} \\ \cline{1-2} \cline{6-8} \cline{10-12}   
  Random effects & Errors & & $\alpha$ & & AD & CVM & KS & & AD & CVM & KS \\ 
   \hline
& && && \multicolumn{7}{c}{$\sigma_{\varepsilon}^2 = 4$, \ \ $\sigma_{b_0}^2 = \sigma_{b_1}^2 = 1$} \\ \cline{6-12}
\rowcolor{gray!20}Normal       & Normal       && 0.05 &&  0.04 & 0.04 & 0.04 && 0.05 & 0.05 & 0.06 \\ 
\rowcolor{gray!20}             &              && 0.10 &&  0.07 & 0.08 & 0.09 && 0.10 & 0.10 & 0.10 \\ 
\rowcolor{gray!20}             & Heavy tailed && 0.05 &&  0.12 & 0.11 & 0.10 && 0.19 & 0.17 & 0.13 \\ 
\rowcolor{gray!20}             &              && 0.10 &&  0.18 & 0.17 & 0.15 && 0.26 & 0.24 & 0.21 \\ 
\rowcolor{gray!20}             & Skewed       && 0.05 &&  0.08 & 0.07 & 0.07 && 0.14 & 0.12 & 0.09 \\ 
\rowcolor{gray!20}             &              && 0.10 &&  0.14 & 0.14 & 0.13 && 0.22 & 0.19 & 0.16 \\ 
             &&&&&&&&&&&\\
Heavy tailed & Normal       && 0.05 &&  0.07 & 0.07 & 0.07 && 0.11 & 0.10 & 0.08 \\ 
             &              && 0.10 &&  0.12 & 0.11 & 0.12 && 0.16 & 0.15 & 0.14 \\ 
             & Heavy tailed && 0.05 &&  0.19 & 0.18 & 0.15 && 0.35 & 0.32 & 0.27 \\ 
             &              && 0.10 &&  0.28 & 0.26 & 0.22 && 0.43 & 0.39 & 0.36 \\ 
             & Skewed       && 0.05 &&  0.12 & 0.11 & 0.08 && 0.26 & 0.22 & 0.17 \\ 
             &              && 0.10 &&  0.19 & 0.17 & 0.14 && 0.34 & 0.30 & 0.25 \\ 
             &&&&&&&&&&&\\
Skewed       & Normal       && 0.05 &&  0.06 & 0.06 & 0.05 && 0.07 & 0.07 & 0.06 \\ 
             &              && 0.10 &&  0.10 & 0.10 & 0.10 && 0.15 & 0.14 & 0.12 \\ 
             & Heavy tailed && 0.05 &&  0.15 & 0.14 & 0.12 && 0.30 & 0.26 & 0.23 \\ 
             &              && 0.10 &&  0.22 & 0.21 & 0.19 && 0.40 & 0.36 & 0.32 \\ 
             & Skewed       && 0.05 &&  0.10 & 0.09 & 0.07 && 0.20 & 0.15 & 0.11 \\ 
             &              && 0.10 &&  0.16 & 0.15 & 0.14 && 0.30 & 0.24 & 0.19 \\ 

&&&&&&&&&&&\\
& && && \multicolumn{7}{c}{$\sigma_{\varepsilon}^2 = 1$, \ \ $\sigma_{b_0}^2 = \sigma_{b_1}^2 = 1$} \\ \cline{6-12}
\rowcolor{gray!20}Normal       & Normal       && 0.05 &&  0.05 & 0.05 & 0.06 && 0.04 & 0.05 & 0.05 \\ 
\rowcolor{gray!20}             &              && 0.10 &&  0.09 & 0.11 & 0.11 && 0.08 & 0.10 & 0.10 \\ 
\rowcolor{gray!20}             & Heavy tailed && 0.05 &&  0.09 & 0.09 & 0.08 && 0.12 & 0.11 & 0.10 \\ 
\rowcolor{gray!20}             &              && 0.10 &&  0.16 & 0.14 & 0.13 && 0.20 & 0.19 & 0.16 \\ 
\rowcolor{gray!20}             & Skewed       && 0.05 &&  0.06 & 0.06 & 0.06 && 0.08 & 0.07 & 0.06 \\ 
\rowcolor{gray!20}             &              && 0.10 &&  0.12 & 0.11 & 0.12 && 0.14 & 0.12 & 0.12 \\ 
             &&&&&&&&&&&\\
Heavy tailed & Normal       && 0.05 &&  0.12 & 0.11 & 0.11 && 0.21 & 0.18 & 0.16 \\ 
             &              && 0.10 &&  0.19 & 0.19 & 0.17 && 0.29 & 0.27 & 0.23 \\ 
             & Heavy tailed && 0.05 &&  0.21 & 0.20 & 0.15 && 0.37 & 0.35 & 0.28 \\ 
             &              && 0.10 &&  0.29 & 0.27 & 0.22 && 0.45 & 0.42 & 0.37 \\ 
             & Skewed       && 0.05 &&  0.14 & 0.13 & 0.11 && 0.28 & 0.26 & 0.20 \\ 
             &              && 0.10 &&  0.22 & 0.20 & 0.18 && 0.39 & 0.34 & 0.29 \\ 
             &&&&&&&&&&&\\
Skewed       & Normal       && 0.05 &&  0.10 & 0.09 & 0.07 && 0.18 & 0.15 & 0.12 \\ 
             &              && 0.10 &&  0.18 & 0.16 & 0.13 && 0.26 & 0.23 & 0.22 \\ 
             & Heavy tailed && 0.05 &&  0.17 & 0.16 & 0.13 && 0.32 & 0.27 & 0.20 \\ 
             &              && 0.10 &&  0.24 & 0.22 & 0.20 && 0.42 & 0.36 & 0.30 \\ 
             & Skewed       && 0.05 &&  0.14 & 0.12 & 0.09 && 0.23 & 0.18 & 0.14 \\ 
             &              && 0.10 &&  0.21 & 0.20 & 0.16 && 0.33 & 0.26 & 0.22 \\ 


&&&&&&&&&&&\\
& && && \multicolumn{7}{c}{$\sigma_{\varepsilon}^2 = 1$, \ \ $\sigma_{b_0}^2 = \sigma_{b_1}^2 = 4$} \\ \cline{6-12}
\rowcolor{gray!20}Normal       & Normal       && 0.05 &&  0.06 & 0.06 & 0.05 && 0.04 & 0.04 & 0.04 \\ 
\rowcolor{gray!20}             &              && 0.10 &&  0.11 & 0.11 & 0.11 && 0.09 & 0.09 & 0.08 \\ 
\rowcolor{gray!20}             & Heavy tailed && 0.05 &&  0.07 & 0.07 & 0.07 && 0.08 & 0.07 & 0.06 \\ 
\rowcolor{gray!20}             &              && 0.10 &&  0.14 & 0.13 & 0.13 && 0.12 & 0.12 & 0.11 \\ 
\rowcolor{gray!20}             & Skewed       && 0.05 &&  0.06 & 0.06 & 0.06 && 0.05 & 0.05 & 0.05 \\ 
\rowcolor{gray!20}             &              && 0.10 &&  0.10 & 0.12 & 0.11 && 0.09 & 0.09 & 0.10 \\ 
             &&&&&&&&&&&\\
Heavy tailed & Normal       && 0.05 &&  0.23 & 0.20 & 0.15 && 0.40 & 0.36 & 0.31 \\ 
             &              && 0.10 &&  0.31 & 0.28 & 0.24 && 0.50 & 0.46 & 0.40 \\ 
             & Heavy tailed && 0.05 &&  0.27 & 0.24 & 0.19 && 0.46 & 0.41 & 0.34 \\ 
             &              && 0.10 &&  0.36 & 0.33 & 0.27 && 0.54 & 0.50 & 0.45 \\ 
             & Skewed       && 0.05 &&  0.22 & 0.20 & 0.17 && 0.44 & 0.40 & 0.31 \\ 
             &              && 0.10 &&  0.31 & 0.28 & 0.24 && 0.54 & 0.49 & 0.42 \\ 
             &&&&&&&&&&&\\
Skewed       & Normal       && 0.05 &&  0.18 & 0.16 & 0.11 && 0.37 & 0.29 & 0.22 \\ 
             &              && 0.10 &&  0.29 & 0.25 & 0.21 && 0.47 & 0.38 & 0.32 \\ 
             & Heavy tailed && 0.05 &&  0.23 & 0.20 & 0.15 && 0.45 & 0.35 & 0.26 \\ 
             &              && 0.10 &&  0.33 & 0.30 & 0.25 && 0.54 & 0.45 & 0.35 \\ 
             & Skewed       && 0.05 &&  0.18 & 0.17 & 0.13 && 0.39 & 0.30 & 0.22 \\ 
             &              && 0.10 &&  0.29 & 0.25 & 0.22 && 0.52 & 0.41 & 0.34 \\ 


\hline
\end{tabular}
\end{scriptsize}
\end{table}

\begin{table}[ht]
\centering
\caption{\label{tab:simb150}Proportion of tests rejecting normality of the random slope using two rotations and $s = 50$.}
\begin{scriptsize}
\begin{tabular}{ll p{.1cm} c p{.1cm} rrr p{.1cm} rrr}
  \hline
  \multicolumn{2}{c}{Distributions}& & Nominal & &  \multicolumn{3}{c}{Rotation} & & \multicolumn{3}{c}{Varimax rotation} \\ \cline{1-2} \cline{6-8} \cline{10-12}   
  Random effects & Errors & & $\alpha$ & & AD & CVM & KS & & AD & CVM & KS \\ 
   \hline
& && && \multicolumn{7}{c}{$\sigma_{\varepsilon}^2 = 4$, \ \ $\sigma_{b_0}^2 = \sigma_{b_1}^2 = 1$} \\ \cline{6-12}
\rowcolor{gray!20}Normal       & Normal       && 0.05 &&  0.04 & 0.04 & 0.04 && 0.05 & 0.05 & 0.06 \\ 
\rowcolor{gray!20}             &              && 0.10 &&  0.08 & 0.08 & 0.09 && 0.10 & 0.10 & 0.11 \\ 
\rowcolor{gray!20}             & Heavy tailed && 0.05 &&  0.10 & 0.11 & 0.09 && 0.16 & 0.15 & 0.12 \\ 
\rowcolor{gray!20}             &              && 0.10 &&  0.16 & 0.15 & 0.15 && 0.24 & 0.21 & 0.20 \\ 
\rowcolor{gray!20}             & Skewed       && 0.05 &&  0.07 & 0.06 & 0.05 && 0.13 & 0.11 & 0.09 \\ 
\rowcolor{gray!20}             &              && 0.10 &&  0.12 & 0.12 & 0.12 && 0.20 & 0.18 & 0.15 \\ 
             &&&&&&&&&&&\\
Heavy tailed & Normal       && 0.05 &&  0.07 & 0.07 & 0.06 && 0.10 & 0.09 & 0.08 \\ 
             &              && 0.10 &&  0.12 & 0.12 & 0.11 && 0.16 & 0.15 & 0.15 \\ 
             & Heavy tailed && 0.05 &&  0.19 & 0.17 & 0.14 && 0.34 & 0.32 & 0.26 \\ 
             &              && 0.10 &&  0.26 & 0.24 & 0.21 && 0.43 & 0.40 & 0.35 \\ 
             & Skewed       && 0.05 &&  0.11 & 0.11 & 0.08 && 0.23 & 0.19 & 0.15 \\ 
             &              && 0.10 &&  0.19 & 0.17 & 0.14 && 0.29 & 0.26 & 0.23 \\ 
             &&&&&&&&&&&\\
Skewed       & Normal       && 0.05 &&  0.06 & 0.06 & 0.06 && 0.08 & 0.08 & 0.07 \\ 
             &              && 0.10 &&  0.11 & 0.12 & 0.11 && 0.14 & 0.13 & 0.13 \\ 
             & Heavy tailed && 0.05 &&  0.14 & 0.12 & 0.11 && 0.29 & 0.25 & 0.19 \\ 
             &              && 0.10 &&  0.19 & 0.19 & 0.17 && 0.37 & 0.33 & 0.29 \\ 
             & Skewed       && 0.05 &&  0.10 & 0.09 & 0.08 && 0.20 & 0.15 & 0.11 \\ 
             &              && 0.10 &&  0.15 & 0.15 & 0.14 && 0.29 & 0.23 & 0.20 \\ 

&&&&&&&&&&&\\
& && && \multicolumn{7}{c}{$\sigma_{\varepsilon}^2 = 1$, \ \ $\sigma_{b_0}^2 = \sigma_{b_1}^2 = 1$} \\ \cline{6-12}
\rowcolor{gray!20}Normal       & Normal       && 0.05 &&  0.05 & 0.05 & 0.05 && 0.05 & 0.04 & 0.05 \\ 
\rowcolor{gray!20}             &              && 0.10 &&  0.10 & 0.11 & 0.10 && 0.10 & 0.10 & 0.09 \\ 
\rowcolor{gray!20}             & Heavy tailed && 0.05 &&  0.10 & 0.09 & 0.08 && 0.12 & 0.10 & 0.09 \\ 
\rowcolor{gray!20}             &              && 0.10 &&  0.15 & 0.15 & 0.12 && 0.19 & 0.17 & 0.14 \\ 
\rowcolor{gray!20}             & Skewed       && 0.05 &&  0.06 & 0.05 & 0.05 && 0.06 & 0.05 & 0.05 \\ 
\rowcolor{gray!20}             &              && 0.10 &&  0.12 & 0.11 & 0.10 && 0.13 & 0.12 & 0.10 \\ 
             &&&&&&&&&&&\\
Heavy tailed & Normal       && 0.05 &&  0.13 & 0.12 & 0.11 && 0.22 & 0.18 & 0.16 \\ 
             &              && 0.10 &&  0.20 & 0.19 & 0.16 && 0.30 & 0.27 & 0.24 \\ 
             & Heavy tailed && 0.05 &&  0.20 & 0.18 & 0.14 && 0.36 & 0.34 & 0.27 \\ 
             &              && 0.10 &&  0.28 & 0.26 & 0.21 && 0.42 & 0.41 & 0.35 \\ 
             & Skewed       && 0.05 &&  0.14 & 0.12 & 0.11 && 0.27 & 0.24 & 0.19 \\ 
             &              && 0.10 &&  0.21 & 0.20 & 0.17 && 0.37 & 0.33 & 0.29 \\ 
             &&&&&&&&&&&\\
Skewed       & Normal       && 0.05 &&  0.10 & 0.08 & 0.07 && 0.19 & 0.15 & 0.10 \\ 
             &              && 0.10 &&  0.17 & 0.15 & 0.14 && 0.27 & 0.23 & 0.18 \\ 
             & Heavy tailed && 0.05 &&  0.17 & 0.15 & 0.13 && 0.30 & 0.26 & 0.19 \\ 
             &              && 0.10 &&  0.24 & 0.23 & 0.19 && 0.40 & 0.35 & 0.28 \\ 
             & Skewed       && 0.05 &&  0.12 & 0.12 & 0.09 && 0.23 & 0.17 & 0.13 \\ 
             &              && 0.10 &&  0.20 & 0.18 & 0.17 && 0.31 & 0.25 & 0.20 \\ 


&&&&&&&&&&&\\
& && && \multicolumn{7}{c}{$\sigma_{\varepsilon}^2 = 1$, \ \ $\sigma_{b_0}^2 = \sigma_{b_1}^2 = 4$} \\ \cline{6-12}
\rowcolor{gray!20}Normal       & Normal       && 0.05 &&  0.05 & 0.05 & 0.06 && 0.04 & 0.04 & 0.04 \\ 
\rowcolor{gray!20}             &              && 0.10 &&  0.10 & 0.10 & 0.11 && 0.10 & 0.10 & 0.09 \\ 
\rowcolor{gray!20}             & Heavy tailed && 0.05 &&  0.06 & 0.06 & 0.05 && 0.06 & 0.06 & 0.06 \\ 
\rowcolor{gray!20}             &              && 0.10 &&  0.12 & 0.11 & 0.12 && 0.13 & 0.12 & 0.10 \\ 
\rowcolor{gray!20}             & Skewed       && 0.05 &&  0.06 & 0.06 & 0.06 && 0.05 & 0.05 & 0.05 \\ 
\rowcolor{gray!20}             &              && 0.10 &&  0.10 & 0.11 & 0.10 && 0.10 & 0.09 & 0.09 \\ 
 &&&&&&&&&&&\\
Heavy tailed & Normal       && 0.05 &&  0.22 & 0.19 & 0.15 && 0.38 & 0.34 & 0.28 \\ 
             &              && 0.10 &&  0.30 & 0.26 & 0.24 && 0.48 & 0.43 & 0.38 \\ 
             & Heavy tailed && 0.05 &&  0.25 & 0.23 & 0.19 && 0.43 & 0.40 & 0.32 \\ 
             &              && 0.10 &&  0.33 & 0.30 & 0.26 && 0.50 & 0.47 & 0.42 \\ 
             & Skewed       && 0.05 &&  0.20 & 0.19 & 0.15 && 0.42 & 0.38 & 0.30 \\ 
             &              && 0.10 &&  0.29 & 0.27 & 0.23 && 0.49 & 0.46 & 0.39 \\ 
              &&&&&&&&&&&\\
Skewed       & Normal       && 0.05 &&  0.18 & 0.15 & 0.13 && 0.36 & 0.30 & 0.22 \\ 
             &              && 0.10 &&  0.29 & 0.25 & 0.21 && 0.48 & 0.39 & 0.32 \\ 
             & Heavy tailed && 0.05 &&  0.22 & 0.19 & 0.15 && 0.41 & 0.33 & 0.26 \\ 
             &              && 0.10 &&  0.33 & 0.30 & 0.24 && 0.53 & 0.43 & 0.36 \\ 
             & Skewed       && 0.05 &&  0.19 & 0.16 & 0.12 && 0.39 & 0.30 & 0.23 \\ 
             &              && 0.10 &&  0.28 & 0.25 & 0.21 && 0.49 & 0.40 & 0.34 \\

\hline
\end{tabular}
\end{scriptsize}
\end{table}


\begin{table}[ht]
\centering
\caption{\label{tab:simb145}Proportion of tests rejecting normality of the random slope using two rotations and $s = 45$.}
\begin{scriptsize}
\begin{tabular}{ll p{.1cm} c p{.1cm} rrr p{.1cm} rrr}
  \hline
  \multicolumn{2}{c}{Distributions}& & Nominal & &  \multicolumn{3}{c}{Rotation} & & \multicolumn{3}{c}{Varimax rotation} \\ \cline{1-2} \cline{6-8} \cline{10-12}   
  Random effects & Errors & & $\alpha$ & & AD & CVM & KS & & AD & CVM & KS \\ 
   \hline
& && && \multicolumn{7}{c}{$\sigma_{\varepsilon}^2 = 4$, \ \ $\sigma_{b_0}^2 = \sigma_{b_1}^2 = 1$} \\ \cline{6-12}
\rowcolor{gray!20}Normal       & Normal       && 0.05 &&  0.04 & 0.03 & 0.04 && 0.05 & 0.05 & 0.05 \\ 
\rowcolor{gray!20}             &              && 0.10 &&  0.07 & 0.08 & 0.08 && 0.10 & 0.10 & 0.09 \\ 
\rowcolor{gray!20}             & Heavy tailed && 0.05 &&  0.10 & 0.10 & 0.09 && 0.15 & 0.13 & 0.10 \\ 
\rowcolor{gray!20}             &              && 0.10 &&  0.16 & 0.16 & 0.14 && 0.21 & 0.18 & 0.15 \\ 
\rowcolor{gray!20}             & Skewed       && 0.05 &&  0.07 & 0.07 & 0.05 && 0.09 & 0.09 & 0.06 \\ 
\rowcolor{gray!20}             &              && 0.10 &&  0.13 & 0.12 & 0.11 && 0.17 & 0.14 & 0.13 \\ 
             &&&&&&&&&&&\\
Heavy tailed & Normal       && 0.05 &&  0.07 & 0.07 & 0.05 && 0.11 & 0.11 & 0.09 \\ 
             &              && 0.10 &&  0.11 & 0.11 & 0.10 && 0.18 & 0.16 & 0.14 \\ 
             & Heavy tailed && 0.05 &&  0.18 & 0.17 & 0.14 && 0.30 & 0.28 & 0.23 \\ 
             &              && 0.10 &&  0.25 & 0.23 & 0.21 && 0.38 & 0.35 & 0.31 \\ 
             & Skewed       && 0.05 &&  0.10 & 0.10 & 0.07 && 0.20 & 0.17 & 0.14 \\ 
             &              && 0.10 &&  0.19 & 0.16 & 0.14 && 0.28 & 0.25 & 0.20 \\ 
             &&&&&&&&&&&\\
Skewed       & Normal       && 0.05 &&  0.07 & 0.06 & 0.06 && 0.08 & 0.07 & 0.07 \\ 
             &              && 0.10 &&  0.12 & 0.12 & 0.11 && 0.14 & 0.13 & 0.12 \\ 
             & Heavy tailed && 0.05 &&  0.12 & 0.11 & 0.10 && 0.24 & 0.21 & 0.16 \\ 
             &              && 0.10 &&  0.19 & 0.18 & 0.17 && 0.33 & 0.30 & 0.25 \\ 
             & Skewed       && 0.05 &&  0.10 & 0.10 & 0.08 && 0.17 & 0.13 & 0.11 \\ 
             &              && 0.10 &&  0.17 & 0.15 & 0.14 && 0.25 & 0.22 & 0.19 \\ 

&&&&&&&&&&&\\
& && && \multicolumn{7}{c}{$\sigma_{\varepsilon}^2 = 1$, \ \ $\sigma_{b_0}^2 = \sigma_{b_1}^2 = 1$} \\ \cline{6-12}
\rowcolor{gray!20}Normal       & Normal       && 0.05 &&  0.05 & 0.05 & 0.05 && 0.04 & 0.05 & 0.05 \\ 
\rowcolor{gray!20}             &              && 0.10 &&  0.10 & 0.10 & 0.10 && 0.09 & 0.09 & 0.09 \\ 
\rowcolor{gray!20}             & Heavy tailed && 0.05 &&  0.10 & 0.08 & 0.07 && 0.11 & 0.10 & 0.09 \\ 
\rowcolor{gray!20}             &              && 0.10 &&  0.15 & 0.14 & 0.13 && 0.16 & 0.14 & 0.14 \\ 
\rowcolor{gray!20}             & Skewed       && 0.05 &&  0.06 & 0.06 & 0.05 && 0.06 & 0.06 & 0.06 \\ 
\rowcolor{gray!20}             &              && 0.10 &&  0.11 & 0.11 & 0.10 && 0.13 & 0.12 & 0.11 \\ 
             &&&&&&&&&&&\\
Heavy tailed & Normal       && 0.05 &&  0.14 & 0.12 & 0.10 && 0.21 & 0.18 & 0.15 \\ 
             &              && 0.10 &&  0.21 & 0.19 & 0.15 && 0.28 & 0.25 & 0.21 \\ 
             & Heavy tailed && 0.05 &&  0.18 & 0.17 & 0.14 && 0.32 & 0.29 & 0.24 \\ 
             &              && 0.10 &&  0.26 & 0.24 & 0.22 && 0.40 & 0.36 & 0.32 \\ 
             & Skewed       && 0.05 &&  0.14 & 0.13 & 0.12 && 0.25 & 0.23 & 0.19 \\ 
             &              && 0.10 &&  0.20 & 0.18 & 0.18 && 0.35 & 0.31 & 0.28 \\ 
             &&&&&&&&&&&\\
Skewed       & Normal       && 0.05 &&  0.10 & 0.09 & 0.07 && 0.18 & 0.15 & 0.13 \\ 
             &              && 0.10 &&  0.17 & 0.15 & 0.14 && 0.27 & 0.23 & 0.19 \\ 
             & Heavy tailed && 0.05 &&  0.16 & 0.15 & 0.13 && 0.28 & 0.23 & 0.17 \\ 
             &              && 0.10 &&  0.23 & 0.22 & 0.19 && 0.37 & 0.33 & 0.25 \\ 
             & Skewed       && 0.05 &&  0.12 & 0.11 & 0.09 && 0.22 & 0.17 & 0.12 \\ 
             &              && 0.10 &&  0.20 & 0.18 & 0.16 && 0.33 & 0.26 & 0.21 \\ 


&&&&&&&&&&&\\
& && && \multicolumn{7}{c}{$\sigma_{\varepsilon}^2 = 1$, \ \ $\sigma_{b_0}^2 = \sigma_{b_1}^2 = 4$} \\ \cline{6-12}
\rowcolor{gray!20}Normal       & Normal       && 0.05 &&  0.05 & 0.05 & 0.05 && 0.05 & 0.05 & 0.05 \\ 
\rowcolor{gray!20}             &              && 0.10 &&  0.11 & 0.11 & 0.10 && 0.10 & 0.10 & 0.10 \\ 
\rowcolor{gray!20}             & Heavy tailed && 0.05 &&  0.05 & 0.05 & 0.06 && 0.05 & 0.05 & 0.05 \\ 
\rowcolor{gray!20}             &              && 0.10 &&  0.11 & 0.11 & 0.12 && 0.11 & 0.10 & 0.10 \\ 
\rowcolor{gray!20}             & Skewed       && 0.05 &&  0.05 & 0.06 & 0.05 && 0.06 & 0.05 & 0.05 \\ 
\rowcolor{gray!20}             &              && 0.10 &&  0.09 & 0.09 & 0.10 && 0.10 & 0.10 & 0.09 \\ 
             &&&&&&&&&&&\\
Heavy tailed & Normal       && 0.05 &&  0.20 & 0.19 & 0.13 && 0.36 & 0.32 & 0.24 \\ 
             &              && 0.10 &&  0.28 & 0.26 & 0.22 && 0.45 & 0.41 & 0.35 \\ 
             & Heavy tailed && 0.05 &&  0.24 & 0.22 & 0.17 && 0.42 & 0.39 & 0.32 \\ 
             &              && 0.10 &&  0.32 & 0.30 & 0.26 && 0.52 & 0.47 & 0.41 \\ 
             & Skewed       && 0.05 &&  0.19 & 0.17 & 0.14 && 0.40 & 0.37 & 0.28 \\ 
             &              && 0.10 &&  0.28 & 0.26 & 0.24 && 0.48 & 0.44 & 0.38 \\ 
             &&&&&&&&&&&\\
Skewed       & Normal       && 0.05 &&  0.18 & 0.15 & 0.12 && 0.35 & 0.29 & 0.21 \\ 
             &              && 0.10 &&  0.27 & 0.24 & 0.21 && 0.47 & 0.37 & 0.32 \\ 
             & Heavy tailed && 0.05 &&  0.20 & 0.18 & 0.15 && 0.40 & 0.32 & 0.24 \\ 
             &              && 0.10 &&  0.29 & 0.27 & 0.22 && 0.48 & 0.41 & 0.33 \\ 
             & Skewed       && 0.05 &&  0.18 & 0.16 & 0.13 && 0.36 & 0.28 & 0.21 \\ 
             &              && 0.10 &&  0.27 & 0.24 & 0.21 && 0.47 & 0.38 & 0.32 \\ 

\hline
\end{tabular}
\end{scriptsize}
\end{table}


\begin{table}[ht]
\centering
\caption{\label{tab:simb140}Proportion of tests rejecting normality of the random slope using two rotations and $s = 40$.}
\begin{scriptsize}
\begin{tabular}{ll p{.1cm} c p{.1cm} rrr p{.1cm} rrr}
  \hline
  \multicolumn{2}{c}{Distributions}& & Nominal & &  \multicolumn{3}{c}{Rotation} & & \multicolumn{3}{c}{Varimax rotation} \\ \cline{1-2} \cline{6-8} \cline{10-12}   
  Random effects & Errors & & $\alpha$ & & AD & CVM & KS & & AD & CVM & KS \\ 
   \hline
& && && \multicolumn{7}{c}{$\sigma_{\varepsilon}^2 = 4$, \ \ $\sigma_{b_0}^2 = \sigma_{b_1}^2 = 1$} \\ \cline{6-12}
\rowcolor{gray!20}Normal       & Normal       && 0.05 &&  0.04 & 0.04 & 0.05 && 0.05 & 0.05 & 0.05 \\ 
\rowcolor{gray!20}             &              && 0.10 &&  0.09 & 0.10 & 0.10 && 0.10 & 0.10 & 0.10 \\ 
\rowcolor{gray!20}             & Heavy tailed && 0.05 &&  0.10 & 0.10 & 0.08 && 0.12 & 0.12 & 0.10 \\ 
\rowcolor{gray!20}             &              && 0.10 &&  0.16 & 0.16 & 0.14 && 0.18 & 0.17 & 0.15 \\ 
\rowcolor{gray!20}             & Skewed       && 0.05 &&  0.06 & 0.06 & 0.05 && 0.07 & 0.06 & 0.05 \\ 
\rowcolor{gray!20}             &              && 0.10 &&  0.12 & 0.11 & 0.10 && 0.14 & 0.13 & 0.10 \\ 
             &&&&&&&&&&&\\
Heavy tailed & Normal       && 0.05 &&  0.06 & 0.05 & 0.06 && 0.11 & 0.10 & 0.08 \\ 
             &              && 0.10 &&  0.11 & 0.10 & 0.10 && 0.17 & 0.15 & 0.13 \\ 
             & Heavy tailed && 0.05 &&  0.16 & 0.15 & 0.12 && 0.28 & 0.24 & 0.19 \\ 
             &              && 0.10 &&  0.22 & 0.21 & 0.19 && 0.36 & 0.33 & 0.28 \\ 
             & Skewed       && 0.05 &&  0.09 & 0.08 & 0.07 && 0.18 & 0.16 & 0.13 \\ 
             &              && 0.10 &&  0.15 & 0.14 & 0.12 && 0.27 & 0.23 & 0.19 \\ 
             &&&&&&&&&&&\\
Skewed       & Normal       && 0.05 &&  0.05 & 0.06 & 0.05 && 0.08 & 0.08 & 0.07 \\ 
             &              && 0.10 &&  0.10 & 0.09 & 0.10 && 0.15 & 0.14 & 0.12 \\ 
             & Heavy tailed && 0.05 &&  0.12 & 0.11 & 0.08 && 0.22 & 0.18 & 0.15 \\ 
             &              && 0.10 &&  0.17 & 0.17 & 0.15 && 0.30 & 0.27 & 0.23 \\ 
             & Skewed       && 0.05 &&  0.10 & 0.09 & 0.07 && 0.16 & 0.13 & 0.09 \\ 
             &              && 0.10 &&  0.15 & 0.14 & 0.14 && 0.24 & 0.20 & 0.17 \\ 

&&&&&&&&&&&\\
& && && \multicolumn{7}{c}{$\sigma_{\varepsilon}^2 = 1$, \ \ $\sigma_{b_0}^2 = \sigma_{b_1}^2 = 1$} \\ \cline{6-12}
\rowcolor{gray!20}Normal       & Normal       && 0.05 &&  0.05 & 0.05 & 0.05 && 0.05 & 0.05 & 0.05 \\ 
\rowcolor{gray!20}             &              && 0.10 &&  0.10 & 0.10 & 0.10 && 0.09 & 0.10 & 0.10 \\ 
\rowcolor{gray!20}             & Heavy tailed && 0.05 &&  0.09 & 0.09 & 0.07 && 0.08 & 0.08 & 0.07 \\ 
\rowcolor{gray!20}             &              && 0.10 &&  0.15 & 0.14 & 0.11 && 0.14 & 0.13 & 0.12 \\ 
\rowcolor{gray!20}             & Skewed       && 0.05 &&  0.06 & 0.06 & 0.05 && 0.06 & 0.05 & 0.05 \\ 
\rowcolor{gray!20}             &              && 0.10 &&  0.12 & 0.12 & 0.11 && 0.11 & 0.12 & 0.11 \\ 
             &&&&&&&&&&&\\
Heavy tailed & Normal       && 0.05 &&  0.13 & 0.12 & 0.09 && 0.19 & 0.17 & 0.13 \\ 
             &              && 0.10 &&  0.19 & 0.19 & 0.16 && 0.26 & 0.24 & 0.21 \\ 
             & Heavy tailed && 0.05 &&  0.18 & 0.16 & 0.12 && 0.31 & 0.29 & 0.22 \\ 
             &              && 0.10 &&  0.24 & 0.23 & 0.21 && 0.39 & 0.36 & 0.32 \\ 
             & Skewed       && 0.05 &&  0.13 & 0.12 & 0.09 && 0.24 & 0.22 & 0.18 \\ 
             &              && 0.10 &&  0.18 & 0.17 & 0.17 && 0.34 & 0.30 & 0.27 \\
             &&&&&&&&&&&\\ 
Skewed       & Normal       && 0.05 &&  0.10 & 0.09 & 0.08 && 0.18 & 0.14 & 0.12 \\ 
             &              && 0.10 &&  0.18 & 0.16 & 0.14 && 0.27 & 0.24 & 0.19 \\ 
             & Heavy tailed && 0.05 &&  0.14 & 0.14 & 0.11 && 0.26 & 0.22 & 0.18 \\ 
             &              && 0.10 &&  0.22 & 0.20 & 0.17 && 0.36 & 0.32 & 0.26 \\ 
             & Skewed       && 0.05 &&  0.13 & 0.11 & 0.07 && 0.21 & 0.16 & 0.12 \\ 
             &              && 0.10 &&  0.18 & 0.18 & 0.16 && 0.30 & 0.25 & 0.21 \\ 


&&&&&&&&&&&\\
& && && \multicolumn{7}{c}{$\sigma_{\varepsilon}^2 = 1$, \ \ $\sigma_{b_0}^2 = \sigma_{b_1}^2 = 4$} \\ \cline{6-12}
\rowcolor{gray!20}Normal       & Normal       && 0.05 &&  0.05 & 0.05 & 0.05 && 0.05 & 0.05 & 0.05 \\ 
\rowcolor{gray!20}             &              && 0.10 &&  0.11 & 0.11 & 0.10 && 0.11 & 0.10 & 0.10 \\ 
\rowcolor{gray!20}             & Heavy tailed && 0.05 &&  0.06 & 0.06 & 0.06 && 0.06 & 0.06 & 0.05 \\ 
\rowcolor{gray!20}             &              && 0.10 &&  0.11 & 0.12 & 0.11 && 0.11 & 0.11 & 0.10 \\ 
\rowcolor{gray!20}             & Skewed       && 0.05 &&  0.04 & 0.05 & 0.05 && 0.04 & 0.04 & 0.04 \\ 
\rowcolor{gray!20}             &              && 0.10 &&  0.09 & 0.09 & 0.08 && 0.09 & 0.10 & 0.10 \\ 
             &&&&&&&&&&&\\
Heavy tailed & Normal       && 0.05 &&  0.19 & 0.17 & 0.13 && 0.36 & 0.32 & 0.27 \\ 
             &              && 0.10 &&  0.27 & 0.24 & 0.21 && 0.46 & 0.41 & 0.35 \\ 
             & Heavy tailed && 0.05 &&  0.23 & 0.21 & 0.16 && 0.38 & 0.34 & 0.29 \\ 
             &              && 0.10 &&  0.30 & 0.28 & 0.24 && 0.47 & 0.43 & 0.38 \\ 
             & Skewed       && 0.05 &&  0.19 & 0.17 & 0.12 && 0.37 & 0.34 & 0.28 \\ 
             &              && 0.10 &&  0.26 & 0.23 & 0.22 && 0.45 & 0.43 & 0.36 \\ 
             &&&&&&&&&&&\\
Skewed       & Normal       && 0.05 &&  0.17 & 0.15 & 0.11 && 0.34 & 0.28 & 0.21 \\ 
             &              && 0.10 &&  0.26 & 0.24 & 0.20 && 0.44 & 0.37 & 0.31 \\ 
             & Heavy tailed && 0.05 &&  0.20 & 0.19 & 0.15 && 0.36 & 0.27 & 0.21 \\ 
             &              && 0.10 &&  0.28 & 0.26 & 0.23 && 0.47 & 0.38 & 0.31 \\ 
             & Skewed       && 0.05 &&  0.17 & 0.14 & 0.12 && 0.35 & 0.27 & 0.20 \\ 
             &              && 0.10 &&  0.28 & 0.24 & 0.21 && 0.46 & 0.37 & 0.29 \\

\hline
\end{tabular}
\end{scriptsize}
\end{table}


\begin{table}[ht]
\centering
\caption{\label{tab:simb135}Proportion of tests rejecting normality of the random slope using two rotations and $s = 35$.}
\begin{scriptsize}
\begin{tabular}{ll p{.1cm} c p{.1cm} rrr p{.1cm} rrr}
  \hline
  \multicolumn{2}{c}{Distributions}& & Nominal & &  \multicolumn{3}{c}{Rotation} & & \multicolumn{3}{c}{Varimax rotation} \\ \cline{1-2} \cline{6-8} \cline{10-12}   
  Random effects & Errors & & $\alpha$ & & AD & CVM & KS & & AD & CVM & KS \\ 
   \hline
& && && \multicolumn{7}{c}{$\sigma_{\varepsilon}^2 = 4$, \ \ $\sigma_{b_0}^2 = \sigma_{b_1}^2 = 1$} \\ \cline{6-12}
\rowcolor{gray!20}Normal       & Normal       && 0.05 &&  0.05 & 0.06 & 0.05 && 0.04 & 0.04 & 0.05 \\ 
\rowcolor{gray!20}             &              && 0.10 &&  0.10 & 0.11 & 0.10 && 0.09 & 0.09 & 0.09 \\ 
\rowcolor{gray!20}             & Heavy tailed && 0.05 &&  0.09 & 0.08 & 0.07 && 0.12 & 0.10 & 0.10 \\ 
\rowcolor{gray!20}             &              && 0.10 &&  0.15 & 0.14 & 0.13 && 0.19 & 0.17 & 0.16 \\ 
\rowcolor{gray!20}             & Skewed       && 0.05 &&  0.06 & 0.05 & 0.04 && 0.07 & 0.07 & 0.05 \\ 
\rowcolor{gray!20}             &              && 0.10 &&  0.10 & 0.11 & 0.10 && 0.12 & 0.11 & 0.10 \\ 
             &&&&&&&&&&&\\
Heavy tailed & Normal       && 0.05 &&  0.07 & 0.06 & 0.07 && 0.12 & 0.11 & 0.09 \\ 
             &              && 0.10 &&  0.12 & 0.11 & 0.11 && 0.18 & 0.16 & 0.15 \\ 
             & Heavy tailed && 0.05 &&  0.14 & 0.14 & 0.11 && 0.26 & 0.23 & 0.19 \\ 
             &              && 0.10 &&  0.21 & 0.20 & 0.16 && 0.33 & 0.31 & 0.27 \\ 
             & Skewed       && 0.05 &&  0.07 & 0.07 & 0.06 && 0.16 & 0.13 & 0.12 \\ 
             &              && 0.10 &&  0.15 & 0.13 & 0.12 && 0.23 & 0.21 & 0.19 \\ 
             &&&&&&&&&&&\\
Skewed       & Normal       && 0.05 &&  0.06 & 0.06 & 0.06 && 0.08 & 0.07 & 0.07 \\ 
             &              && 0.10 &&  0.10 & 0.11 & 0.11 && 0.14 & 0.13 & 0.12 \\ 
             & Heavy tailed && 0.05 &&  0.11 & 0.10 & 0.08 && 0.20 & 0.18 & 0.14 \\ 
             &              && 0.10 &&  0.19 & 0.17 & 0.14 && 0.28 & 0.26 & 0.22 \\ 
             & Skewed       && 0.05 &&  0.08 & 0.08 & 0.06 && 0.14 & 0.13 & 0.10 \\ 
             &              && 0.10 &&  0.16 & 0.14 & 0.12 && 0.22 & 0.19 & 0.18 \\ 

&&&&&&&&&&&\\
& && && \multicolumn{7}{c}{$\sigma_{\varepsilon}^2 = 1$, \ \ $\sigma_{b_0}^2 = \sigma_{b_1}^2 = 1$} \\ \cline{6-12}
\rowcolor{gray!20}Normal       & Normal       && 0.05 &&  0.05 & 0.05 & 0.04 && 0.05 & 0.05 & 0.04 \\ 
\rowcolor{gray!20}             &              && 0.10 &&  0.09 & 0.08 & 0.09 && 0.09 & 0.10 & 0.10 \\ 
\rowcolor{gray!20}             & Heavy tailed && 0.05 &&  0.08 & 0.07 & 0.07 && 0.08 & 0.07 & 0.07 \\ 
\rowcolor{gray!20}             &              && 0.10 &&  0.14 & 0.14 & 0.12 && 0.14 & 0.13 & 0.12 \\ 
\rowcolor{gray!20}             & Skewed       && 0.05 &&  0.06 & 0.06 & 0.05 && 0.04 & 0.05 & 0.04 \\ 
\rowcolor{gray!20}             &              && 0.10 &&  0.11 & 0.10 & 0.10 && 0.10 & 0.11 & 0.10 \\ 
             &&&&&&&&&&&\\
Heavy tailed & Normal       && 0.05 &&  0.13 & 0.11 & 0.10 && 0.22 & 0.19 & 0.16 \\ 
             &              && 0.10 &&  0.19 & 0.19 & 0.17 && 0.29 & 0.27 & 0.23 \\ 
             & Heavy tailed && 0.05 &&  0.17 & 0.15 & 0.12 && 0.29 & 0.28 & 0.22 \\ 
             &              && 0.10 &&  0.23 & 0.22 & 0.20 && 0.36 & 0.35 & 0.31 \\ 
             & Skewed       && 0.05 &&  0.10 & 0.10 & 0.09 && 0.24 & 0.21 & 0.18 \\ 
             &              && 0.10 &&  0.17 & 0.15 & 0.15 && 0.33 & 0.30 & 0.26 \\ 
             &&&&&&&&&&&\\
Skewed       & Normal       && 0.05 &&  0.12 & 0.10 & 0.08 && 0.19 & 0.15 & 0.12 \\ 
             &              && 0.10 &&  0.18 & 0.16 & 0.15 && 0.28 & 0.24 & 0.20 \\ 
             & Heavy tailed && 0.05 &&  0.12 & 0.10 & 0.09 && 0.22 & 0.20 & 0.16 \\ 
             &              && 0.10 &&  0.18 & 0.17 & 0.14 && 0.31 & 0.27 & 0.23 \\ 
             & Skewed       && 0.05 &&  0.12 & 0.11 & 0.08 && 0.20 & 0.17 & 0.13 \\ 
             &              && 0.10 &&  0.18 & 0.17 & 0.15 && 0.28 & 0.24 & 0.21 \\ 


&&&&&&&&&&&\\
& && && \multicolumn{7}{c}{$\sigma_{\varepsilon}^2 = 1$, \ \ $\sigma_{b_0}^2 = \sigma_{b_1}^2 = 4$} \\ \cline{6-12}
\rowcolor{gray!20}Normal       & Normal       && 0.05 &&  0.05 & 0.06 & 0.06 && 0.06 & 0.07 & 0.06 \\ 
\rowcolor{gray!20}             &              && 0.10 &&  0.11 & 0.11 & 0.11 && 0.11 & 0.11 & 0.11 \\ 
\rowcolor{gray!20}             & Heavy tailed && 0.05 &&  0.05 & 0.05 & 0.05 && 0.05 & 0.05 & 0.04 \\ 
\rowcolor{gray!20}             &              && 0.10 &&  0.10 & 0.11 & 0.10 && 0.09 & 0.10 & 0.10 \\ 
\rowcolor{gray!20}             & Skewed       && 0.05 &&  0.04 & 0.04 & 0.04 && 0.04 & 0.04 & 0.05 \\ 
\rowcolor{gray!20}             &              && 0.10 &&  0.08 & 0.08 & 0.09 && 0.11 & 0.11 & 0.10 \\ 
             &&&&&&&&&&&\\
Heavy tailed & Normal       && 0.05 &&  0.17 & 0.15 & 0.12 && 0.34 & 0.31 & 0.26 \\ 
             &              && 0.10 &&  0.24 & 0.24 & 0.18 && 0.43 & 0.40 & 0.35 \\ 
             & Heavy tailed && 0.05 &&  0.21 & 0.20 & 0.16 && 0.37 & 0.34 & 0.29 \\ 
             &              && 0.10 &&  0.27 & 0.27 & 0.21 && 0.45 & 0.41 & 0.38 \\ 
             & Skewed       && 0.05 &&  0.17 & 0.16 & 0.12 && 0.35 & 0.32 & 0.25 \\ 
             &              && 0.10 &&  0.25 & 0.23 & 0.20 && 0.43 & 0.39 & 0.34 \\ 
             &&&&&&&&&&&\\
Skewed       & Normal       && 0.05 &&  0.17 & 0.16 & 0.12 && 0.32 & 0.26 & 0.20 \\ 
             &              && 0.10 &&  0.25 & 0.23 & 0.20 && 0.42 & 0.35 & 0.29 \\ 
             & Heavy tailed && 0.05 &&  0.17 & 0.16 & 0.12 && 0.35 & 0.27 & 0.19 \\ 
             &              && 0.10 &&  0.26 & 0.25 & 0.20 && 0.44 & 0.38 & 0.30 \\ 
             & Skewed       && 0.05 &&  0.17 & 0.14 & 0.11 && 0.33 & 0.27 & 0.19 \\ 
             &              && 0.10 &&  0.24 & 0.22 & 0.20 && 0.44 & 0.36 & 0.29 \\ 

\hline
\end{tabular}
\end{scriptsize}
\end{table}

\begin{table}[ht]
\centering
\caption{\label{tab:simb130}Proportion of tests rejecting normality of the random slope using two rotations and $s = 30$.}
\begin{scriptsize}
\begin{tabular}{ll p{.1cm} c p{.1cm} rrr p{.1cm} rrr}
  \hline
  \multicolumn{2}{c}{Distributions}& & Nominal & &  \multicolumn{3}{c}{Rotation} & & \multicolumn{3}{c}{Varimax rotation} \\ \cline{1-2} \cline{6-8} \cline{10-12}   
  Random effects & Errors & & $\alpha$ & & AD & CVM & KS & & AD & CVM & KS \\ 
   \hline
& && && \multicolumn{7}{c}{$\sigma_{\varepsilon}^2 = 4$, \ \ $\sigma_{b_0}^2 = \sigma_{b_1}^2 = 1$} \\ \cline{6-12}
\rowcolor{gray!20}Normal       & Normal       && 0.05 &&  0.05 & 0.05 & 0.04 && 0.06 & 0.06 & 0.06 \\ 
\rowcolor{gray!20}             &              && 0.10 &&  0.11 & 0.11 & 0.10 && 0.12 & 0.12 & 0.10 \\ 
\rowcolor{gray!20}             & Heavy tailed && 0.05 &&  0.08 & 0.07 & 0.06 && 0.12 & 0.10 & 0.09 \\ 
\rowcolor{gray!20}             &              && 0.10 &&  0.14 & 0.14 & 0.12 && 0.17 & 0.16 & 0.15 \\ 
\rowcolor{gray!20}             & Skewed       && 0.05 &&  0.06 & 0.05 & 0.04 && 0.06 & 0.06 & 0.06 \\ 
\rowcolor{gray!20}             &              && 0.10 &&  0.10 & 0.10 & 0.10 && 0.12 & 0.12 & 0.10 \\ 
             &&&&&&&&&&&\\
Heavy tailed & Normal       && 0.05 &&  0.07 & 0.06 & 0.06 && 0.10 & 0.09 & 0.07 \\ 
             &              && 0.10 &&  0.12 & 0.12 & 0.11 && 0.17 & 0.14 & 0.14 \\ 
             & Heavy tailed && 0.05 &&  0.14 & 0.13 & 0.11 && 0.23 & 0.19 & 0.17 \\ 
             &              && 0.10 &&  0.20 & 0.19 & 0.17 && 0.30 & 0.28 & 0.24 \\ 
             & Skewed       && 0.05 &&  0.07 & 0.06 & 0.06 && 0.14 & 0.13 & 0.11 \\ 
             &              && 0.10 &&  0.13 & 0.13 & 0.12 && 0.25 & 0.21 & 0.19 \\ 
             &&&&&&&&&&&\\
Skewed       & Normal       && 0.05 &&  0.06 & 0.06 & 0.05 && 0.09 & 0.08 & 0.08 \\ 
             &              && 0.10 &&  0.12 & 0.11 & 0.10 && 0.15 & 0.14 & 0.13 \\ 
             & Heavy tailed && 0.05 &&  0.12 & 0.11 & 0.08 && 0.18 & 0.15 & 0.11 \\ 
             &              && 0.10 &&  0.17 & 0.18 & 0.15 && 0.27 & 0.23 & 0.19 \\ 
             & Skewed       && 0.05 &&  0.09 & 0.08 & 0.07 && 0.13 & 0.11 & 0.09 \\ 
             &              && 0.10 &&  0.14 & 0.13 & 0.13 && 0.20 & 0.17 & 0.15 \\ 

&&&&&&&&&&&\\
& && && \multicolumn{7}{c}{$\sigma_{\varepsilon}^2 = 1$, \ \ $\sigma_{b_0}^2 = \sigma_{b_1}^2 = 1$} \\ \cline{6-12}
\rowcolor{gray!20}Normal       & Normal       && 0.05 &&  0.06 & 0.06 & 0.05 && 0.05 & 0.05 & 0.05 \\ 
\rowcolor{gray!20}             &              && 0.10 &&  0.10 & 0.10 & 0.10 && 0.09 & 0.09 & 0.10 \\ 
\rowcolor{gray!20}             & Heavy tailed && 0.05 &&  0.07 & 0.07 & 0.06 && 0.07 & 0.07 & 0.06 \\ 
\rowcolor{gray!20}             &              && 0.10 &&  0.13 & 0.13 & 0.12 && 0.14 & 0.13 & 0.12 \\ 
\rowcolor{gray!20}             & Skewed       && 0.05 &&  0.06 & 0.06 & 0.05 && 0.04 & 0.05 & 0.04 \\ 
\rowcolor{gray!20}             &              && 0.10 &&  0.11 & 0.10 & 0.11 && 0.10 & 0.09 & 0.09 \\ 
             &&&&&&&&&&&\\
Heavy tailed & Normal       && 0.05 &&  0.13 & 0.12 & 0.10 && 0.22 & 0.19 & 0.15 \\ 
             &              && 0.10 &&  0.18 & 0.18 & 0.16 && 0.28 & 0.26 & 0.23 \\ 
             & Heavy tailed && 0.05 &&  0.14 & 0.14 & 0.11 && 0.27 & 0.24 & 0.21 \\ 
             &              && 0.10 &&  0.21 & 0.19 & 0.17 && 0.35 & 0.33 & 0.28 \\ 
             & Skewed       && 0.05 &&  0.11 & 0.09 & 0.08 && 0.22 & 0.19 & 0.16 \\ 
             &              && 0.10 &&  0.17 & 0.16 & 0.14 && 0.30 & 0.28 & 0.24 \\ 
             &&&&&&&&&&&\\
Skewed       & Normal       && 0.05 &&  0.12 & 0.10 & 0.09 && 0.19 & 0.17 & 0.12 \\ 
             &              && 0.10 &&  0.19 & 0.18 & 0.15 && 0.27 & 0.24 & 0.20 \\ 
             & Heavy tailed && 0.05 &&  0.11 & 0.10 & 0.09 && 0.23 & 0.20 & 0.14 \\ 
             &              && 0.10 &&  0.18 & 0.16 & 0.14 && 0.32 & 0.28 & 0.22 \\ 
             & Skewed       && 0.05 &&  0.12 & 0.10 & 0.08 && 0.20 & 0.16 & 0.12 \\ 
             &              && 0.10 &&  0.19 & 0.17 & 0.13 && 0.28 & 0.24 & 0.20 \\ 


&&&&&&&&&&&\\
& && && \multicolumn{7}{c}{$\sigma_{\varepsilon}^2 = 1$, \ \ $\sigma_{b_0}^2 = \sigma_{b_1}^2 = 4$} \\ \cline{6-12}
\rowcolor{gray!20}Normal       & Normal       && 0.05 &&  0.05 & 0.05 & 0.05 && 0.05 & 0.05 & 0.04 \\ 
\rowcolor{gray!20}             &              && 0.10 &&  0.10 & 0.10 & 0.10 && 0.10 & 0.10 & 0.10 \\ 
\rowcolor{gray!20}             & Heavy tailed && 0.05 &&  0.04 & 0.04 & 0.04 && 0.05 & 0.06 & 0.06 \\ 
\rowcolor{gray!20}             &              && 0.10 &&  0.10 & 0.09 & 0.09 && 0.11 & 0.10 & 0.10 \\ 
\rowcolor{gray!20}             & Skewed       && 0.05 &&  0.05 & 0.05 & 0.04 && 0.05 & 0.05 & 0.05 \\ 
\rowcolor{gray!20}             &              && 0.10 &&  0.08 & 0.08 & 0.07 && 0.10 & 0.10 & 0.10 \\ 
             &&&&&&&&&&&\\
Heavy tailed & Normal       && 0.05 &&  0.17 & 0.16 & 0.12 && 0.30 & 0.28 & 0.23 \\ 
             &              && 0.10 &&  0.24 & 0.23 & 0.18 && 0.40 & 0.36 & 0.33 \\ 
             & Heavy tailed && 0.05 &&  0.19 & 0.17 & 0.14 && 0.34 & 0.31 & 0.25 \\ 
             &              && 0.10 &&  0.26 & 0.24 & 0.22 && 0.43 & 0.40 & 0.33 \\ 
             & Skewed       && 0.05 &&  0.16 & 0.14 & 0.12 && 0.32 & 0.29 & 0.23 \\ 
             &              && 0.10 &&  0.24 & 0.23 & 0.20 && 0.40 & 0.37 & 0.33 \\ 
             &&&&&&&&&&&\\
Skewed       & Normal       && 0.05 &&  0.17 & 0.15 & 0.12 && 0.31 & 0.25 & 0.19 \\ 
             &              && 0.10 &&  0.25 & 0.23 & 0.19 && 0.40 & 0.34 & 0.29 \\ 
             & Heavy tailed && 0.05 &&  0.15 & 0.14 & 0.12 && 0.30 & 0.25 & 0.17 \\ 
             &              && 0.10 &&  0.23 & 0.20 & 0.18 && 0.41 & 0.34 & 0.28 \\ 
             & Skewed       && 0.05 &&  0.15 & 0.14 & 0.13 && 0.30 & 0.22 & 0.17 \\ 
             &              && 0.10 &&  0.24 & 0.22 & 0.20 && 0.40 & 0.33 & 0.28 \\ 

\hline
\end{tabular}
\end{scriptsize}
\end{table}


%\section{Alternative rotation}\label{supp:simstudy-alt}
%
%In this section we present results of the tests of normality if we use an alternative $\bm{W}$.
%
%\include{alt_rotation_simulation_results}


%----------------------------------------------------------------------------------
%----------------------------------------------------------------------------------
\bibliographystyle{asa}
\bibliography{lcresid_bib}
%----------------------------------------------------------------------------------
%----------------------------------------------------------------------------------


\end{document}